\documentclass[12pt, a4paper]{article}

% Required Packages
\usepackage{amsmath}
\usepackage{amssymb}
\usepackage{kotex} % For Korean text
\usepackage{geometry}
\geometry{a4paper, margin=1in}

% Title
\title{Calculus - Chapter 10.3 Exercises}
\author{}
\date{}

\begin{document}
\maketitle
\hrulefill
\vspace{1em}

\subsection*{난이도 하}

\begin{enumerate}
    \item \textbf{Exercise 1:} Plot the point whose polar coordinates are given. Then find two other pairs of polar coordinates of this point, one with $r>0$ and one with $r<0$.
    \[ (a) (1, \pi/4) \quad (b) (-2, 3\pi/2) \quad (c) (3, -\pi/3) \]

    \item \textbf{Exercise 3:} Plot the point whose polar coordinates are given. Then find the Cartesian coordinates of the point.
    \[ (a) (2, 3\pi/2) \quad (b) (\sqrt{2}, \pi/4) \quad (c) (-1, -\pi/6) \]

    \item \textbf{Exercise 5:} The Cartesian coordinates of a point are given. Find polar coordinates $(r, \theta)$ of the point, where $r>0$ and $0 \le \theta < 2\pi$.
    \[ (a) (-4, 4) \quad (b) (3, 3\sqrt{3}) \]

    \item \textbf{Exercise 7:} Sketch the region in the plane consisting of points whose polar coordinates satisfy the given conditions.
    \[ 1 < r \le 3 \]

    \item \textbf{Exercise 17:} Identify the curve by finding a Cartesian equation for the curve.
    \[ r = 5\cos\theta \]

    \item \textbf{Exercise 21:} Find a polar equation for the curve represented by the given Cartesian equation.
    \[ x^2 + y^2 = 7 \]
\end{enumerate}

\hrulefill
\vspace{1em}

\subsection*{난이도 중}

\begin{enumerate}
    \setcounter{enumi}{6}
    \item \textbf{Exercise 33:} Sketch the curve with the given polar equation by first sketching the graph of $r$ as a function of $\theta$ in Cartesian coordinates.
    \[ r = -2\sin\theta \]

    \item \textbf{Exercise 35:} Sketch the curve with the given polar equation.
    \[ r = 2(1+\cos\theta) \]

    \item \textbf{Exercise 37:} Sketch the curve with the given polar equation.
    \[ r = \theta, \quad \theta \ge 0 \]

    \item \textbf{Exercise 39:} Sketch the curve with the given polar equation.
    \[ r = 3\cos(3\theta) \]

    \item \textbf{Exercise 41:} Sketch the curve with the given polar equation.
    \[ r = 2\cos(4\theta) \]

    \item \textbf{Exercise 45:} Sketch the curve with the given polar equation.
    \[ r^2 = 9\sin(2\theta) \]

    \item \textbf{Exercise 49:} Sketch the curve with the given polar equation.
    \[ r = \sin(\theta/2) \]

    \item \textbf{Exercise 23:} Find a polar equation for the curve represented by the given Cartesian equation.
    \[ y = \sqrt{3}x \]

    \item \textbf{Exercise 25:} Find a polar equation for the curve represented by the given Cartesian equation.
    \[ x^2 + y^2 = 4y \]

    \item \textbf{Exercise 29:} The figure shows a graph of $r$ as a function of $\theta$ in Cartesian coordinates. Use it to sketch the corresponding polar curve.

    \item \textbf{Exercise 31:} The figure shows a graph of $r$ as a function of $\theta$ in Cartesian coordinates. Use it to sketch the corresponding polar curve.

    \item \textbf{Exercise 56 (a, c):} Match the polar equations with the graphs labeled I-IX. Give reasons for your choices.
    \[ (a) \ r = \cos(3\theta) \quad (c) \ r = \cos(\theta/2) \]

    \item \textbf{Exercise 58:} Show that the curves $r = a\sin\theta$ and $r = a\cos\theta$ intersect at right angles.
\end{enumerate}

\hrulefill
\vspace{1em}

\subsection*{난이도 상}

\begin{enumerate}
    \setcounter{enumi}{19}
    \item \textbf{Exercise 51:} Show that the polar curve $r = 4 + 2\sec\theta$ (a conchoid) has the line $x=2$ as a vertical asymptote by showing that $\lim_{r \to \pm\infty} x = 2$. Use this fact to help sketch the conchoid.

    \item \textbf{Exercise 53:} Show that the curve $r = \sin\theta\tan\theta$ (a cissoid of Diocles) has the line $x=1$ as a vertical asymptote. Use this fact to help sketch the cissoid.

    \item \textbf{Exercise 55:} (a) In Example 10 the graphs suggest that the limaçon $r=1+c\sin\theta$ has an inner loop when $|c| > 1$. Prove that this is true, and find the values of $\theta$ that correspond to the inner loop. (b) From Figure 18 it appears that the limaçon loses its dimple when $c=1/2$. Prove this.

    \item \textbf{Exercise 57:} Show that the polar equation $r = a\sin\theta + b\cos\theta$, where $ab \ne 0$, represents a circle. Find its center and radius.

    \item \textbf{Exercise 66:} Use a graph to estimate the y-coordinate of the highest points on the curve $r=\sin(2\theta)$. Then use calculus to find the exact value.

    \item \textbf{Exercise 67:} Investigate the family of curves with polar equations $r=1+c\cos\theta$ where $c$ is a real number. How does the shape change as $c$ changes?
\end{enumerate}

\end{document}
