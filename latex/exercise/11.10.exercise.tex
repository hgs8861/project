\documentclass[12pt, a4paper]{article}

% Required Packages
\usepackage{amsmath}
\usepackage{amssymb}
\usepackage{amsfonts} % For \mathbb
\usepackage{kotex} % 한국어 사용을 위한 패키지
\usepackage{geometry}
\geometry{a4paper, margin=1in} % 페이지 여백 설정

% Title
\title{Chapter 11.10 Exercises: Taylor and Maclaurin Series}
\author{James Stewart, Calculus, Metric Edition}
\date{}

\begin{document}
\maketitle
\hrulefill % 제목 아래 수평선
\vspace{1em} % 제목과 내용 사이 간격

% --- 난이도 하 (Easy) ---
\subsection*{Difficulty: Easy (10 Problems)}
\begin{enumerate}
    \item \textbf{Exercise 3:} If $f^{(n)}(0) = (n+1)!$ for $n = 0, 1, 2, \dots$, find the Maclaurin series for $f$ and its radius of convergence.
    
    \item \textbf{Exercise 5:} Use the definition of a Taylor series to find the first four nonzero terms of the series for $f(x) = xe^x$ centered at $a=0$.
    
    \item \textbf{Exercise 8:} Use the definition of a Taylor series to find the first four nonzero terms of the series for $f(x) = \ln x$ centered at $a=1$.
    
    \item \textbf{Exercise 11:} Find the Maclaurin series for $f(x) = (1-x)^{-2}$ using the definition of a Maclaurin series. Also find the associated radius of convergence.
    
    \item \textbf{Exercise 13:} Find the Maclaurin series for $f(x) = \cos x$ using the definition of a Maclaurin series. Also find the associated radius of convergence.
    
    \item \textbf{Exercise 14:} Find the Maclaurin series for $f(x) = e^{-2x}$ using the definition of a Maclaurin series. Also find the associated radius of convergence.
    
    \item \textbf{Exercise 16:} Find the Maclaurin series for $f(x) = \sin 3x$ using the definition of a Maclaurin series. Also find the associated radius of convergence.
    
    \item \textbf{Exercise 39:} Use a Maclaurin series in Table 1 to obtain the Maclaurin series for the function $f(x) = \arctan(x^2)$.
    
    \item \textbf{Exercise 41:} Use a Maclaurin series in Table 1 to obtain the Maclaurin series for the function $f(x) = x \cos 2x$.
    
    \item \textbf{Exercise 83:} Find the sum of the series.
    \[ \sum_{n=0}^{\infty} \frac{(-1)^n}{n!} \]
\end{enumerate}

\hrulefill
\vspace{1em}

% --- 난이도 중 (Medium) ---
\subsection*{Difficulty: Medium (21 Problems)}
\begin{enumerate}
    \setcounter{enumi}{10} % 이전 목록 번호 이어서 시작
    \item \textbf{Exercise 6:} Use the definition of a Taylor series to find the first four nonzero terms of the series for $f(x) = \frac{1}{1+x}$ centered at $a=2$.
    
    \item \textbf{Exercise 7:} Use the definition of a Taylor series to find the first four nonzero terms of the series for $f(x) = \sqrt[3]{x}$ centered at $a=8$.
    
    \item \textbf{Exercise 9:} Use the definition of a Taylor series to find the first four nonzero terms of the series for $f(x) = \sin x$ centered at $a=\pi/6$.
    
    \item \textbf{Exercise 15:} Find the Maclaurin series for $f(x) = 2x^4 - 3x^2 + 3$ using the definition of a Maclaurin series. Also find the associated radius of convergence.
    
    \item \textbf{Exercise 17:} Find the Maclaurin series for $f(x) = 2^x$ using the definition of a Maclaurin series. Also find the associated radius of convergence.
    
    \item \textbf{Exercise 21:} Find the Taylor series for $f(x) = x^5 + 2x^3 + x$ centered at $a=2$.
    
    \item \textbf{Exercise 23:} Find the Taylor series for $f(x) = \ln x$ centered at $a=2$.
    
    \item \textbf{Exercise 25:} Find the Taylor series for $f(x) = e^{2x}$ centered at $a=3$.
    
    \item \textbf{Exercise 27:} Find the Taylor series for $f(x) = \sin x$ centered at $a=\pi$.
        
    \item \textbf{Exercise 35:} Use the binomial series to expand the function $f(x) = \sqrt[4]{1-x}$ as a power series. State the radius of convergence.
    
    \item \textbf{Exercise 37:} Use the binomial series to expand the function $f(x) = \dfrac{1}{(2+x)^3}$ as a power series. State the radius of convergence.
    
    \item \textbf{Exercise 44:} Use a Maclaurin series in Table 1 to obtain the Maclaurin series for the function $f(x) = x^2 \ln(1+x^3)$.
    
    \item \textbf{Exercise 47:} Use a Maclaurin series in Table 1 to obtain the Maclaurin series for the function $f(x) = \sin^2 x$. [Hint: Use $\sin^2 x = \frac{1}{2}(1 - \cos 2x)$.]
    
    \item \textbf{Exercise 51:} Find the Maclaurin series of $f(x) = \cos(x^2)$ and the associated radius of convergence. Graph $f$ and its first few Taylor polynomials on the same screen.
    
    \item \textbf{Exercise 55:} Use the Maclaurin series for $\cos x$ to compute $\cos 5^\circ$ correct to five decimal places.
    
    \item \textbf{Exercise 59:} Evaluate the indefinite integral as an infinite series.
    \[ \int \sqrt{1+x^3} dx \]
    
    \item \textbf{Exercise 63:} Use series to approximate the definite integral to within the indicated accuracy.
    \[ \int_{0}^{\frac{1}{2}} x^3 \arctan x dx \quad (\text{four decimal places}) \]
    
    \item \textbf{Exercise 67:} Use series to evaluate the limit.
    \[ \lim_{x \to 0} \frac{x - \ln(1+x)}{x^2} \]
    
    \item \textbf{Exercise 85:} Find the sum of the series.
    \[ \sum_{n=1}^{\infty} \frac{(-1)^{n-1} 3^n}{n 5^n} \]
\end{enumerate}

\hrulefill
\vspace{1em}

% --- 난이도 상 (Hard) ---
\subsection*{Difficulty: Hard (11 Problems)}
\begin{enumerate}
    \setcounter{enumi}{31} % 이전 목록 번호 이어서 시작
    \item \textbf{Exercise 29:} Find the Taylor series for $f(x) = \sin 2x$ centered at $a=\pi$.
    
    \item \textbf{Exercise 30:} Find the Taylor series for $f(x) = \sqrt{x}$ centered at $a=16$.
    
    \item \textbf{Exercise 48:} Find the Maclaurin series for $f(x) = \begin{cases} (x - \sin x)/x^3 & \text{if } x \neq 0 \\ 1/6 & \text{if } x = 0 \end{cases}$.
    
    \item \textbf{Exercise 50:} Use the formula $\tanh^{-1} x = \frac{1}{2} \ln\left(\frac{1+x}{1-x}\right)$ and the Maclaurin series for $\ln(1+x)$ to show that
    \[ \tanh^{-1} x = \sum_{n=0}^{\infty} \frac{x^{2n+1}}{2n+1} \]
    
    \item \textbf{Exercise 57:} (a) Use the binomial series to expand $1/\sqrt{1-x^2}$. (b) Use part (a) to find the Maclaurin series for $\sin^{-1} x$.
    
    \item \textbf{Exercise 61:} Evaluate the indefinite integral as an infinite series.
    \[ \int \frac{\cos x - 1}{x} dx \]
    
    \item \textbf{Exercise 71:} Use series to evaluate the limit.
    \[ \lim_{x \to 0} \frac{x^3 - 3x + 3 \tan^{-1} x}{x^5} \]
    
    \item \textbf{Exercise 74:} Use multiplication or division of power series to find the first three nonzero terms in the Maclaurin series for $y = \sec x$.
    
    \item \textbf{Exercise 77:} Use multiplication or division of power series to find the first three nonzero terms in the Maclaurin series for $y = (\arctan x)^2$.
    
    \item \textbf{Exercise 96:} (a) Show that the function defined by
    \[ f(x) = \begin{cases} e^{-1/x^2} & \text{if } x \neq 0 \\ 0 & \text{if } x = 0 \end{cases} \]
    is not equal to its Maclaurin series.
    (b) Graph the function in part (a) and comment on its behavior near the origin.
    
    \item \textbf{Exercise 97:} Use the following steps to prove Theorem 17 (Binomial Series).
    \begin{enumerate}
        \item[(a)] Let $g(x) = \sum_{n=0}^{\infty} \binom{k}{n} x^n$. Differentiate this series to show that
        \[ g'(x) = \frac{k g(x)}{1+x} \quad -1 < x < 1 \]
        \item[(b)] Let $h(x) = (1+x)^{-k} g(x)$ and show that $h'(x) = 0$.
        \item[(c)] Deduce that $g(x) = (1+x)^k$.
    \end{enumerate}
\end{enumerate}

\end{document}