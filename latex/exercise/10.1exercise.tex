\documentclass[12pt, a4paper]{article}

% Required Packages
\usepackage{amsmath}
\usepackage{amssymb}
\usepackage{kotex} % For Korean text
\usepackage{geometry}
\geometry{a4paper, margin=1in}

% Title
\title{Calculus - Chapter 10.1 Exercises}
\author{}
\date{}

\begin{document}
\maketitle
\hrulefill
\vspace{1em}

\subsection*{난이도 하}
\begin{enumerate}
    \item \textbf{Exercise 3:} Sketch the curve by using the parametric equations to plot points. Indicate with an arrow the direction in which the curve is traced as t increases.
    \[ x = 1 - t^2, \quad y = 2t - t^2, \quad -1 \le t \le 2 \]

    \item \textbf{Exercise 7:} (a) Sketch the curve by using the parametric equations to plot points. (b) Eliminate the parameter to find a Cartesian equation of the curve.
    \[ x = 2t - 1, \quad y = \frac{1}{2}t + 1 \]

    \item \textbf{Exercise 9:} (a) Sketch the curve. (b) Eliminate the parameter to find a Cartesian equation.
    \[ x = t^2 - 3, \quad y = t + 2, \quad -3 \le t \le 3 \]

    \item \textbf{Exercise 11:} (a) Sketch the curve. (b) Eliminate the parameter to find a Cartesian equation.
    \[ x = \sqrt{t}, \quad y = 1 - t \]

    \item \textbf{Exercise 13:} (a) Eliminate the parameter to find a Cartesian equation of the curve. (b) Sketch the curve and indicate the direction.
    \[ x = 3\cos t, \quad y = 3\sin t, \quad 0 \le t \le \pi \]

    \item \textbf{Exercise 21:} (a) Eliminate the parameter. (b) Sketch the curve.
    \[ x = \sin^2 t, \quad y = \cos^2 t \]
    
    \item \textbf{Exercise 37(b):} Find parametric equations to represent the line segment from $(-2, 7)$ to $(3, -1)$.
\end{enumerate}

\hrulefill
\vspace{1em}

\subsection*{난이도 중 }
\begin{enumerate}
    \setcounter{enumi}{7} % Continue numbering
    \item \textbf{Exercise 15:} (a) Eliminate the parameter. (b) Sketch the curve.
    \[ x = \cos\theta, \quad y = \sec^2\theta, \quad 0 \le \theta < \pi/2 \]

    \item \textbf{Exercise 19:} (a) Eliminate the parameter. (b) Sketch the curve.
    \[ x = \ln t, \quad y = \sqrt{t}, \quad t \ge 1 \]

    \item \textbf{Exercise 25:} Describe the motion of a particle with position $(x,y)$ as t varies in the given interval.
    \[ x = 5 + 2\cos(\pi t), \quad y = 3 + 2\sin(\pi t), \quad 1 \le t \le 2 \]

    \item \textbf{Exercise 27:} Describe the motion of a particle with position $(x,y)$ as t varies in the given interval.
    \[ x = 5\sin t, \quad y = 2\cos t, \quad -\pi \le t \le 5\pi \]

    \item \textbf{Exercise 34(a, b):} Match the parametric equations with the graphs labeled I-VI. Give reasons for your choices.
    \begin{itemize}
        \item[(a)] $x = t^4 - t + 1, \quad y = t^2$
        \item[(b)] $x = t^2 - 2t, \quad y = \sqrt{t}$
    \end{itemize}

    \item \textbf{Exercise 34(c, d):} Match the parametric equations with the graphs labeled I-VI. Give reasons for your choices.
    \begin{itemize}
        \item[(c)] $x = t^3 - 2t, \quad y = t^2 - t$
        \item[(d)] $x = \cos(5t), \quad y = \sin(2t)$
    \end{itemize}

    \item \textbf{Exercise 41(a, c):} Find parametric equations for the path of a particle that moves along the circle $x^2 + (y-1)^2 = 4$ in the manner described.
    \begin{itemize}
        \item[(a)] Once around clockwise, starting at $(2, 1)$.
        \item[(c)] Halfway around counterclockwise, starting at $(0, 3)$.
    \end{itemize}

    \item \textbf{Exercise 42(a):} Find parametric equations for the ellipse $\frac{x^2}{a^2} + \frac{y^2}{b^2} = 1$.

    \item \textbf{Exercise 45(b):} Sketch the graph of each curve and explain how the curves differ from one another.
    (i) $x=t^2, y=t$ \quad (ii) $x=t, y=\sqrt{t}$

    \item \textbf{Exercise 49:} A curve traced out by a point P at a distance d from the center of a circle of radius r as the circle rolls along a straight line is called a trochoid. Show that its parametric equations are:
    \[ x = r\theta - d\sin\theta, \quad y = r - d\cos\theta \]
    
    \item \textbf{Exercise 58(a):} If a gun is fired with $\alpha=30^{\circ}$ and $v_0 = 500$ m/s, when will the bullet hit the ground? How far from the gun will it hit the ground?
\end{enumerate}

\hrulefill
\vspace{1em}

\subsection*{난이도 상 }
\begin{enumerate}
    \setcounter{enumi}{18} % Continue numbering
    \item \textbf{Exercise 28:} Describe the motion of a particle with position $(x,y)$ as t varies in the given interval.
    \[ x = \sin t, \quad y = \cos^2 t, \quad -2\pi \le t \le 2\pi \]

    \item \textbf{Exercise 34(e, f):} Match the parametric equations with the graphs labeled I-VI.
    \begin{itemize}
        \item[(e)] $x = t + \sin(4t), \quad y = t^2 + \cos(3t)$
        \item[(f)] $x = t + \sin(2t), \quad y = t + \sin(3t)$
    \end{itemize}

    \item \textbf{Exercise 51:} Find parametric equations for the curve that consists of all possible positions of the point P in the figure, using the angle $\theta$ as the parameter.
    
    \item \textbf{Exercise 53:} A curve, called a witch of Maria Agnesi, consists of all possible positions of the point P in the figure. Show that parametric equations for this curve can be written as:
    \[ x = 2a\cot\theta, \quad y = 2a\sin^2\theta \]

    \item \textbf{Exercise 54:} Find parametric equations for the set of all points P as shown in the figure such that $|OP| = |AB|$. (This curve is called the cissoid of Diocles).

    \item \textbf{Exercise 56:} The position of one particle is $x=3\sin t, y=2\cos t$ and a second particle is $x=-3+\cos t, y=1+\sin t$ for $0 \le t \le 2\pi$.
    \begin{itemize}
        \item[(a)] Graph the paths. At how many points do they intersect?
        \item[(b)] Do the particles collide? If so, find the collision points.
    \end{itemize}

    \item \textbf{Exercise 59:} Investigate the family of curves defined by the parametric equations $x=t^2, y=t^3-ct$. How does the shape change as c increases?
\end{enumerate}

\end{document}
