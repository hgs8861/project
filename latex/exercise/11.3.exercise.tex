\documentclass[12pt, a4paper]{article}

% Required Packages
\usepackage{amsmath}
\usepackage{amssymb}
\usepackage{kotex}
\usepackage{geometry}
\geometry{a4paper, margin=1in}

% Title
\title{Chapter 11.3 Exercises: The Integral Test and Estimation of Sums}
\author{James Stewart, Calculus, Metric Edition}
\date{}

\begin{document}
\maketitle
\hrulefill
\vspace{1em}

\subsection*{Difficulty: Easy}
\begin{enumerate}
    \item \textbf{Exercise 3:} Use the Integral Test to determine whether the series is convergent or divergent.
    \[ \sum_{n=1}^{\infty} \dfrac{1}{n^4} \]

    \item \textbf{Exercise 5:} Use the Integral Test to determine whether the series is convergent or divergent.
    \[ \sum_{n=1}^{\infty} \dfrac{1}{(n+3)^{5/2}} \]

    \item \textbf{Exercise 7:} Use the Integral Test to determine whether the series is convergent or divergent.
    \[ \sum_{n=1}^{\infty} \dfrac{n}{n^2 + 1} \]

    \item \textbf{Exercise 9:} Use the Integral Test to determine whether the series is convergent or divergent.
    \[ \sum_{n=1}^{\infty} n^2 e^{-n^3} \]
    
    \item \textbf{Exercise 13:} Determine whether the series is convergent or divergent.
    \[ \sum_{n=1}^{\infty} \dfrac{1}{2n+3} \]
    
    \item \textbf{Exercise 15:} Determine whether the series is convergent or divergent.
    \[ \sum_{n=1}^{\infty} \dfrac{n^2}{n^3 + 1} \]
    
    \item \textbf{Exercise 16:} Determine whether the series is convergent or divergent.
    \[ \sum_{n=1}^{\infty} \dfrac{n}{n^4 + 1} \]
\end{enumerate}

\hrulefill
\vspace{1em}

\subsection*{Difficulty: Medium}
\begin{enumerate}
    \setcounter{enumi}{7}
    \item \textbf{Exercise 4:} Explain why the Integral Test can't be used to determine whether the series is convergent.
    \[ \sum_{n=1}^{\infty} \dfrac{n \sin n}{1 + n^2} \]
    
    \item \textbf{Exercise 10:} Use the Integral Test to determine whether the series is convergent or divergent.
    \[ \sum_{n=2}^{\infty} \dfrac{1}{n \ln n} \]

    \item \textbf{Exercise 11:} Use the Integral Test to determine whether the series is convergent or divergent.
    \[ \sum_{n=1}^{\infty} n e^{-n} \]
    
    \item \textbf{Exercise 17:} Determine whether the series is convergent or divergent.
    \[ \sum_{n=2}^{\infty} \dfrac{1}{n \sqrt{\ln n}} \]
    
    \item \textbf{Exercise 21:} Find the values of $p$ for which the series $\sum_{n=2}^{\infty} \dfrac{1}{n (\ln n)^p}$ is convergent.

    \item \textbf{Exercise 23(a):} Find the partial sum $s_{10}$ of the series $\sum_{n=1}^{\infty} 1/n^2$. Estimate the error in using $s_{10}$ as an approximation to the sum of the series.
    
    \item \textbf{Exercise 23(b):} Use (2) with $n=10$ to give an improved estimate of the sum.
    
    \item \textbf{Exercise 23(c):} Find a value of $n$ so that $s_n$ is within 0.0001 of the sum.
    
    \item \textbf{Exercise 27(a):} Use the sum of the first 10 terms to estimate the sum of the series $\sum_{n=2}^{\infty} \dfrac{1}{n(\ln n)^2}$. Estimate the error.
\end{enumerate}

\hrulefill
\vspace{1em}

\subsection*{Difficulty: Hard}
\begin{enumerate}
    \setcounter{enumi}{16}
    \item \textbf{Exercise 31:} How many terms of the series $\sum_{n=2}^{\infty} \dfrac{1}{n(\ln n)^2}$ would you need to add to find its sum to within 0.01?
    
    \item \textbf{Exercise 33:} If $\sum a_n$ is a convergent series with positive terms, is it true that $\sum \sin(a_n)$ is also convergent?

    \item \textbf{Exercise 35(a):} Show that $\sum_{n=2}^{\infty} \dfrac{1}{n (\ln n)^p}$ converges if $p > 1$ and diverges if $p \le 1$. (This is the $p$-series test for integrals.)
    
    \item \textbf{Exercise 37(a):} Find the partial sum $s_{10}$ of the series $\sum_{n=1}^{\infty} 1/n^4$. Estimate the error in using $s_{10}$ as an approximation to the sum of the series.
    
    \item \textbf{Exercise 37(b):} Use (2) with $n=10$ to give an improved estimate of the sum.
    
    \item \textbf{Exercise 40:} Find all positive values of $b$ for which the series $\sum_{n=1}^{\infty} b^{\ln n}$ converges.
\end{enumerate}

\end{document}
