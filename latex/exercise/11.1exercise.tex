\documentclass[12pt, a4paper]{article}

% Required Packages
\usepackage{amsmath}
\usepackage{amssymb}
\usepackage{kotex}
\usepackage{geometry}
\geometry{a4paper, margin=1in}

% Title
\title{Chapter 11.1 Exercises: Sequences}
\author{James Stewart, Calculus, Metric Edition}
\date{}

\begin{document}
\maketitle
\hrulefill
\vspace{1em}

\subsection*{Difficulty: Easy}
\begin{enumerate}
    \item \textbf{Exercise 3:} List the first five terms of the sequence.
    \[ a_n = n^3 - 1 \]

    \item \textbf{Exercise 4:} List the first five terms of the sequence.
    \[ a_n = \frac{1}{3n + 1} \]
    
    \item \textbf{Exercise 7:} List the first five terms of the sequence.
    \[ a_n = (-1)^{n-1} \frac{n}{2^n} \]
    
    \item \textbf{Exercise 8:} List the first five terms of the sequence.
    \[ a_n = \frac{(-1)^n}{4^n} \]

    \item \textbf{Exercise 13:} List the first five terms of the sequence.
    \[ a_1 = 1, \quad a_{n+1} = 2a_n + 1 \]
    
    \item \textbf{Exercise 14:} List the first five terms of the sequence.
    \[ a_1 = 6, \quad a_{n+1} = \frac{a_n}{n} \]

    \item \textbf{Exercise 17:} Find a formula for the general term $a_n$ of the sequence, assuming that the pattern of the first few terms continues.
    \[ \left\{ \frac{1}{2}, \frac{1}{4}, \frac{1}{6}, \frac{1}{8}, \frac{1}{10}, \dots \right\} \]
    
    \item \textbf{Exercise 18:} Find a formula for the general term $a_n$ of the sequence.
    \[ \left\{ 4, -1, \frac{1}{4}, -\frac{1}{16}, \frac{1}{64}, \dots \right\} \]

    \item \textbf{Exercise 19:} Find a formula for the general term $a_n$ of the sequence.
    \[ \left\{ -\frac{2}{3}, 2, -\frac{4}{3}, \frac{8}{9}, -\frac{16}{27}, \dots \right\} \]

    \item \textbf{Exercise 20:} Find a formula for the general term $a_n$ of the sequence.
    \[ \{ 5, 8, 11, 14, 17, \dots \} \]
\end{enumerate}

\hrulefill
\vspace{1em}

\subsection*{Difficulty: Medium}
\begin{enumerate}
    \setcounter{enumi}{10}
    \item \textbf{Exercise 27:} Determine whether the sequence converges or diverges. If it converges, find the limit.
    \[ a_n = \frac{5}{n+2} \]

    \item \textbf{Exercise 29:} Determine whether the sequence converges or diverges. If it converges, find the limit.
    \[ a_n = \frac{4n^2 - 3n}{2n^2 + 1} \]
    
    \item \textbf{Exercise 33:} Determine whether the sequence converges or diverges. If it converges, find the limit.
    \[ a_n = \frac{3^n}{7^{2n}} \]
    
    \item \textbf{Exercise 34:} Determine whether the sequence converges or diverges. If it converges, find the limit.
    \[ a_n = \frac{3\sqrt{n}}{\sqrt{n} + 2} \]

    \item \textbf{Exercise 37:} Determine whether the sequence converges or diverges. If it converges, find the limit.
    \[ a_n = \sqrt{\frac{1 + 4n^2}{1 + n^2}} \]

    \item \textbf{Exercise 38:} Determine whether the sequence converges or diverges. If it converges, find the limit.
    \[ a_n = \cos\left(\frac{n\pi}{n+1}\right) \]
    
    \item \textbf{Exercise 39:} Determine whether the sequence converges or diverges. If it converges, find the limit.
    \[ a_n = \frac{n^2}{\sqrt{n^3 + 4n}} \]
    
    \item \textbf{Exercise 40:} Determine whether the sequence converges or diverges. If it converges, find the limit.
    \[ a_n = e^{2n/(n+2)} \]
    
    \item \textbf{Exercise 41:} Determine whether the sequence converges or diverges. If it converges, find the limit.
    \[ a_n = \frac{(-1)^n}{2\sqrt{n}} \]
    
    \item \textbf{Exercise 42:} Determine whether the sequence converges or diverges. If it converges, find the limit.
    \[ a_n = \frac{(-1)^{n+1}n}{n + \sqrt{n}} \]
    
    \item \textbf{Exercise 43:} Determine whether the sequence converges or diverges. If it converges, find the limit.
    \[ \left\{ \frac{(2n-1)!}{(2n+1)!} \right\} \]
    
    \item \textbf{Exercise 44:} Determine whether the sequence converges or diverges. If it converges, find the limit.
    \[ \left\{ \frac{\ln n}{\ln(2n)} \right\} \]

    \item \textbf{Exercise 47:} Determine whether the sequence converges or diverges. If it converges, find the limit.
    \[ \{ n^2 e^{-n} \} \]
    
    \item \textbf{Exercise 48:} Determine whether the sequence converges or diverges. If it converges, find the limit.
    \[ a_n = \ln(n+1) - \ln n \]

    \item \textbf{Exercise 49:} Determine whether the sequence converges or diverges. If it converges, find the limit.
    \[ a_n = \frac{\cos^2 n}{2^n} \]

    \item \textbf{Exercise 51:} Determine whether the sequence converges or diverges. If it converges, find the limit.
    \[ a_n = n \sin(1/n) \]
    
    \item \textbf{Exercise 53:} Determine whether the sequence converges or diverges. If it converges, find the limit.
    \[ a_n = \left(1 + \frac{2}{n}\right)^n \]

    \item \textbf{Exercise 54:} Determine whether the sequence converges or diverges. If it converges, find the limit.
    \[ a_n = \sqrt[n]{n} \]
    
    \item \textbf{Exercise 61:} Determine whether the sequence converges or diverges. If it converges, find the limit.
    \[ a_n = \frac{n!}{2^n} \]

    \item \textbf{Exercise 62:} Determine whether the sequence converges or diverges. If it converges, find the limit.
    \[ a_n = \frac{(-3)^n}{n!} \]
\end{enumerate}

\hrulefill
\vspace{1em}

\subsection*{Difficulty: Hard}
\begin{enumerate}
    \setcounter{enumi}{30}
    \item \textbf{Exercise 70:} (a) Determine whether the sequence defined as follows is convergent or divergent:
    \[ a_1 = 1 \quad a_{n+1} = 4 - a_n \text{ for } n \ge 1 \]
    (b) What happens if the first term is $a_1 = 2$?

    \item \textbf{Exercise 75:} For what values of $r$ is the sequence $\{nr^n\}$ convergent?

    \item \textbf{Exercise 76:} (a) If $\{a_n\}$ is convergent, show that $\lim_{n \to \infty} a_{n+1} = \lim_{n \to \infty} a_n$.
    (b) A sequence $\{a_n\}$ is defined by $a_1 = 1$ and $a_{n+1} = 1/(1+a_n)$ for $n \ge 1$. Assuming that $\{a_n\}$ is convergent, find its limit.

    \item \textbf{Exercise 78:} Determine whether the sequence is increasing, decreasing, or not monotonic. Is the sequence bounded?
    \[ a_n = \cos n \]

    \item \textbf{Exercise 79:} Determine whether the sequence is increasing, decreasing, or not monotonic. Is the sequence bounded?
    \[ a_n = \frac{1}{2n+3} \]

    \item \textbf{Exercise 85:} Find the limit of the sequence
    \[ \{ \sqrt{2}, \sqrt{2\sqrt{2}}, \sqrt{2\sqrt{2\sqrt{2}}}, \dots \} \]
    
    \item \textbf{Exercise 86:} A sequence $\{a_n\}$ is given by $a_1 = \sqrt{2}$, $a_{n+1} = \sqrt{2 + a_n}$.
    (a) By induction or otherwise, show that $\{a_n\}$ is increasing and bounded above by 3. Apply the Monotonic Sequence Theorem to show that $\lim_{n \to \infty} a_n$ exists.
    (b) Find $\lim_{n \to \infty} a_n$.

    \item \textbf{Exercise 87:} Show that the sequence defined by
    \[ a_1 = 1 \quad a_{n+1} = 3 - \frac{1}{a_n} \]
    is increasing and $a_n < 3$ for all $n$. Deduce that $\{a_n\}$ is convergent and find its limit.
    
    \item \textbf{Exercise 89:} (a) Fibonacci posed the following problem: Suppose that rabbits live forever and that every month each pair produces a new pair which becomes productive at age 2 months. If we start with one newborn pair, how many pairs of rabbits will we have in the $n$th month? Show that the answer is $f_n$, where $\{f_n\}$ is the Fibonacci sequence.
    (b) Let $a_n = f_{n+1}/f_n$ and show that $a_{n-1} = 1 + 1/a_{n-2}$. Assuming that $\{a_n\}$ is convergent, find its limit.

    \item \textbf{Exercise 96:} Let $a_n = (1 + 1/n)^n$.
    (a) Show that if $0 \le a < b$, then $\frac{b^{n+1}-a^{n+1}}{b-a} < (n+1)b^n$.
    (b) Deduce that $b^n[(n+1)a - nb] < a^{n+1}$.
    (c) Use $a=1+1/(n+1)$ and $b=1+1/n$ in part (b) to show that $\{a_n\}$ is increasing.
    (d) Use $a=1$ and $b=1+1/(2n)$ in part (b) to show that $a_{2n} < 4$.
    (e) Use parts (c) and (d) to show that $a_n < 4$ for all $n$.
    (f) Use the Monotonic Sequence Theorem to show that $\lim_{n \to \infty} (1 + 1/n)^n$ exists.
\end{enumerate}

\end{document}