\documentclass[12pt, a4paper]{article}

% Required Packages
\usepackage{amsmath}
\usepackage{amssymb}
\usepackage{kotex} % For Korean text
\usepackage{geometry}
\geometry{a4paper, margin=1in}

% Title
\title{Calculus - Chapter 10.6 Exercises}
\author{}
\date{}

\begin{document}
\maketitle
\hrulefill
\vspace{1em}

\subsection*{난이도 하}

\begin{enumerate}
    \item \textbf{Exercise 1:} Write a polar equation of a conic with the focus at the origin and the given data. Parabola, directrix $x=2$.

    \item \textbf{Exercise 2:} Write a polar equation of a conic with the focus at the origin and the given data. Ellipse, eccentricity $1/3$, directrix $y=6$.

    \item \textbf{Exercise 15:} (a) Find the eccentricity, (b) identify the conic, (c) give an equation of the directrix, and (d) sketch the conic.
    \[ r = \frac{4}{5 - 4\sin\theta} \]

    \item \textbf{Exercise 17:} (a) Find the eccentricity, (b) identify the conic, (c) give an equation of the directrix, and (d) sketch the conic.
    \[ r = \frac{2}{3 + 3\sin\theta} \]
\end{enumerate}

\hrulefill
\vspace{1em}

\subsection*{난이도 중}

\begin{enumerate}
    \setcounter{enumi}{4}
    \item \textbf{Exercise 5:} Write a polar equation of a conic with the focus at the origin and the given data. Ellipse, eccentricity $2/3$, vertex $(2, \pi)$.

    \item \textbf{Exercise 7:} Write a polar equation of a conic with the focus at the origin and the given data. Parabola, vertex $(3, \pi/2)$.

    \item \textbf{Exercise 21:} (a) Find the eccentricity, (b) identify the conic, (c) give an equation of the directrix, and (d) sketch the conic.
    \[ r = \frac{3}{4 - 8\cos\theta} \]

    \item \textbf{Exercise 23:} (a) Find the eccentricity and directrix of the conic $r = 1/(1 - 2\sin\theta)$ and graph the conic and its directrix. (b) If this conic is rotated counterclockwise about the origin through an angle $3\pi/4$, write the resulting equation and graph its curve.

    \item \textbf{Exercise 25:} Graph the conics $r = \dfrac{e}{1 - e\cos\theta}$ with $e = 0.4, 0.6, 0.8,$ and $1.0$ on a common screen. How does the value of $e$ affect the shape of the curve?

    \item \textbf{Exercise 31:} The orbit of Mars around the sun is an ellipse with eccentricity 0.093 and semimajor axis $2.28 \times 10^8$ km. Find a polar equation for the orbit.

    \item \textbf{Exercise 35:} The planet Mercury travels in an elliptical orbit with eccentricity 0.206. Its minimum distance from the sun is $4.6 \times 10^7$ km. Find its maximum distance from the sun.
\end{enumerate}

\hrulefill
\vspace{1em}

\subsection*{난이도 상}

\begin{enumerate}
    \setcounter{enumi}{11}
    \item \textbf{Exercise 27:} Show that a conic with focus at the origin, eccentricity $e$, and directrix $x = -d$ has polar equation
    \[ r = \frac{ed}{1 - e\cos\theta} \]

    \item \textbf{Exercise 30:} Show that the parabolas $r = c/(1 + \cos\theta)$ and $r = d/(1 - \cos\theta)$ intersect at right angles.
    
    \item \textbf{Exercise 33:} The orbit of Halley's comet, last seen in 1986 and due to return in 2061, is an ellipse with eccentricity 0.97 and one focus at the sun. The length of its major axis is 36.18 AU. Find a polar equation for the orbit of Halley's comet. What is the maximum distance from the comet to the sun?

    \item \textbf{Exercise 37:} Using the data from Exercise 35, find the distance traveled by the planet Mercury during one complete orbit around the sun. (Evaluate the resulting definite integral numerically.)
\end{enumerate}

\end{document}
