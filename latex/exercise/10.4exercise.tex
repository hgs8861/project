% Required Packages
\usepackage{amsmath}
\usepackage{amssymb}
\usepackage{kotex} % For Korean text
\usepackage{geometry}
\geometry{a4paper, margin=1in}

% Title
\title{Calculus - Chapter 10.4 Exercises}
\author{}
\date{}

\begin{document}
\maketitle
\hrulefill
\vspace{1em}

\subsection*{Difficulty: Easy}

\begin{enumerate}
    \item \textbf{Exercise 1:} Find the area of the region that is bounded by the given curve and lies in the specified sector.
    \[ r = \sqrt{2\theta}, \quad 0 \le \theta \le \pi/2 \]

    \item \textbf{Exercise 9:} Sketch the curve and find the area that it encloses.
    \[ r = 4\cos\theta \]

    \item \textbf{Exercise 17:} Find the area of the region enclosed by one loop of the curve.
    \[ r = 4\cos(3\theta) \]

    \item \textbf{Exercise 23:} Find the area of the region that lies inside the first curve and outside the second curve.
    \[ r = 4\sin\theta, \quad r = 2 \]

    \item \textbf{Exercise 29:} Find the area of the region that lies inside both curves.
    \[ r = 3\sin\theta, \quad r = 3\cos\theta \]

    \item \textbf{Exercise 49:} Find the exact length of the polar curve.
    \[ r = 2\cos\theta, \quad 0 \le \theta \le \pi \]

    \item \textbf{Exercise 63:} Find the slope of the tangent line to the given polar curve at the point specified by the value of $\theta$.
    \[ r = 2\cos\theta, \quad \theta = \pi/3 \]
\end{enumerate}

\hrulefill
\vspace{1em}

\subsection*{Difficulty: Medium}

\begin{enumerate}
    \setcounter{enumi}{7}
    \item \textbf{Exercise 11:} Sketch the curve and find the area that it encloses.
    \[ r = 3 - 2\sin\theta \]

    \item \textbf{Exercise 13:} Graph the curve and find the area that it encloses.
    \[ r = 2 + \sin(4\theta) \]

    \item \textbf{Exercise 21:} Find the area of the region enclosed by the inner loop of the curve.
    \[ r = 1 + 2\sin\theta \]
    
    \item \textbf{Exercise 27:} Find the area of the region that lies inside the first curve and outside the second curve.
    \[ r = 3\cos\theta, \quad r = 1 + \cos\theta \]

    \item \textbf{Exercise 31:} Find the area of the region that lies inside both curves.
    \[ r = \sin(2\theta), \quad r = \cos(2\theta) \]

    \item \textbf{Exercise 35:} Find the area inside the larger loop and outside the smaller loop of the limaçon $r = \frac{1}{2} + \cos\theta$.

    \item \textbf{Exercise 37:} Find all points of intersection of the given curves.
    \[ r = \sin\theta, \quad r = 1 - \sin\theta \]

    \item \textbf{Exercise 39:} Find all points of intersection of the given curves.
    \[ r = 2\sin(2\theta), \quad r = 1 \]

    \item \textbf{Exercise 51:} Find the exact length of the polar curve.
    \[ r = \theta^2, \quad 0 \le \theta \le 2\pi \]

    \item \textbf{Exercise 52:} Find the exact length of the polar curve.
    \[ r = 2(1 + \cos\theta) \]

    \item \textbf{Exercise 55:} Find the exact length of the curve. Use a graph to determine the parameter interval.
    \[ r = \cos^4(\theta/4) \]

    \item \textbf{Exercise 65:} Find the slope of the tangent line to the given polar curve at the point specified by the value of $\theta$.
    \[ r = 1/\theta, \quad \theta = \pi \]

    \item \textbf{Exercise 67:} Find the slope of the tangent line to the given polar curve at the point specified by the value of $\theta$.
    \[ r = \cos(2\theta), \quad \theta = \pi/4 \]

    \item \textbf{Exercise 69:} Find the points on the given curve where the tangent line is horizontal or vertical.
    \[ r = \sin\theta \]

    \item \textbf{Exercise 71:} Find the points on the given curve where the tangent line is horizontal or vertical.
    \[ r = 1 + \cos\theta \]
    
    \item \textbf{Exercise 72:} Find the points on the given curve where the tangent line is horizontal or vertical.
    \[ r = e^\theta \]
\end{enumerate}

\hrulefill
\vspace{1em}

\subsection*{Difficulty: Hard}

\begin{enumerate}
    \setcounter{enumi}{23}
    \item \textbf{Exercise 22:} Find the area enclosed by the loop of the strophoid $r = 2\cos\theta - \sec\theta$.

    \item \textbf{Exercise 36:} Find the area between a large loop and the enclosed small loop of the curve $r = 1 + 2\cos(3\theta)$.

    \item \textbf{Exercise 41:} Find all points of intersection of the given curves.
    \[ r^2 = 2\cos(2\theta), \quad r = 1 \]

    \item \textbf{Exercise 47:} The points of intersection of the cardioid $r=1+\sin\theta$ and the spiral loop $r=2\theta, -\pi/2 \le \theta \le \pi/2$, can't be found exactly. Use a graph to find the approximate values of $\theta$ at which the curves intersect. Then use these values to estimate the area that lies inside both curves.

    \item \textbf{Exercise 48:} When recording live performances, sound engineers often use a microphone with a cardioid pickup pattern... Suppose the microphone is placed 4 m from the front of the stage and the boundary of the optimal pickup region is given by the cardioid $r = 8 + 8\sin\theta$... The musicians want to know the area they will have on stage within the optimal pickup range... Answer their question.

    \item \textbf{Exercise 73:} Let P be any point (except the origin) on the curve $r=f(\theta)$. If $\psi$ is the angle between the tangent line at P and the radial line OP, show that
    \[ \tan\psi = \frac{r}{dr/d\theta} \]

    \item \textbf{Exercise 75:} (a) Use Formula 10.2.9 to show that the area of the surface generated by rotating the polar curve $r=f(\theta), a \le \theta \le b$ about the polar axis is
    \[ S = \int_a^b 2\pi r \sin\theta \sqrt{r^2 + \left(\frac{dr}{d\theta}\right)^2} d\theta \]
    (b) Use the formula in part (a) to find the surface area generated by rotating the lemniscate $r^2 = \cos(2\theta)$ about the polar axis.
\end{enumerate}

\end{document}