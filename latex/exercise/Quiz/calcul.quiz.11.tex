\documentclass{exam}
\addpoints % Enable the grade table

% --- 한글 사용을 위한 패키지 설정 ---
\usepackage[utf8]{inputenc}
\usepackage{kotex}
\usepackage{amsmath}
\usepackage{amssymb}
\usepackage{tcolorbox}

\begin{document} % (10점)
\begin{center}
    \bfseries\Large 퀴즈 2차 (현경서t) 

    \bigskip
    \large 미적분학1  
    
    \bigskip
    % 이름과 학번을 적는 란을 만듭니다.
    \makebox[0.2\textwidth]{반: \hrulefill} \hspace{1cm}
    \makebox[0.2\textwidth]{번호: \hrulefill} \hspace{1cm}
    \makebox[0.4\textwidth]{이름: \hrulefill}
\end{center}

\bigskip % 간격 띄우기
\begin{questions}

% \item 대신 \question 명령어를 사용합니다.
% \question[배점] 형식으로 배점을 자동으로 관리할 수 있습니다.
\question[10]
Prove the next theorem.
\begin{tcolorbox}
    Suppose \(f\) is a continuous, positive, decreasing function on \([1, \infty]\) and let \(a_n = f(n)\). 
    \begin{itemize}
        \item[(i)] If \( \int_1^\infty f(x) \,dx \) is \textbf{convergent}, then \( \sum_{n=1}^{\infty} a_n \) is \textbf{convergent}.
        \item[(ii)] If \( \int_1^\infty f(x) \,dx \) is \textbf{divergent}, then \( \sum_{n=1}^{\infty} a_n \) is \textbf{divergent}.
    \end{itemize}
\end{tcolorbox}

\vspace{7cm}

\question[10]
Prove the next theorem.
\begin{tcolorbox}
    Suppose that \( \sum a_n \) and \( \sum b_n \) are series with positive terms.
    \begin{itemize}
        \item[(i)] If \( \sum b_n \) is \textbf{convergent} and \( a_n \le b_n \) for all \( n \), then \( \sum a_n \) is also \textbf{convergent}.
        \item[(ii)] If \( \sum b_n \) is \textbf{divergent} and \( a_n \ge b_n \) for all \( n \), then \( \sum a_n \) is also \textbf{divergent}.
    \end{itemize}
\end{tcolorbox}
    
\vspace{6cm}

\question[10]
Prove the next theorem.
\begin{tcolorbox}
    If the alternating series
    $\sum_{n=1}^{\infty} (-1)^{n-1}b_n (b_n > 0)$ satisfies
    \begin{itemize}
        \item[(i)] \( b_{n+1} \le b_n \) for all \(n\)
        \item[(ii)] \( \lim_{n\to\infty} b_n = 0 \)
    \end{itemize}
    then the series is convergent.
\end{tcolorbox}

\vspace{7cm}

\question[10]
Prove the next theorem.
\begin{tcolorbox}
    Let \( \sum a_n \) be a series and suppose that the following limit exists:
    \[ \lim_{n\to\infty} \left| \dfrac{a_{n+1}}{a_n} \right| = L \]
    \begin{itemize}
        \item[(i)] If \(L < 1\), then the series \( \sum a_n \) is absolutely convergent.
        \item[(ii)] If \(L > 1\) or \(L = \infty\), then the series \( \sum a_n \) is divergent.
    \end{itemize}
\end{tcolorbox}
    
\vspace{7cm}

\question[10]
Find the Macalurin seires for next function.
\begin{parts}
    \part[2] $\dfrac{1}{1-x}=$ \\
    \part[2] $e^x=$ \\
    \part[2] $\sin x=$ \\
    \part[2] $\cos x=$ \\
    \part[2] $(1+x)^k=$ \\
\end{parts}
    
% \end{questions}로 문제 환경을 종료합니다.
\end{questions}

\vfill % 페이지 하단으로 내용을 밀어냄

\centerline{\bfseries --- 수고하셨습니다 ---}

\end{document}