\documentclass[10pt, a4paper]{article}

% --- 1. 필수 패키지 설정 ---
\usepackage[a4paper, top=18mm, bottom=15mm, left=12mm, right=12mm, headheight=25pt, headsep=8mm]{geometry}
\usepackage{amsmath, amssymb, amsthm}
\usepackage{fancyhdr}
\usepackage{enumitem}
\usepackage{array}
\usepackage{longtable}
\usepackage{kotex}

% --- 2. 헤더 설정 ---
\pagestyle{fancy}
\fancyhf{}
\lhead{\large \textbf{미적분 : 제15장 정적분의 계산 연습문제}}
\rhead{\small 학번: \underline{\hspace{3.5cm}} 이름: \underline{\hspace{3cm}}}
\cfoot{\small \thepage}
\renewcommand{\headrulewidth}{0.6pt}

% --- 3. 문제 박스 명령어 정의 ---
\newcounter{probcount}
\newcommand{\exercise}[1]{%
    \stepcounter{probcount}%
    \begin{minipage}[t][110mm][t]{0.44\textwidth}
        \small 
        \vspace{1mm}
        \noindent \textbf{문제 \arabic{probcount}.} #1
    \end{minipage}%
}

\begin{document}

\noindent
\begin{longtable}{p{0.47\textwidth} | p{0.47\textwidth}}
    
    % [1페이지 상단]
    \exercise{다음 정적분의 값을 구하여라.
    \begin{enumerate}[label=(\arabic*), leftmargin=8mm, nosep]
        \item $\displaystyle \int_{0}^{1} \frac{x^2}{x+1} dx$
        \item $\displaystyle \int_{0}^{1} x\sqrt{1-x^2} dx$
        \item $\displaystyle \int_{0}^{\pi/4} \tan^2 x dx$
    \end{enumerate}} 
    & 
    \exercise{함수 $f(x)$가 모든 실수 $x, y$에 대하여 $f(x+y)=f(x)f(y)$를 만족시키고 $f'(0)=1, f(0) \ne 0$일 때, $\displaystyle \int_{0}^{1} f(x) dx$의 값을 구하여라.} \\
    \hline

    % [1페이지 하단]
    \exercise{다음 정적분의 값을 구하여라.
    \begin{enumerate}[label=(\arabic*), leftmargin=8mm, nosep]
        \item $\displaystyle \int_{0}^{\pi} e^x \sin x dx$
        \item $\displaystyle \int_{1}^{e} \frac{\ln x}{x^2} dx$
    \end{enumerate}} 
    & 
    \exercise{연속함수 $f(x)$가 모든 실수 $x$에 대하여 $f(x)+f(-x)=\cos x$를 만족시킬 때, $\displaystyle \int_{-\pi/2}^{\pi/2} f(x) dx$의 값을 구하여라.} \\

    % [2페이지 상단]
    \exercise{다음을 만족시키는 연속함수 $f(x)$를 구하여라.
    \[ f(x) = \sin x + \int_{0}^{\pi/2} f(t) \cos t dt \]} 
    & 
    \exercise{정적분 $\displaystyle \int_{0}^{1} \frac{1}{x^2+x+1} dx$의 값을 구하여라.} \\
     \hline

    % [2페이지 하단]
    \exercise{$I_n = \displaystyle \int_{0}^{1} x^n e^x dx$에 대하여 다음 물음에 답하여라.
    \begin{enumerate}[label=(\arabic*), leftmargin=8mm, nosep]
        \item $I_n$과 $I_{n-1}$ 사이의 관계식을 구하여라.
        \item $I_4$의 값을 구하여라.
    \end{enumerate}} 
    & 
    \exercise{모든 실수 $x$에 대하여 $f(x+2)=f(x)$이고, $-1 \le x \le 1$에서 $f(x)=x^2$일 때, $\displaystyle \int_{0}^{10} f(x) dx$의 값을 구하여라.} \\

\end{longtable}

\end{document}