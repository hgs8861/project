\documentclass[10pt, a4paper]{article}

% --- 1. 필수 패키지 설정 ---
\usepackage[a4paper, top=18mm, bottom=15mm, left=12mm, right=12mm, headheight=25pt, headsep=8mm]{geometry}
\usepackage{amsmath, amssymb, amsthm}
\usepackage{fancyhdr}
\usepackage{enumitem}
\usepackage{array}
\usepackage{longtable}
\usepackage{kotex}

% --- 2. 헤더 설정 ---
\pagestyle{fancy}
\fancyhf{}
\lhead{\large \textbf{미적분 : 제15장 정적분의 계산 연습문제}}
\rhead{\small 학번: \underline{\hspace{3.5cm}} 이름: \underline{\hspace{3cm}}}
\cfoot{\small \thepage}
\renewcommand{\headrulewidth}{0.6pt}

% --- 3. 문제 박스 명령어 정의 ---
\newcounter{probcount}
\newcommand{\exercise}[1]{%
    \stepcounter{probcount}%
    \begin{minipage}[t][110mm][t]{0.44\textwidth}
        \small 
        \vspace{1mm}
        \noindent \textbf{문제 \arabic{probcount}.} #1
    \end{minipage}% 
    }

\begin{document}

\noindent
\begin{longtable}{p{0.47\textwidth} | p{0.47\textwidth}}
    
    % [기본 문제 섹션 - 6문항 선별]
    \exercise{다음 정적분의 값을 구하여라.
    \begin{enumerate}[label=(\arabic*), leftmargin=8mm, nosep]
        \item $\displaystyle \int_{0}^{2} x(2-x)^5 dx$
        \item $\displaystyle \int_{0}^{\pi/2} \frac{\cos x}{1+\sin x} dx$
    \end{enumerate}} 
    & 
    \exercise{정적분 $\displaystyle \int_{0}^{1} \frac{1}{x^2+1} dx$ 의 값을 구하여라.} \\
    \hline 
    
    \exercise{다음 정적분의 값을 구하여라.
    \begin{enumerate}[label=(\arabic*), leftmargin=8mm, nosep]
        \item $\displaystyle \int_{0}^{\pi} x \sin x dx$
        \item $\displaystyle \int_{1}^{e} \ln x dx$
    \end{enumerate}} 
    & 
    \exercise{함수 $f(x) = \displaystyle \int_{0}^{\pi/2} | \sin t - x | dt$ 의 최솟값을 구하여라.} \\
    
    \exercise{다음을 만족시키는 연속함수 $f(x)$를 구하여라.
    \[ f(x) = e^x + \int_{0}^{1} t f(t) dt \]} 
    & 
    \exercise{모든 실수 $x$에 대하여 $f(x) = f(x+2)$이고, $-1 \le x \le 1$에서 $f(x)=x^2$일 때, $\displaystyle \int_{0}^{6} f(x) dx$의 값을 구하여라.} \\
    \hline

    % [실력 문제 섹션 - 5문항 선별]
    \exercise{다음 정적분의 값을 구하여라.
    \begin{enumerate}[label=(\arabic*), leftmargin=8mm, nosep]
        \item $\displaystyle \int_{0}^{1} \frac{1}{(1+x^2)^2} dx$
        \item $\displaystyle \int_{0}^{\pi} e^x \cos^2 x dx$
    \end{enumerate}} 
    & 
    \exercise{연속함수 $f(x)$가 $\displaystyle \int_{0}^{x} f(t) dt = \frac{x}{2} \{ f(0) + f(x) \}$ 를 만족시킬 때, $f(x)$는 일차함수임을 증명하여라.} \\
    
    \exercise{자연수 $n$에 대하여 $I_n = \displaystyle \int_{0}^{\pi/2} \sin^n x dx$ 라 할 때, $I_6$와 $I_7$의 값을 각각 구하여라.} 
    & 
    \exercise{함수 $f(x) = \displaystyle \int_{0}^{x} \frac{dt}{1+t^2}$ 일 때, $f(x) + f(1/x)$ 의 값을 구하여라. ($x>0$)} \\
    \hline

    \exercise{미분가능한 함수 $f(x)$가 $f(x) = x + \displaystyle \int_{0}^{1} e^t f(x-t) dt$ 를 만족시킬 때, $f(x)$를 구하여라.} 
    & 
    \\ % 빈 칸으로 유지하여 균형을 맞춤

\end{longtable}

\end{document}