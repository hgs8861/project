\documentclass[10pt, a4paper]{article}

% --- 1. 필수 패키지 설정 ---
\usepackage[a4paper, top=18mm, bottom=15mm, left=12mm, right=12mm, headheight=25pt, headsep=8mm]{geometry}
\usepackage{amsmath, amssymb, amsthm}
\usepackage{fancyhdr}
\usepackage{enumitem}
\usepackage{array}
\usepackage{longtable}
\usepackage{kotex}

% --- 2. 헤더 설정 ---
\pagestyle{fancy}
\fancyhf{}
\lhead{\large \textbf{미적분학 1 : 제15장 연습문제}}
\rhead{\small 학번: \underline{\hspace{3.5cm}} 이름: \underline{\hspace{3cm}}}
\cfoot{\small \thepage}
\renewcommand{\headrulewidth}{0.6pt}

% --- 3. 문제 박스 명령어 정의 ---
\newcounter{probcount}
\newcommand{\exercise}[1]{%
    \stepcounter{probcount}%
    \begin{minipage}[t][110mm][t]{0.44\textwidth}
        \small 
        \vspace{1mm}
        \noindent \textbf{문제 \arabic{probcount}.} #1
    \end{minipage}%
}

\begin{document}

\noindent
\begin{longtable}{p{0.47\textwidth} | p{0.47\textwidth}}
    
    % [기본문제 선별]
    \exercise{다음 정적분의 값을 구하여라.
    \begin{enumerate}[label=(\arabic*), leftmargin=8mm, nosep]
        \item $\displaystyle \int_{0}^{1} x(1-x)^9 dx$
        \item $\displaystyle \int_{0}^{1} \frac{e^x}{e^x+1} dx$
        \item $\displaystyle \int_{1}^{e} \frac{(\ln x)^3}{x} dx$
    \end{enumerate}} 
    & 
    \exercise{다음 정적분의 값을 구하여라.
    \begin{enumerate}[label=(\arabic*), leftmargin=8mm, nosep]
        \item $\displaystyle \int_{0}^{\pi} x^2 \sin x dx$
        \item $\displaystyle \int_{0}^{1} \ln(x+1) dx$
    \end{enumerate}} \\
    \hline 
    
    \exercise{삼각치환법을 이용하여 다음 정적분의 값을 구하여라.
    \begin{enumerate}[label=(\arabic*), leftmargin=8mm, nosep]
        \item $\displaystyle \int_{0}^{1} \sqrt{1-x^2} dx$
        \item $\displaystyle \int_{0}^{\sqrt{3}} \frac{1}{x^2+3} dx$
    \end{enumerate}} 
    & 
    \exercise{연속함수 $f(x)$가 모든 실수 $x$에 대하여 $f(x)+f(1-x)=1$을 만족시킬 때, 정적분 $\displaystyle \int_{0}^{1} f(x) dx$의 값을 구하여라.} \\
   

    % [실력문제 선별]
    \exercise{함수 $f(x)$가 $f(x) = \sin x + \displaystyle \int_{0}^{\pi/2} f(t) \cos t dt$ 를 만족시킬 때, $f(x)$를 구하여라.} 
    & 
    \exercise{정적분 $\displaystyle \int_{0}^{\pi} e^x \cos^2 x dx$ 의 값을 구하여라.} \\
    \hline 
    
    \exercise{자연수 $n$에 대하여 $I_n = \displaystyle \int_{0}^{\pi/2} \sin^n x dx$ 라 할 때, $I_n = \frac{n-1}{n} I_{n-2}$ 임을 이용하여 $I_6$의 값을 구하여라.} 
    & 
    \exercise{함수 $f(x)$가 $f(x) = \displaystyle \int_{1}^{x} \frac{\ln t}{1+t} dt$ 일 때, $f(x) + f(1/x)$ 를 간단히 하여라.} \\

\end{longtable}

\end{document}