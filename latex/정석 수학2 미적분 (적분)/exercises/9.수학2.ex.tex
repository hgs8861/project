\documentclass[a4paper,11pt]{article}
\usepackage{kotex}
\usepackage{amsmath, amssymb}
\usepackage{geometry}
\geometry{left=2.5cm, right=2.5cm, top=2.5cm, bottom=2.5cm}
\linespread{1.3} % 줄 간격 1.3배
\setlength{\parskip}{1em} % 문단 간격

\title{\textbf{수학 II - Chapter 9 연습문제}}
\date{}

\begin{document}
\maketitle

\section*{기본 문제}

\begin{enumerate}
    \item \textbf{Exercise 9-1} \\
    다음 등식을 만족시키는 함수 $f(x)$를 구하여라.
    \[ \frac{d}{dx} \int f(x) dx = x^2 - 4x + 3 \]

    \item \textbf{Exercise 9-2} \\
    함수 $f(x)$에 대하여 $f'(x) = (x+1)(x^2-x+1)$이고 $f(0)=2$일 때, $f(x)$를 구하여라.

    \item \textbf{Exercise 9-3} \\
    다항함수 $f(x)$가 $\int f(x) dx = x f(x) - 2x^3 + x^2$을 만족시킬 때, $f(1)$의 값을 구하여라.

    \item \textbf{Exercise 9-4} \\
    곡선 $y=f(x)$ 위의 임의의 점 $P(x, y)$에서의 접선의 기울기가 $3x^2-6x$이고, 이 곡선이 $x$축에 접할 때, $f(x)$를 구하여라.

    \item \textbf{Exercise 9-5} \\
    $f'(x) = |x-1|$이고 $f(0)=1$인 함수 $f(x)$에 대하여 $f(2)$의 값을 구하여라.
\end{enumerate}

\section*{실력 문제}

\begin{enumerate}
    \item \textbf{Exercise 9-6} \\
    미분가능한 함수 $f(x)$가 임의의 실수 $x, y$에 대하여 $f(x+y) = f(x) + f(y) + 2xy$를 만족시키고 $f'(0)=1$일 때, $f(3)$의 값을 구하여라.

    \item \textbf{Exercise 9-7} \\
    두 다항함수 $f(x), g(x)$가 다음 조건을 만족시킨다.
    \[ \frac{d}{dx}\{f(x) + g(x)\} = 2, \quad \frac{d}{dx}\{f(x)g(x)\} = 2x - 5 \]
    $f(0)=2, g(0)=-1$일 때, $f(x)$와 $g(x)$를 구하여라.

    \item \textbf{Exercise 9-8} \\
    함수 $f(x)$의 도함수 $y=f'(x)$의 그래프가 오른쪽 그림과 같이 원점과 $(2, 0)$을 지나고 꼭짓점의 $y$좌표가 $-1$인 포물선이다. $f(x)$의 극댓값이 $5$일 때, $f(x)$의 극솟값을 구하여라.

    \item \textbf{Exercise 9-9} \\
    삼차함수 $f(x)$가 다음 조건을 만족시킨다.
    \begin{enumerate}
        \item $f(0)=0$
        \item 모든 실수 $x$에 대하여 $f'(-x) = f'(x)$
        \item $f(1)=2$
    \end{enumerate}
    이때 $\int f(x) dx$를 구하여라. (단, 적분상수는 $C$)
\end{enumerate}

\end{document}