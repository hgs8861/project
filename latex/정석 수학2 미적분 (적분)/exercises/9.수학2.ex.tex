\documentclass[a4paper,11pt]{article}
\usepackage{kotex}
\usepackage{amsmath, amssymb}
\usepackage{geometry}
\usepackage{enumitem}

% --- 여백 및 간격 설정 ---
\geometry{left=2.5cm, right=2.5cm, top=2.5cm, bottom=2.5cm}
\linespread{1.3} % 줄 간격 1.3배
\setlength{\parskip}{1em} % 문단 간격

\title{\textbf{수학 II - Chapter 9 연습문제}}
\date{}

\begin{document}
\maketitle

\section*{기본 문제}

\begin{enumerate}
    \item \textbf{Exercise 9-1} \\
    함수 $f(x)$가 다음 등식을 만족시킬 때, $f(x)$의 극값을 구하여라.
    \[ \int \{1-f(x)\} dx = \dfrac{3}{2}x^2 - \dfrac{1}{4}x^4 + C \]

    \item \textbf{Exercise 9-2} \\
    삼차함수 $f(x)$는 $x=1$에서 극댓값 4를 가진다. $f'(x)=3x^2-6x+a$일 때, 상수 $a$의 값과 $f(x)$의 극솟값을 구하여라.

    \item \textbf{Exercise 9-3} \\
    곡선 $y=f(x)$ 위의 임의의 점 $(x, y)$에서의 접선의 기울기가 $3x^2-4x+1$이고, 이 곡선이 원점을 지날 때, $f(x)$를 구하여라.

    \item \textbf{Exercise 9-4} \\
    함수 $y=f(x)$에 대하여 $x$의 증분 $\Delta x$와 $y$의 증분 $\Delta y$ 사이에
    \[ \Delta y = x \Delta x + k(\Delta x)^2 \quad (k\text{는 상수}) \]
    인 관계가 성립할 때, $f'(x)$와 $f(x)$를 구하여라.

    \item \textbf{Exercise 9-5} \\
    "$f(x)$를 적분하여라."라는 문제를 잘못 보아 미분하였더니 $3x^2+x-1$이 되었다. $f(0)=1$일 때, 옳은 답을 구하여라.
\end{enumerate}

\section*{실력 문제}

\begin{enumerate}
    \item \textbf{Exercise 9-14} \\
    다항함수 $f(x)$의 부정적분을 $F(x)$라고 할 때,
    \[ F(x) = x f(x) - 2x^3 + x^2 \]
    이 성립한다. $f(0)=1$일 때, $f(x)$를 구하여라.

    \item \textbf{Exercise 9-17} \\
    함수 $f(x)$의 극댓값과 극솟값의 차가 36이다. $f'(x) = x^2 - (a+1)x + a$일 때, 실수 $a$의 값을 구하여라. (단, $a>1$)

    \item \textbf{Exercise 9-20} \\
    모든 실수 $x, y$에 대하여 $f(x+y) = f(x) + f(y) + 1$을 만족시키고 $f'(0)=2$인 미분가능한 함수 $f(x)$에 대하여 $f(0)$과 $f(x)$를 구하여라.

    \item \textbf{Exercise 9-22} \\
    미분가능한 두 함수 $f(x), g(x)$가
    \[ f'(g(x)) = x, \quad g'(x) = 3x^2 \]
    을 만족시킬 때, $f(g(x))$를 구하여라. (단, $f(g(0))=1$)
\end{enumerate}

\end{document}