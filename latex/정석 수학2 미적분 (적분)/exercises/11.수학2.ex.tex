\documentclass[a4paper,11pt]{article}
\usepackage{kotex}
\usepackage{amsmath, amssymb}
\usepackage{geometry}
\usepackage{enumitem}

% --- 여백 및 간격 설정 ---
\geometry{left=2.5cm, right=2.5cm, top=2.5cm, bottom=2.5cm}
\linespread{1.3}
\setlength{\parskip}{1em}

\title{\textbf{수학 II - Chapter 11 연습문제}}
\date{}

\begin{document}
\maketitle

\section*{기본 문제}

\begin{enumerate}
    \item \textbf{Exercise 11-1} \\
    $f(x) = \int_{-1}^{x} |(t+2)(t-1)| dt$ 일 때, 다음 극한값을 구하여라.
    \begin{enumerate}[label=(\arabic*)]
        \item $\displaystyle \lim_{h \to 0} \frac{f(3+2h) - f(3)}{h}$
        \item $\displaystyle \lim_{h \to 0} \frac{f(x+h) - f(x-h)}{h}$
    \end{enumerate}

    \item \textbf{Exercise 11-3} \\
    함수 $f(x) = \int_{0}^{x} (t-1)(t-2) dt$ 가 증가하는 $x$의 범위를 구하여라.

    \item \textbf{Exercise 11-5} \\
    $\displaystyle \int_{1}^{x} f(t) dt = x^3 + ax^2 - 2$ 를 만족시키는 다항함수 $f(x)$와 상수 $a$를 구하여라.

    \item \textbf{Exercise 11-7} \\
    함수 $f(x) = x^2 + ax + b$ 에 대하여 $\displaystyle \frac{d}{dx} \int_{0}^{x} f(t) dt = \int_{1}^{x} f'(t) dt$ 가 성립할 때, $a$의 값을 구하여라.
\end{enumerate}

\section*{실력 문제}

\begin{enumerate}
    \item \textbf{Exercise 11-9} \\
    $x \ge -1$ 일 때, 함수 $f(x) = \int_{-1}^{x} |t|(1-t) dt$ 의 최댓값을 구하여라.

    \item \textbf{Exercise 11-11} \\
    모든 실수 $x$에 대하여 $f(x) = x^2 + \int_{0}^{x} (t-x)g(t) dt$ 를 만족시키는 다항함수 $f(x), g(x)$가 있다. $f(x)$가 $(x-2)^2$으로 나누어떨어질 때, $g(2)$의 값을 구하여라.

    \item \textbf{Exercise 11-13} \\
    연속함수 $f(x)$가 모든 실수 $x$에 대하여 $f(x) = x^2 - 2x + \int_{0}^{2} |x-t|f(t) dt$ 를 만족시킬 때, $f(1)$의 값을 구하여라.

    \item \textbf{Exercise 11-15} \\
    삼차함수 $f(x)$에 대하여 $g(x) = \int_{0}^{x} f(t) dt$ 라 하자. $y=f(x)$의 그래프가 $x$축과 서로 다른 세 점 $(0,0), (\alpha, 0), (\beta, 0)$에서 만날 때, $g(x)$가 극값을 갖지 않기 위한 조건을 구하여라.
\end{enumerate}

\end{document}