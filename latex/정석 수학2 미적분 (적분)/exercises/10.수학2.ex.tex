\documentclass[10pt, a4paper]{article}

% --- 1. 필수 패키지 설정 ---
\usepackage[a4paper, top=18mm, bottom=15mm, left=12mm, right=12mm, headheight=25pt, headsep=8mm]{geometry}
\usepackage{amsmath, amssymb, amsthm}
\usepackage{fancyhdr}
\usepackage{enumitem}
\usepackage{array}
\usepackage{longtable}
\usepackage{kotex}

% --- 2. 헤더 설정 ---
\pagestyle{fancy}
\fancyhf{}
\lhead{\large \textbf{수학2: 제10장 정적분의 계산 연습문제}}
\rhead{\small 학번: \underline{\hspace{3.5cm}} 이름: \underline{\hspace{3cm}}}
\cfoot{\small \thepage}
\renewcommand{\headrulewidth}{0.6pt}

% --- 3. 문제 박스 명령어 정의 ---
\newcounter{probcount}
\newcommand{\exercise}[1]{%
    \stepcounter{probcount}%
    \begin{minipage}[t][110mm][t]{0.44\textwidth}
        \small 
        \vspace{1mm}
        \noindent \textbf{문제 \arabic{probcount}.} #1
    \end{minipage}% 
    }

\begin{document}

\noindent
\begin{longtable}{p{0.47\textwidth} | p{0.47\textwidth}}
    
    % [기본 문제 섹션 - 6문항 선별]
    \exercise{$F(t)=\int_{0}^{t}(1+x+x^{2}+\cdot\cdot\cdot+x^{n})dx$ 일 때, $\int_{0}^{1}F(t)dt=\frac{11}{12}$ 을 만족시키는 자연수 $n$의 값을 구하여라.} 
    & 
    \exercise{함수 $f(x)=x^{3}-(a+1)x^{2}+ax$ 가 $f(x)=\int_{0}^{1}\{f(x)-f(t)\}dt$ 를 만족시킬 때, 상수 $a$의 값을 구하여라.} \\
    \hline

    \exercise{$\text{Max}(a, b)$는 $a, b$ 중에서 작지 않은 것을 나타낼 때, $f^{+}(x)=\text{Max}(x, 0)$, $f^{-}(x)=\text{Max}(-x, 0)$ 으로 정의하자. 이때, $\int_{-1}^{2}f^{+}(x)dx+\int_{-1}^{2}f^{-}(x)dx$ 의 값을 구하여라.} 
    & 
    \exercise{함수 $f(x)$가 모든 실수 $x$에 대하여 $f(x+2)=f(x)$ 를 만족시키고, $-1 \le x \le 1$ 에서
    \[ f(x)=\begin{cases} x+1 & (-1 \le x \le 0) \\ -x+1 & (0 \le x \le 1) \end{cases} \]
    일 때, $\int_{-5}^{5}f(x+1)dx$ 의 값을 구하여라.} \\
    
    \exercise{연속함수 $f(x)$가 다음 세 조건을 만족시킨다.
    \begin{enumerate}[label=(\text{가})]
        \item 모든 실수 $x$에 대하여 $f(2+x)=f(2-x)$
        \item $\int_{-2}^{2}f(x)dx=2k+4$
        \item $\int_{0}^{6}f(x)dx=k^{2}$
    \end{enumerate}
    $\int_{0}^{4}f(x)dx$ 의 값이 최소가 되는 상수 $k$의 값을 구하여라.} 
    & 
    \exercise{함수 $f(x)=x^{3}-6x^{2}+8$ 에 대하여 $0 \le x \le r$ 에서 $|f(x)|$ 의 최댓값을 $M(r)$ 라고 할 때, $\int_{0}^{5}M(r)dr$ 의 값을 구하여라.} \\
    \hline

    
    \exercise{$f(x)=\begin{cases} 2x-1 & (x \le 1) \\ x^{2} & (x \ge 1) \end{cases}$ 으로 정의된 함수 $f(x)$ 에 대하여 $\int_{1}^{3}xf(x-1)dx$ 의 값을 구하여라.} 
    & 
    \exercise{$f(x)=\int_{0}^{1}|t^{2}-xt|dt$ 로 정의된 함수 $f(x)$ 의 최솟값을 구하여라.} \\
    
    \exercise{다음과 같이 정의된 다항함수 $f_{n}(x)$ 를 구하여라.
    \[ f_{1}(x)=2x, \quad f_{n+1}(x)=x^{3}+\frac{1}{2}\int_{0}^{1}f_{n}(x)dx \quad (\text{단, } n=1, 2, 3, \dots) \]} 
    & 
    \exercise{모든 일차함수 $g(x)$ 에 대하여 $\int_{0}^{1}g(x)f(x)dx=0$ 을 만족시키고, $f(0)=1$인 이차함수 $f(x)$ 를 구하여라.} \\
    \hline

      
    \exercise{함수 $y=f(x)$ 의 그래프의 개형이 아래로 볼록할 때, 실수 $a, b$에 대하여 다음 두 식의 대소를 비교하여라. (단, $f(x)>0$ 이다.)
    \[ \frac{1}{b-a}\int_{a}^{b}f(x)dx, \quad \frac{f(a)+f(b)}{2} \]} 
    & 
    \exercise{함수 $f(x)$ 가 구간 $[a, b]$ 에서 연속일 때, $\int_{a}^{b}f(x)dx=(b-a)f(c)$ 를 만족시키는 $c$가 구간 $[a, b]$ 에 적어도 하나 존재함을 증명하여라.} \\
    %
\end{longtable}

\end{document}