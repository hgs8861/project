\documentclass[a4paper,11pt]{article}
\usepackage{kotex}
\usepackage{amsmath, amssymb}
\usepackage{geometry}
\usepackage{enumitem}

% --- 여백 및 간격 설정 ---
\geometry{left=2.5cm, right=2.5cm, top=2.5cm, bottom=2.5cm}
\linespread{1.3}
\setlength{\parskip}{1em}

\title{\textbf{수학 II - Chapter 13 연습문제}}
\date{}

\begin{document}
\maketitle

\section*{기본 문제}

\begin{enumerate}
    \item \textbf{Exercise 13-1} \\
    원점을 출발하여 수직선 위를 움직이는 점 P의 시각 $t$에서의 속도 $v(t)$가 $v(t)=4-2t$일 때, $t=0$에서 $t=4$까지 점 P가 움직인 거리를 구하여라.

    \item \textbf{Exercise 13-3} \\
    지면에서 $30\text{m/s}$의 속도로 똑바로 위로 던진 물체의 $t$초 후의 속도가 $v(t)=30-10t$이다. 이 물체가 지면에 떨어질 때까지 움직인 총 거리를 구하여라.

    \item \textbf{Exercise 13-5} \\
    수직선 위를 움직이는 두 점 P, Q의 시각 $t$에서의 속도가 각각 $v_P=4t^3-12t, v_Q=2t$이다. $t=0$일 때 두 점 P, Q의 위치가 각각 $10, 2$라면, 두 점이 만나는 시각 $t$를 구하여라.
\end{enumerate}

\section*{실력 문제}

\begin{enumerate}
    \item \textbf{Exercise 13-7} \\
    시각 $t=0$일 때 동시에 원점을 출발하여 수직선 위를 움직이는 두 점 P, Q의 시각 $t$에서의 속도가 각각 $v_P = 1-2t, v_Q = 3t^2-1$이다. 출발 후 두 점 P, Q가 다시 만날 때까지 두 점 사이의 거리의 최댓값을 구하여라.

    \item \textbf{Exercise 13-9} \\
    원점을 출발하여 수직선 위를 움직이는 점 P의 시각 $t$에서의 속도 $v(t)$의 그래프가 아래와 같을 때, $t=0$에서 $t=6$까지 점 P가 실제로 움직인 거리를 구하여라. (그래프는 $t=2, 4$에서 $t$축과 만나는 직선들로 구성됨)
    

    \item \textbf{Exercise 13-11} \\
    가속도가 $a(t)=6t-12$인 물체가 $t=0$일 때 속도 $v_0=9$로 원점을 출발하였다. 이 물체가 다시 원점을 통과하는 시각 $t$를 구하여라.
\end{enumerate}

\end{document}