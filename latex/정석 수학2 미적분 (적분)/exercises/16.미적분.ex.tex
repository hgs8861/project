\documentclass[10pt, a4paper]{article}

% --- 1. 필수 패키지 설정 ---
\usepackage[a4paper, top=18mm, bottom=15mm, left=12mm, right=12mm, headheight=25pt, headsep=8mm]{geometry}
\usepackage{amsmath, amssymb, amsthm}
\usepackage{fancyhdr}
\usepackage{enumitem}
\usepackage{array}
\usepackage{longtable}
\usepackage{kotex}

% --- 2. 헤더 설정 ---
\pagestyle{fancy}
\fancyhf{}
\lhead{\large \textbf{미적분 : 제16장 여러 가지 정적분에 관한 문제 연습문제}}
\rhead{\small 학번: \underline{\hspace{3.5cm}} 이름: \underline{\hspace{3cm}}}
\cfoot{\small \thepage}
\renewcommand{\headrulewidth}{0.6pt}

% --- 3. 문제 박스 명령어 정의 ---
\newcounter{probcount}
\newcommand{\exercise}[1]{%
    \stepcounter{probcount}%
    \begin{minipage}[t][120mm][t]{0.44\textwidth}
        \small 
        \vspace{1mm}
        \noindent \textbf{문제 \arabic{probcount}.} #1
    \end{minipage}%
}

\begin{document}

\noindent
\begin{longtable}{p{0.47\textwidth} | p{0.47\textwidth}}
    
    % [기본 문제 섹션]
    \exercise{정적분을 이용하여 다음 급수의 합을 구하여라.
    \begin{enumerate}[label=(\arabic*), leftmargin=8mm, nosep]
        \item $\displaystyle \lim_{n \to \infty} \sum_{k=1}^n \dfrac{1}{n+k}$
        \item $\displaystyle \lim_{n \to \infty} \sum_{k=n+1}^{2n} \dfrac{1}{2n+k}$
    \end{enumerate}} 
    & 
    \exercise{$f(x) = \displaystyle \int_{-x}^{x} \dfrac{\cos t}{1+e^t} dt$ 라고 할 때, $f^{\prime}(x)$ 와 $f(x)$ 를 구하여라.} \\
    \hline 
    
    \exercise{모든 실수 $x$ 에 대하여 다음 등식을 만족시키고 $f(0)=3$ 인 연속함수 $f(x)$ 를 구하여라.
    \[ (1-x) \int_0^x f(t) dt = x \int_x^1 f(t) dt \]} 
    & 
    \exercise{연속함수 $f(x)$ 가 모든 실수 $t$ 에 대하여 $\displaystyle \int_0^2 xf(tx) dx = 4t^2$ 을 만족시킬 때, $f(2)$ 의 값을 구하여라.} \\


    \exercise{다음 세 조건을 만족시키는 모든 미분가능한 함수 $f(x)$ 에 대하여 $\displaystyle \int_0^2 f(x) dx$ 의 최솟원을 구하여라.
    \begin{enumerate}[label=(\roman*), nosep]
        \item $f(0)=1, f^{\prime}(0)=1$
        \item $0 < a < b < 2$ 이면 $f^{\prime}(a) \le f^{\prime}(b)$ 이다.
        \item 구간 $(0, 1)$ 에서 $f^{\prime\prime}(x) = e^x$ 이다.
    \end{enumerate}} 
    & 
    % [실력 문제 섹션]
    \exercise{반지름의 길이가 $r$ 인 구의 부피를 구분구적법으로 구하여라.} \\
    \hline 
    
    \exercise{정적분을 이용하여 다음 극한값을 구하여라.
    \begin{enumerate}[label=(\arabic*), leftmargin=8mm, nosep]
        \item $\displaystyle \lim_{n \to \infty} \dfrac{1}{n} (\sum_{k=n+1}^{2n} \ln k - n \ln n)$
        \item $\displaystyle \lim_{n \to \infty} \{ \dfrac{(2n)!}{n!n^n} \}^{1/n}$
    \end{enumerate}} 
    & 
    \exercise{반지름의 길이가 $r$ 인 원 위에 한 점 $P$ 와 이 점에서 원에 접하는 직선 $l$ 이 있다. 점 $P$ 를 지나는 현이 직선 $l$ 과 이루는 각의 크기가 $\frac{\pi k}{2n} (k=1, 2, \dots, 2n-1)$ 일 때, 현의 길이를 $r_k$ 라고 하자. $\displaystyle \lim_{n \to \infty} \dfrac{1}{2n-1} \sum_{k=1}^{2n-1} r_k^2$ 의 값을 구하여라.} \\
   

    \exercise{$0 \le x \le \frac{\pi}{2}$ 에서 연속인 함수 $f(x)$ 가 다음 두 조건을 만족시킬 때, $f(\frac{\pi}{4})$ 의 값을 구하여라.
    \begin{enumerate}[label=(\roman*), nosep]
        \item $\displaystyle \int_0^{\frac{\pi}{2}} f(t) dt = 1$
        \item $\cos x \displaystyle \int_0^x f(t) dt = \sin x \int_x^{\frac{\pi}{2}} f(t) dt$
    \end{enumerate}} 
    & 
    \exercise{$a>0, b>0$ 일 때, 다음 부등식을 증명하여라.
    \[ \int_0^a \ln(x+1) dx + \int_0^b (e^x-1) dx \ge ab \]} \\
    \hline

\end{longtable}

\end{document}