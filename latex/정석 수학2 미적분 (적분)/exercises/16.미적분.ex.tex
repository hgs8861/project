\documentclass[10pt, a4paper]{article}

% --- 1. 필수 패키지 설정 ---
\usepackage[a4paper, top=18mm, bottom=15mm, left=12mm, right=12mm, headheight=25pt, headsep=8mm]{geometry}
\usepackage{amsmath, amssymb, amsthm}
\usepackage{fancyhdr}
\usepackage{enumitem}
\usepackage{array}
\usepackage{longtable}
\usepackage{kotex}

% --- 2. 헤더 설정 ---
\pagestyle{fancy}
\fancyhf{}
\lhead{\large \textbf{미적분 : 제16장 여러 가지 정적분에 관한 문제 연습문제}}
\rhead{\small 학번: \underline{\hspace{3.5cm}} 이름: \underline{\hspace{3cm}}}
\cfoot{\small \thepage}
\renewcommand{\headrulewidth}{0.6pt}

% --- 3. 문제 박스 명령어 정의 ---
\newcounter{probcount}
\newcommand{\exercise}[1]{%
    \stepcounter{probcount}%
    \begin{minipage}[t][115mm][t]{0.44\textwidth}
        \small 
        \vspace{1mm}
        \noindent \textbf{문제 \arabic{probcount}.} #1
    \end{minipage}%
}

\begin{document}

\noindent
\begin{longtable}{p{0.47\textwidth} | p{0.47\textwidth}}
    
    % [기본 문제 섹션 - 7문항 선정]
    \exercise{정적분을 이용하여 다음 급수의 합을 구하여라.
    \begin{enumerate}[label=(\arabic*), leftmargin=8mm, nosep]
        \item $\displaystyle \lim_{n \to \infty} \sum_{k=1}^n \dfrac{1}{n+k}$
        \item $\displaystyle \lim_{n \to \infty} \sum_{k=1}^{2n} \dfrac{1}{2n+k}$
        \item $\displaystyle \lim_{n \to \infty} \sum_{k=n+1}^{2n} \dfrac{1}{2n+k}$
    \end{enumerate}} 
    & 
    \exercise{$f(x) = \displaystyle \int_{-x}^{x} \dfrac{\cos t}{1+e^t} dt$ 라고 할 때, $f^{\prime}(x)$ 와 $f(x)$ 를 구하여라.} \\
    \hline 
    
    \exercise{모든 실수 $x$ 에 대하여 다음 등식을 만족시키고 $f(0)=3$ 인 연속함수 $f(x)$ 를 구하여라.
    \[ (1-x) \int_0^x f(t) dt = x \int_x^1 f(t) dt \]} 
    & 
    \exercise{모든 실수 $x$ 에 대하여 다음 등식을 만족시키는 연속함수 $f(x)$ 와 상수 $a$ 의 값을 구하여라.
    \[ \int_0^x f(t) dt = e^x - ae^{2x} \int_0^1 f(t) e^{-t} dt \]} \\
    \hline

    \exercise{다음 방정식을 풀어라.
    \[ \sin \left\{ \dfrac{1}{2} \pi \log_x \left( \dfrac{d}{dx} \int_1^x t^2 dt \right) \right\} = x^2 - 2x \]} 
    & 
    \exercise{함수 $f(x) = \displaystyle \int_0^x e^t (t-1)(t-2) dt$ 가 극댓값을 갖는 $x$ 와 극솟값을 갖는 $x$ 를 구하여라.} \\
    \hline

    \exercise{함수 $f(x)$ 가 모든 실수 $x$ 에 대하여 $f(x+2)=f(x)$ 이고, $\displaystyle \int_{-1}^1 f(x)dx=4$ 일 때, $\displaystyle \int_0^{10} f(x)dx$ 의 값을 구하여라.} 
    & 
    % [실력 문제 섹션 - 6문항 선정]
    \exercise{정적분을 이용하여 다음 극한값을 구하여라.
    \begin{enumerate}[label=(\arabic*), leftmargin=8mm, nosep]
        \item $\displaystyle \lim_{n \to \infty} \dfrac{1}{n} \left( \sum_{k=n+1}^{2n} \ln k - n \ln n \right)$
        \item $\displaystyle \lim_{n \to \infty} \left\{ \dfrac{(2n)!}{n! n^n} \right\}^{1/n}$
    \end{enumerate}} \\
    \hline 
    
    \exercise{연속함수 $f(x)$ 에 대하여 $\displaystyle \int_0^x f(t) dt = \sin^2 x$ 일 때, $\displaystyle \int_0^{\pi/2} t^2 f(t) dt$ 의 값을 구하여라.} 
    & 
    \exercise{연속함수 $f(x)$ 에 대하여 $\displaystyle \int_0^{\pi} f(\sin x) dx = K$ 일 때, 정적분 
    \[ \int_0^{\pi} x f(\sin x) dx \]
    의 값을 $K$ 로 나타내어라.} \\
    \hline 

    \exercise{정적분 
    \[ \int_0^{\pi/2} \dfrac{\sin^n x}{\sin^n x + \cos^n x} dx \]
    의 값을 구하여라. (단, $n$ 은 자연수)} 
    & 
    % [문제 12 & 13]
    \exercise{$I_n = \displaystyle \int_0^1 \dfrac{x^n}{1+x} dx$ ($n=1, 2, 3, \dots$) 일 때, 다음 물음에 답하여라.
    \begin{enumerate}[label=(\arabic*), leftmargin=8mm, nosep]
        \item $I_n + I_{n-1} = \dfrac{1}{n}$ 임을 증명하여라.
        \item $\displaystyle \lim_{n \to \infty} I_n = 0$ 임을 보여라.
    \end{enumerate}}  \\
    \hline 
    
    \exercise{함수 $f(x) = \displaystyle \int_1^x \dfrac{\ln t}{1+t} dt$ 일 때, 모든 양수 $x$ 에 대하여 
    \[ f(x) + f\left(\dfrac{1}{x}\right) = \dfrac{1}{2}(\ln x)^2 \]
    이 성립함을 증명하여라.} \\

\end{longtable}

\end{document}