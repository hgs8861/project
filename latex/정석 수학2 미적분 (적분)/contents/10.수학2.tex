\documentclass[a4paper,11pt]{article}
\usepackage{kotex}
\usepackage{amsmath, amssymb, amsthm}
\usepackage{geometry}
\usepackage{tcolorbox}
\usepackage{enumitem}
\tcbuselibrary{skins, breakable}

% --- 여백 및 간격 설정 ---
\geometry{left=2.5cm, right=2.5cm, top=2.5cm, bottom=2.5cm}
\linespread{1.3} % 줄 간격 1.3배
\setlength{\parskip}{1em} % 문단 간격
\setlist[enumerate]{itemsep=5pt, topsep=0pt}

% --- 사용자 정의 박스 스타일 ---
% 1. 기본정석 (개념): 파랑 계열
\newtcolorbox{conceptbox}[1][]{
  colback=blue!5, colframe=blue!60, coltitle=white, fonttitle=\bfseries,
  title={#1}, arc=1mm, boxrule=0.5mm, breakable, parskip=1em
}
% 2. Advice (조언): 주황 계열
\newtcolorbox{advicebox}{
  colback=orange!5, colframe=orange!60, title={Advice}, fonttitle=\bfseries,
  coltitle=white, arc=1mm, boxrule=0.5mm, breakable, parskip=1em
}
% 3. 정석연구 (풀이/심화): 회색 계열
\newtcolorbox{studybox}{
  colback=gray!10, colframe=gray!60, title={정석연구}, fonttitle=\bfseries,
  coltitle=white, arc=1mm, boxrule=0.5mm, breakable, parskip=1em
}

\title{\textbf{수학 II - 제10장 정적분}}
\author{학습자료}
\date{}

\begin{document}
\maketitle

\section{정적분의 정의와 성질}

\subsection{정적분의 정의}

\begin{conceptbox}[기본정석: 정적분의 정의]
함수 $f(x)$가 두 실수 $a, b$를 포함하는 구간에서 연속일 때, $f(x)$의 부정적분 중 하나를 $F(x)$라고 하면
\[ \int_{a}^{b} f(x) dx = \left[ F(x) \right]_{a}^{b} = F(b) - F(a) \]
이 값을 $f(x)$의 $a$에서 $b$까지의 \textbf{정적분}이라고 한다.
이때 $a$를 \textbf{아래끝}, $b$를 \textbf{위끝}이라 한다.
\end{conceptbox}

\begin{advicebox}
\textbf{Advice 1$^\circ$: 정적분과 부정적분}
부정적분 $\int f(x)dx$는 $x$의 **함수**이고, 정적분 $\int_{a}^{b} f(x)dx$는 **실수값(상수)**이다.
또한, 정적분의 값은 적분변수에 무관하게 결정되므로 다음이 성립한다.
\[ \int_{a}^{b} f(x)dx = \int_{a}^{b} f(t)dt = \int_{a}^{b} f(u)du \]
\end{advicebox}

\subsection{정적분의 기본 성질}

\begin{conceptbox}[기본정석: 정적분의 성질]
\begin{enumerate}
    \item $\int_{a}^{a} f(x) dx = 0$
    \item $\int_{a}^{b} f(x) dx = - \int_{b}^{a} f(x) dx$
    \item $\int_{a}^{b} k f(x) dx = k \int_{a}^{b} f(x) dx$ (단, $k$는 상수)
    \item $\int_{a}^{b} \{f(x) \pm g(x)\} dx = \int_{a}^{b} f(x) dx \pm \int_{a}^{b} g(x) dx$
    \item $\int_{a}^{c} f(x) dx + \int_{c}^{b} f(x) dx = \int_{a}^{b} f(x) dx$ (구간의 연결)
\end{enumerate}
\end{conceptbox}

\newpage

\section{필수 예제 및 유제}

\subsection*{[필수 예제 10-1] 정적분의 계산}
다음 정적분의 값을 구하여라.
\begin{enumerate}
    \item $\int_{1}^{2} (x+1)(x^2-x+1) dx$
    \item $\int_{0}^{1} \dfrac{x^2}{x+1} dx - \int_{0}^{1} \dfrac{1}{x+1} dx$
\end{enumerate}

\begin{studybox}
(1) 피적분함수가 곱의 꼴로 되어 있으면 전개하여 적분한다.
(2) 적분 구간이 같으므로 $\int f(x)dx - \int g(x)dx = \int \{f(x)-g(x)\}dx$를 이용하여 식을 간단히 한다.
\end{studybox}

\textbf{모범답안}
\begin{enumerate}
    \item $\int_{1}^{2} (x^3+1) dx = \left[ \dfrac{1}{4}x^4 + x \right]_{1}^{2} = \left( \dfrac{16}{4} + 2 \right) - \left( \dfrac{1}{4} + 1 \right) = 6 - \dfrac{5}{4} = \dfrac{19}{4}$
    \item $\int_{0}^{1} \left( \dfrac{x^2}{x+1} - \dfrac{1}{x+1} \right) dx = \int_{0}^{1} \dfrac{x^2-1}{x+1} dx = \int_{0}^{1} (x-1) dx$
    \[ = \left[ \dfrac{1}{2}x^2 - x \right]_{0}^{1} = \left( \dfrac{1}{2} - 1 \right) - 0 = -\dfrac{1}{2} \]
\end{enumerate}

\textbf{유제 10-1} 다음 정적분의 값을 구하여라.
\begin{enumerate}
    \item $\int_{-1}^{1} (2x+1)^2 dx$
    \item $\int_{1}^{2} \dfrac{x^3}{x-1} dx - \int_{2}^{1} \dfrac{1}{1-x} dx$
\end{enumerate}

\vspace{0.5cm}
\hrule
\vspace{0.5cm}

\subsection*{[필수 예제 10-2] 절댓값 기호를 포함한 함수의 적분}
다음 정적분의 값을 구하여라.
\[ \int_{0}^{3} |x^2 - 2x| dx \]

\begin{studybox}
피적분함수에 절댓값 기호가 있으면 절댓값 안의 식의 부호가 바뀌는 점을 경계로 적분 구간을 나눈다.
$x^2-2x = x(x-2)$이므로 $0 \le x \le 2$에서는 음수, $2 \le x \le 3$에서는 양수이다.
\end{studybox}

\textbf{모범답안}
\[ \int_{0}^{3} |x^2 - 2x| dx = \int_{0}^{2} -(x^2-2x) dx + \int_{2}^{3} (x^2-2x) dx \]
\[ = \left[ -\dfrac{1}{3}x^3 + x^2 \right]_{0}^{2} + \left[ \dfrac{1}{3}x^3 - x^2 \right]_{2}^{3} \]
\[ = \left( -\dfrac{8}{3} + 4 \right) + \left\{ (9-9) - \left( \dfrac{8}{3} - 4 \right) \right\} = \dfrac{4}{3} + \dfrac{4}{3} = \dfrac{8}{3} \]

\textbf{유제 10-2} 정적분 $\int_{0}^{2} |x(x-1)| dx$ 의 값을 구하여라.

\vspace{0.5cm}
\hrule
\vspace{0.5cm}

\subsection*{[필수 예제 10-3] 우함수와 기함수의 정적분}
다음 정적분의 값을 구하여라.
\[ \int_{-2}^{2} (x^7 + 5x^5 - 4x^3 + 3x^2 + 2x - 1) dx \]

\begin{studybox}
적분 구간이 $[-a, a]$ 꼴이고 피적분함수가 다항함수일 때:
\begin{itemize}
    \item $f(-x)=-f(x)$ (기함수, 홀수 차수): $\int_{-a}^{a} f(x) dx = 0$
    \item $f(-x)=f(x)$ (우함수, 짝수 차수): $\int_{-a}^{a} f(x) dx = 2 \int_{0}^{a} f(x) dx$
\end{itemize}
\end{studybox}

\textbf{모범답안}
홀수 차수 항($x^7, 5x^5, -4x^3, 2x$)은 기함수이므로 정적분 값이 0이다.
따라서 짝수 차수 항과 상수항만 남겨서 계산한다.
\[ \int_{-2}^{2} (3x^2 - 1) dx = 2 \int_{0}^{2} (3x^2 - 1) dx \]
\[ = 2 \left[ x^3 - x \right]_{0}^{2} = 2 (8 - 2) = 12 \]

\textbf{유제 10-3} 함수 $f(x)=x^3-3x$와 $g(x)=x^2+1$에 대하여 $\int_{-1}^{1} \{f(x)+g(x)\} dx$ 의 값을 구하여라.

\vspace{0.5cm}
\hrule
\vspace{0.5cm}

\subsection*{[필수 예제 10-4] 정적분으로 정의된 함수}
다음 등식을 만족시키는 다항함수 $f(x)$와 상수 $a$의 값을 구하여라.
\[ \int_{a}^{x} f(t) dt = x^3 - 2x^2 + x \]

\begin{studybox}
정적분으로 표시된 함수의 미분: $\dfrac{d}{dx} \int_{a}^{x} f(t) dt = f(x)$
1. 양변을 $x$에 대하여 미분한다.
2. 아래끝과 위끝이 같아지는 $x$값(여기서는 $a$)을 대입한다.
\end{studybox}

\textbf{모범답안}
양변을 $x$에 대하여 미분하면
\[ f(x) = 3x^2 - 4x + 1 \]
양변에 $x=a$를 대입하면 좌변은 $\int_{a}^{a} f(t)dt = 0$이므로
\[ 0 = a^3 - 2a^2 + a \implies a(a-1)^2 = 0 \]
$\therefore a=0$ 또는 $a=1$

\textbf{유제 10-4} $\int_{1}^{x} f(t) dt = x^2 - ax + 3$ 을 만족시키는 함수 $f(x)$와 상수 $a$의 값을 구하여라.

\end{document}