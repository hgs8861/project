\documentclass[a4paper,11pt]{article}
\usepackage{kotex}
\usepackage{amsmath, amssymb, amsthm}
\usepackage{geometry}
\usepackage{tcolorbox}
\usepackage{enumitem}
\tcbuselibrary{skins, breakable}

% --- 여백 및 간격 설정 ---
\geometry{left=2.5cm, right=2.5cm, top=2.5cm, bottom=2.5cm}
\linespread{1.4}
\setlength{\parskip}{1em}
\setlist[enumerate]{itemsep=5pt, topsep=0pt}

% --- 사용자 정의 박스 스타일 ---
\newtcolorbox{conceptbox}[1][]{
  colback=blue!5, colframe=blue!60, coltitle=white, fonttitle=\bfseries,
  title={#1}, arc=1mm, boxrule=0.5mm, breakable, parskip=1em
}
\newtcolorbox{advicebox}{
  colback=orange!5, colframe=orange!60, title={Advice}, fonttitle=\bfseries,
  coltitle=white, arc=1mm, boxrule=0.5mm, breakable, parskip=1em
}
\newtcolorbox{studybox}{
  colback=gray!10, colframe=gray!60, title={정석연구}, fonttitle=\bfseries,
  coltitle=white, arc=1mm, boxrule=0.5mm, breakable, parskip=1em
}

\title{\textbf{미적분 - 제14장 치환적분과 부분적분}}
\author{수학의 정석}
\date{}

\begin{document}

\maketitle

\section*{§1. 치환적분법}

\begin{conceptbox}[기본 정석]
\begin{enumerate}
    \item \textbf{치환적분에 관한 공식} \\
    미분가능한 함수 $g(t)$ 에 대하여 $x=g(t)$ 로 놓으면
    \[ \int f(x)dx=\int f(g(t))g^{\prime}(t)dt \]
    \item \textbf{치환적분에 관한 기본 유형} \\
    (1) $\int f(x)dx=F(x)+C$ 이면 $\int f(ax+b)dx=\dfrac{1}{a}F(ax+b)+C(a\ne0)$ \\
    (2) $g(x)=t$ 라고 하면 $\int f(g(x))g^{\prime}(x)dx=\int f(t)dt$ \\
    (3) $\int\dfrac{f^{\prime}(x)}{f(x)}dx=\ln|f(x)|+C$
\end{enumerate}
\end{conceptbox}

\begin{advicebox}
\textbf{1° 치환적분에 관한 공식} \\
일반적으로 함수 $f(x)$ 의 부정적분을 $y=\int f(x)dx$ 라고 할 때, $x$ 가 $t$ 의 미분가능한 함수 $x=g(t)$ 로 나타내어지면 합성함수의 미분법에 의하여 $\dfrac{dy}{dt}=f(g(t))g^{\prime}(t)$ 이다. 따라서 $y=\int f(g(t))g^{\prime}(t)dt$ 가 성립한다. 이 공식을 이용하는 적분법을 치환적분법이라고 한다.
\end{advicebox}

\begin{advicebox}
\textbf{2° 치환적분에 관한 기본 유형} \\
(1) $\int f(ax+b)dx$ 꼴은 $ax+b=t$ 로 치환하여 공식을 적용하면 편리하다. \\
(2) $\int f(g(x))g^{\prime}(x)dx$ 꼴은 $g(x)=t$ 로 치환하면 $\int f(t)dt$ 가 된다. \\
(3) $\int\dfrac{f^{\prime}(x)}{f(x)}dx$ 꼴은 분모를 $t$ 로 치환하면 $\int \dfrac{1}{t}dt = \ln|t|+C = \ln|f(x)|+C$ 가 된다.
\end{advicebox}

\section*{§2. 부분적분법}

\begin{conceptbox}[기본 정석]
\textbf{부분적분에 관한 공식} \\
\[ \int f^{\prime}(x)g(x)dx=f(x)g(x)-\int f(x)g^{\prime}(x)dx \] \\
\[ \int u^{\prime}v~dx=uv-\int uv^{\prime}dx \quad (\text{단, } u=f(x), v=g(x)) \]
\end{conceptbox}

\begin{advicebox}
\textbf{2° 부분적분법을 적용하는 방법} \\
피적분함수가 두 함수의 곱으로 되어 있을 때 어느 것을 $u^{\prime}$ 이라 하고, 어느 것을 $v$ 라고 할 것인가를 잘 판단해야 한다. 대부분의 경우 다음 순서에 따른다: \\
\textbf{$\ln x$ (로그), $x^{n}$ (다항), $\sin x$ (삼각), $e^{x}$ (지수)} \\
위 순서에서 오른쪽으로 갈수록 $u^{\prime}$ 으로, 왼쪽으로 갈수록 $v$ 로 정한다.
\end{advicebox}

\newpage

\section*{필수 예제}

\subsection*{필수 예제 14-1}
다음 부정적분을 구하여라.
\begin{enumerate}[label=(\arabic*)]
    \item $\int x(1-x)^{10}dx$
    \item $\int\dfrac{1}{2x+1}dx$
    \item $\int\dfrac{1}{\sqrt[4]{2x+3}}dx$
    \item $\int\sin^{2}x~dx$
    \item $\int10^{3x+2}dx$
    \item $\int(e^{x}-e^{-x})^{2}dx$
\end{enumerate}

\begin{studybox}
함수 $f(ax+b)$ (괄호 안이 일차식) 꼴의 부정적분이다. 이와 같은 꼴의 부정적분은 $\int f(x)dx=F(x)+C$ 이면 $\Rightarrow\int f(ax+b)dx=\dfrac{1}{a}F(ax+b)+C(a\ne0)$ 를 기억해 두고서 활용하는 것이 편리하다.
\end{studybox}

\textbf{모범답안}
\begin{enumerate}[label=(\arabic*)]
    \item $\int x(1-x)^{10}dx=\int\{1-(1-x)\}(1-x)^{10}dx=\int\{(1-x)^{10}-(1-x)^{11}\}dx$ \\
    $=-\dfrac{1}{11}(1-x)^{11}+\dfrac{1}{12}(1-x)^{12}+C$
    \item $\dfrac{1}{2}\ln|2x+1|+C$
    \item $\int(2x+3)^{-\dfrac{1}{4}}dx=\dfrac{1}{2}\times\dfrac{4}{3}(2x+3)^{\dfrac{3}{4}}+C=\dfrac{2}{3}\sqrt[4]{(2x+3)^{3}}+C$
    \item $\int\dfrac{1-\cos 2x}{2}dx=\dfrac{1}{2}x-\dfrac{1}{4}\sin 2x+C$
    \item $\dfrac{1}{3}\times\dfrac{10^{3x+2}}{\ln 10}+C=\dfrac{10^{3x+2}}{3\ln 10}+C$
    \item $\int(e^{2x}-2+e^{-2x})dx=\dfrac{1}{2}e^{2x}-2x-\dfrac{1}{2}e^{-2x}+C$
\end{enumerate}

\vspace{0.5cm}
\hrule
\vspace{0.5cm}

\subsection*{필수 예제 14-2}
다음 부정적분을 구하여라.
\begin{enumerate}[label=(\arabic*)]
    \item $\int\dfrac{x^{2}+1}{x+1}dx$
    \item $\int\dfrac{2x+1}{(x-1)(x+2)^{2}}dx$
\end{enumerate}

\begin{studybox}
(1) $x^{2}+1$ 을 $x+1$ 로 나눈 몫은 $x-1$, 나머지는 2이므로 피적분함수를 다음과 같이 변형할 수 있다. $\dfrac{x^{2}+1}{x+1}=x-1+\dfrac{2}{x+1}$ \\
(2) 피적분함수를 부분분수로 변형한다. \\
\textbf{정석}: $\int\dfrac{1}{ax+b}dx=\dfrac{1}{a}\ln|ax+b|+C(a\ne0)$
\end{studybox}

\textbf{모범답안}
\begin{enumerate}[label=(\arabic*)]
    \item $\int(x-1+\dfrac{2}{x+1})dx=\dfrac{1}{2}x^{2}-x+2\ln|x+1|+C$
    \item $\dfrac{2x+1}{(x-1)(x+2)^{2}}=\dfrac{a}{x-1}+\dfrac{b}{x+2}+\dfrac{c}{(x+2)^{2}}$ 라 하면 $a=\dfrac{1}{3}, b=-\dfrac{1}{3}, c=1$ \\
    $\int(\dfrac{1/3}{x-1}-\dfrac{1/3}{x+2}+\dfrac{1}{(x+2)^{2}})dx=\dfrac{1}{3}\ln|x-1|-\dfrac{1}{3}\ln|x+2|-\dfrac{1}{x+2}+C$
\end{enumerate}

\vspace{0.5cm}
\hrule
\vspace{0.5cm}

\subsection*{필수 예제 14-3}
다음 부정적분을 구하여라.
\begin{enumerate}[label=(\arabic*)]
    \item $\int\dfrac{x}{\sqrt{x+2}}dx$
    \item $\int\dfrac{1}{\sqrt{x+1}-\sqrt{x-1}}dx$
\end{enumerate}

\begin{studybox}
(1)은 분자의 $x$ 를 $(x+2)-2$ 로 변형하고, (2)는 분모를 유리화한다. 그리고 이들을 $\int(ax+b)^{r}dx$ 의 꼴로 변형한 다음 $\dfrac{1}{a}\times\dfrac{1}{r+1}(ax+b)^{r+1}+C$ 를 이용한다.
\end{studybox}

\textbf{모범답안}
\begin{enumerate}[label=(\arabic*)]
    \item $\int(\sqrt{x+2}-\dfrac{2}{\sqrt{x+2}})dx=\int\{(x+2)^{\dfrac{1}{2}}-2(x+2)^{-\dfrac{1}{2}}\}dx=\dfrac{2}{3}(x-4)\sqrt{x+2}+C$
    \item $\int\dfrac{1}{2}(\sqrt{x+1}+\sqrt{x-1})dx=\dfrac{1}{3}\{(x+1)\sqrt{x+1}+(x-1)\sqrt{x-1}\}+C$
\end{enumerate}

\vspace{0.5cm}
\hrule
\vspace{0.5cm}

\subsection*{필수 예제 14-4}
다음 부정적분을 구하여라.
\begin{enumerate}[label=(\arabic*)]
    \item $\int(x^{2}+2x+2)^{5}(x+1)dx$
    \item $\int(x^{2}-1)\sqrt{x^{3}-3x}dx$
    \item $\int(1+\sin x)^{2}\cos x~dx$
    \item $\int\sin x\cos 2x~dx$
    \item $\int\dfrac{e^{x}}{\sqrt{e^{x}+1}}dx$
\end{enumerate}

\begin{studybox}
\textbf{정석}: $g(x)=t$ 라고 하면 $\int f(g(x))g^{\prime}(x)dx=\int f(t)dt$
\end{studybox}

\textbf{모범답안}
\begin{enumerate}[label=(\arabic*)]
    \item $x^{2}+2x+2=t \Rightarrow (x+1)dx=\dfrac{1}{2}dt$. $\int t^{5}\times\dfrac{1}{2}dt=\dfrac{1}{12}(x^{2}+2x+2)^{6}+C$
    \item $x^{3}-3x=t \Rightarrow (x^{2}-1)dx=\dfrac{1}{3}dt$. $\int\sqrt{t}\times\dfrac{1}{3}dt=\dfrac{2}{9}(x^{3}-3x)\sqrt{x^{3}-3x}+C$
    \item $1+\sin x=t \Rightarrow \cos x~dx=dt$. $\int t^{2}dt=\dfrac{1}{3}(1+\sin x)^{3}+C$
    \item $\cos x=t \Rightarrow \sin x~dx=-dt$. $\int(2t^{2}-1)(-dt)=-\dfrac{2}{3}\cos^{3}x+\cos x+C$
    \item $e^{x}+1=t \Rightarrow e^{x}dx=dt$. $\int\dfrac{1}{\sqrt{t}}dt=2\sqrt{e^{x}+1}+C$
\end{enumerate}

\vspace{0.5cm}
\hrule
\vspace{0.5cm}

\subsection*{필수 예제 14-5}
다음 부정적분을 구하여라.
\begin{enumerate}[label=(\arabic*)]
    \item $\int\dfrac{2x+1}{x^{2}+x+1}dx$
    \item $\int\dfrac{x+1}{x^{3}-1}dx$
    \item $\int\tan 2x~dx$
    \item $\int\dfrac{1}{x\ln x}dx$
    \item $\int\dfrac{e^{x}-e^{-x}}{e^{x}+e^{-x}}dx$
\end{enumerate}

\begin{studybox}
\textbf{정석}: $\int\dfrac{f^{\prime}(x)}{f(x)}dx=\ln|f(x)|+C$
\end{studybox}

\textbf{모범답안}
\begin{enumerate}[label=(\arabic*)]
    \item $\ln(x^{2}+x+1)+C$
    \item $\int(\dfrac{2/3}{x-1}-\dfrac{1/3(2x+1)}{x^{2}+x+1})dx=\dfrac{2}{3}\ln|x-1|-\dfrac{1}{3}\ln(x^{2}+x+1)+C$
    \item $-\dfrac{1}{2}\ln|\cos 2x|+C$
    \item $\ln|\ln x|+C$
    \item $\ln(e^{x}+e^{-x})+C$
\end{enumerate}

\vspace{0.5cm}
\hrule
\vspace{0.5cm}

\subsection*{필수 예제 14-6}
다음 부정적분을 구하여라.
\begin{enumerate}[label=(\arabic*)]
    \item $\int(2+3x)\sqrt{1+2x}dx$
    \item $\int\dfrac{x-1}{\sqrt{x+1}}dx$
    \item $\int\dfrac{x^{3}}{\sqrt{1-x^{2}}}dx$
\end{enumerate}

\begin{studybox}
\textbf{정석}: $\sqrt{f(x)}$ 를 포함한 부정적분은 $\sqrt{f(x)}=t$ 또는 $f(x)=t$ 로 치환하여라.
\end{studybox}

\textbf{모범답안}
\begin{enumerate}[label=(\arabic*)]
    \item $\sqrt{1+2x}=t \Rightarrow dx=tdt$. $\dfrac{1}{15}(9x+7)(2x+1)\sqrt{1+2x}+C$
    \item $\sqrt{x+1}=t \Rightarrow dx=2tdt$. $\dfrac{2}{3}(x-5)\sqrt{x+1}+C$
    \item $\sqrt{1-x^{2}}=t \Rightarrow xdx=-tdt$. $-\dfrac{1}{3}(x^{2}+2)\sqrt{1-x^{2}}+C$
\end{enumerate}

\vspace{0.5cm}
\hrule
\vspace{0.5cm}

\subsection*{필수 예제 14-7}
다음 물음에 답하여라.
\begin{enumerate}[label=(\arabic*)]
    \item $x=a\sin\theta(-\dfrac{\pi}{2}\le\theta\le\dfrac{\pi}{2})$ 로 치환하여 $\int\dfrac{1}{\sqrt{(a^{2}-x^{2})^{3}}}dx(a>0)$ 를 구하여라.
    \item $x=a\tan\theta(-\dfrac{\pi}{2}<\theta<\dfrac{\pi}{2})$ 로 치환하여 $\int\dfrac{1}{\sqrt{(a^{2}+x^{2})^{3}}}dx(a>0)$ 를 구하여라.
\end{enumerate}

\begin{studybox}
\textbf{정석}: $\sqrt{a^{2}-x^{2}}$ 가 있으면 $x=a\sin\theta$ 로, $\sqrt{a^{2}+x^{2}}$ 가 있으면 $x=a\tan\theta$ 로 치환한다.
\end{studybox}

\textbf{모범답안}
\begin{enumerate}[label=(\arabic*)]
    \item $\dfrac{1}{a^{2}}\tan\theta+C = \dfrac{x}{a^{2}\sqrt{a^{2}-x^{2}}}+C$
    \item $\dfrac{1}{a^{2}}\sin\theta+C = \dfrac{x}{a^{2}\sqrt{a^{2}+x^{2}}}+C$
\end{enumerate}

\vspace{0.5cm}
\hrule
\vspace{0.5cm}


\subsection*{필수 예제 14-8}
다음 부정적분을 구하여라.
\begin{enumerate}[label=(\arabic*)]
    \item $\int xe^{-x}dx$
    \item $\int x\cos^{2}x~dx$
    \item $\int \ln(x+2)dx$
\end{enumerate}

\textbf{모범답안}
\begin{enumerate}[label=(\arabic*)]
    \item $u^{\prime}=e^{-x}, v=x \Rightarrow -(x+1)e^{-x}+C$
    \item $\dfrac{1}{2}\int x(1+\cos 2x)dx = \dfrac{1}{8}(2x^{2}+2x\sin 2x+\cos 2x)+C$
    \item $u^{\prime}=1, v=\ln(x+2) \Rightarrow (x+2)\ln(x+2)-x+C$
\end{enumerate}

\vspace{0.5cm}
\hrule
\vspace{0.5cm}

\subsection*{필수 예제 14-9}
다음 부정적분을 구하여라.
\begin{enumerate}[label=(\arabic*)]
    \item $\int x^{2}e^{x}dx$
    \item $\int x^{2}\sin x~dx$
    \item $\int(\ln x)^{2}dx$
\end{enumerate}

\begin{studybox}
부분적분법을 한 번 적용해서 적분이 안 되면 다시 적용한다.
\end{studybox}

\textbf{모범답안}
\begin{enumerate}[label=(\arabic*)]
    \item $(x^{2}-2x+2)e^{x}+C$
    \item $(2-x^{2})\cos x+2x\sin x+C$
    \item $x(\ln x)^{2}-2x\ln x+2x+C$
\end{enumerate}

\vspace{0.5cm}
\hrule
\vspace{0.5cm}

\subsection*{필수 예제 14-10}
다음 부정적분을 구하여라.
\begin{enumerate}[label=(\arabic*)]
    \item $\int e^{x}\cos x~dx$
    \item $\int e^{x}\sin^{2}x~dx$
\end{enumerate}

\begin{studybox}
부분적분법을 거듭 적용하면 원래 함수가 포함된 적분 꼴이 반복하여 나타난다.
\end{studybox}

\textbf{모범답안}
\begin{enumerate}[label=(\arabic*)]
    \item $I=\int e^{x}\cos x~dx \Rightarrow I=\dfrac{1}{2}e^{x}(\sin x+\cos x)+C$
    \item $e^{x}\sin^{2}x-\dfrac{1}{5}e^{x}(\sin 2x-2\cos 2x)+C$
\end{enumerate}

\vspace{0.5cm}
\hrule
\vspace{0.5cm}

\subsection*{필수 예제 14-11}
자연수 $n$ 에 대하여 $I_{n}=\int(\ln x)^{n}dx$ 라고 할 때,
\begin{enumerate}[label=(\arabic*)]
    \item $I_{n}=x(\ln x)^{n}-nI_{n-1}(n\ge 2)$ 이 성립함을 증명하여라.
    \item $I_{1}, I_{2}, I_{3}$ 을 구하여라.
\end{enumerate}

\begin{studybox}
\textbf{정석}: 적분의 점화식은 부분적분을 이용한다.
\end{studybox}

\textbf{모범답안}
\begin{enumerate}[label=(\arabic*)]
    \item $u^{\prime}=1, v=(\ln x)^{n}$ 이라 하면 $I_{n}=x(\ln x)^{n}-n\int(\ln x)^{n-1}dx = x(\ln x)^{n}-nI_{n-1}$
    \item $I_{1}=x\ln x-x+C$, $I_{2}=x(\ln x)^{2}-2x\ln x+2x+C$, $I_{3}=x(\ln x)^{3}-3x(\ln x)^{2}+6x\ln x-6x+C$
\end{enumerate}

\end{document}