\documentclass[a4paper,11pt]{article}
\usepackage{kotex}
\usepackage{amsmath, amssymb, amsthm}
\usepackage{geometry}
\usepackage{tcolorbox}
\usepackage{enumitem}
\tcbuselibrary{skins, breakable}

% --- 여백 및 간격 설정 ---
\geometry{left=2.5cm, right=2.5cm, top=2.5cm, bottom=2.5cm}
\linespread{1.4}
\setlength{\parskip}{1em}
\setlist[enumerate]{itemsep=5pt, topsep=0pt}

% --- 사용자 정의 박스 스타일 ---
\newtcolorbox{conceptbox}[1][]{
  colback=blue!5, colframe=blue!60, coltitle=white, fonttitle=\bfseries,
  title={#1}, arc=1mm, boxrule=0.5mm, breakable, parskip=1em
}
\newtcolorbox{advicebox}{
  colback=orange!5, colframe=orange!60, title={Advice}, fonttitle=\bfseries,
  coltitle=white, arc=1mm, boxrule=0.5mm, breakable, parskip=1em
}
\newtcolorbox{studybox}{
  colback=gray!10, colframe=gray!60, title={정석연구}, fonttitle=\bfseries,
  coltitle=white, arc=1mm, boxrule=0.5mm, breakable, parskip=1em
}

\title{\textbf{수학 II - 제12장 넓이와 적분}}
\author{현경서T}
\date{}

\begin{document}

\maketitle

\section*{§1. 곡선과 좌표축 사이의 넓이}

\begin{conceptbox}[기본 정석]
\begin{enumerate}
    \item \textbf{곡선과 $x$축 사이의 넓이}\\
    (i) 구간 $[a, b]$에서 $f(x) \ge 0$인 경우: $S = \int_{a}^{b} f(x) dx$\\
    (ii) 구간 $[a, b]$에서 $f(x) \le 0$인 경우: $S = -\int_{a}^{b} f(x) dx$\\
    (iii) 구간 $[a, b]$에서 일반적인 경우: $S = \int_{a}^{b} |f(x)| dx$
    \item \textbf{곡선과 $y$축 사이의 넓이}\\
    함수 $g(y)$가 구간 $[\alpha, \beta]$에서 연속일 때\\
    (i) $g(y) \ge 0$인 경우: $S = \int_{\alpha}^{\beta} g(y) dy$\\
    (ii) $g(y) \le 0$인 경우: $S = -\int_{\alpha}^{\beta} g(y) dy$\\
    (iii) 일반적인 경우: $S = \int_{\alpha}^{\beta} |g(y)| dy$
\end{enumerate}
\end{conceptbox}

\begin{advicebox}
\textbf{곡선과 $x$축 사이의 넓이}\\
함수 $f(x)$가 구간 $[a, b]$에서 부호가 일정하지 않을 때의 넓이는 $f(x)$의 값이 양수인 구간과 음수인 구간으로 나누어서 구해야 한다. 절댓값 기호를 써서 하나의 식으로 나타내면 $S = \int_{a}^{b} |f(x)| dx$와 같다.
\end{advicebox}

\begin{advicebox}
\textbf{곡선과 $y$축 사이의 넓이}\\
곡선과 $x$축 사이의 넓이를 구할 때와 같은 방법으로 생각하면 된다. 곡선 $x=g(y)$가 $y$축의 오른쪽에 있을 때($g(y) \ge 0$)는 정적분 값이 곧 넓이이고, 왼쪽에 있을 때($g(y) \le 0$)는 정적분 값에 '$-$'(마이너스)를 붙여야 넓이가 된다.
\end{advicebox}

\begin{advicebox}
\textbf{$x=ay^2 (a \ne 0)$ 꼴의 곡선}\\
곡선 $x=y^2$은 곡선 $y=x^2$과 직선 $y=x$에 대하여 대칭인 포물선이다. $x=ay^2$ 꼴은 꼭짓점이 원점이고 축이 $x$축인 포물선이 된다. 이를 평행이동한 $x=a(y-n)^2+m$ 꼴도 같은 방법으로 이해할 수 있다.
\end{advicebox}

\section*{§2. 두 곡선 사이의 넓이}

\begin{conceptbox}[기본 정석]
\begin{enumerate}
    \item \textbf{두 곡선 $y=f(x), y=g(x)$ 사이의 넓이}\\
    구간 $[a, b]$에서 $f(x) \ge g(x)$일 때, 두 곡선으로 둘러싸인 도형의 넓이 $S$는
    \[ S = \int_{a}^{b} \{f(x) - g(x)\} dx \]
    \item \textbf{두 곡선 $x=f(y), x=g(y)$ 사이의 넓이}\\
    구간 $[\alpha, \beta]$에서 $f(y) \ge g(y)$일 때, 두 곡선으로 둘러싸인 도형의 넓이 $S$는
    \[ S = \int_{\alpha}^{\beta} \{f(y) - g(y)\} dy \]
\end{enumerate}
\end{conceptbox}

\begin{advicebox}
두 곡선 $y=f(x), y=g(x)$와 두 직선 $x=a, x=b$로 둘러싸인 도형의 넓이 $S$는 위에 있는 그래프의 식에서 아래에 있는 그래프의 식을 뺀 것을 적분하면 된다. 이는 두 곡선이 모두 $x$축 아래에 있거나 $x$축을 사이에 두고 있는 경우에도 성립한다. 일반적으로 $S = \int_{a}^{b} |f(x) - g(x)| dx$이다.
\end{advicebox}

\newpage

\section*{필수 예제}

\subsection*{필수 예제 12-1}
함수 $f(x)=x^{3}-x^{2}-x+a$ 가 극솟값 0을 가질 때, 다음 물음에 답하여라.
\begin{enumerate}[label=(\arabic*)]
    \item 상수 $a$의 값을 구하여라.
    \item 곡선 $y=f(x)$와 $x$축으로 둘러싸인 도형의 넓이를 구하여라.
\end{enumerate}

\begin{studybox}
넓이를 구하는 기본 방법은 (i) 넓이를 구하는 도형이 어떤 것인가 그린다. (ii) $dx$를 쓸 것인가, $dy$를 쓸 것인가를 판단한다. $dx$를 쓸 때는 $\int_{a}^{b} |y| dx$, $dy$를 쓸 때는 $\int_{\alpha}^{\beta} |x| dy$를 이용한다.
\end{studybox}

\textbf{모범답안}\\
(1) $f'(x)=3x^2-2x-1=(3x+1)(x-1)$. 증감표를 조사하면 $x=1$에서 극솟값 $f(1)=a-1$을 갖는다. 조건에서 $a-1=0$이므로 $a=1$이다.\\
(2) $f(x)=x^3-x^2-x+1=(x-1)^2(x+1)$이므로 $x$축과의 교점은 $x=-1, 1$(중근)이다. 구간 $[-1, 1]$에서 $f(x) \ge 0$이므로 구하는 넓이 $S$는
\[ S = \int_{-1}^{1} (x^3-x^2-x+1) dx = 2 \int_{0}^{1} (-x^2+1) dx = 2 \left[ -\dfrac{1}{3}x^3 + x \right]_0^1 = \dfrac{4}{3} \]

\vspace{0.5cm}
\hrule
\vspace{0.5cm}

\subsection*{필수 예제 12-2}
다음 곡선과 직선으로 둘러싸인 도형의 넓이를 구하여라.
\begin{enumerate}[label=(\arabic*)]
    \item $y=3-|x^2-1|$, $y=0$
    \item $y=x|1-x|$, $y=0$, $x=-1$, $x=2$
\end{enumerate}

\begin{studybox}
절댓값 기호를 포함한 식은 기호 안의 식의 부호에 따라 구간을 나누어 그래프를 그린 후 넓이를 구한다.
\end{studybox}

\textbf{모범답안}\\
(1) $|x|<1$이면 $y=x^2+2$, $|x| \ge 1$이면 $y=-x^2+4$이다. $y=0$과의 교점은 $x = \pm 2$이다. 그래프의 대칭성을 이용하면
\[ S = 2 \left\{ \int_0^1 (x^2+2) dx + \int_1^2 (-x^2+4) dx \right\} = 2 \left( \left[ \dfrac{1}{3}x^3+2x \right]_0^1 + \left[ -\dfrac{1}{3}x^3+4x \right]_1^2 \right) = 8 \]
(2) $x \ge 1$이면 $y=x^2-x$, $x < 1$이면 $y=-x^2+x$이다.
\[ S = \int_{-1}^0 (x^2-x) dx + \int_0^1 (-x^2+x) dx + \int_1^2 (x^2-x) dx = \dfrac{5}{6} + \dfrac{1}{6} + \dfrac{5}{6} = \dfrac{11}{6} \]

\vspace{0.5cm}
\hrule
\vspace{0.5cm}

\subsection*{필수 예제 12-3}
곡선 $y=x^3-(a+2)x^2+2ax$ (단, $0 < a < 2$)가 있다.
\begin{enumerate}[label=(\arabic*)]
    \item 이 곡선과 $x$축으로 둘러싸인 두 부분의 넓이가 같을 때, 상수 $a$의 값을 구하여라.
    \item 이 곡선과 $x$축으로 둘러싸인 도형의 넓이가 최소일 때, 상수 $a$의 값을 구하여라.
\end{enumerate}

\begin{studybox}
(1) 두 부분의 넓이가 같으면 전체 구간에 대한 정적분 값이 0임을 이용한다. $\int_{\alpha}^{\gamma} f(x) dx = 0$.\\
(2) 넓이를 $a$에 관한 함수 $S(a)$로 나타낸 후 미분을 통해 최솟값을 찾는다.
\end{studybox}

\textbf{모범답안}\\
교점은 $x=0, a, 2$이다.\\
(1) $\int_{0}^{2} \{x^3-(a+2)x^2+2ax\} dx = 0$ 이어야 하므로 $\left[ \dfrac{1}{4}x^4 - \dfrac{1}{3}(a+2)x^3 + ax^2 \right]_0^2 = 0 \Rightarrow a=1$.\\
(2) $S(a) = \int_0^a y dx - \int_a^2 y dx = -\dfrac{1}{6}a^4 + \dfrac{2}{3}a^3 - \dfrac{4}{3}a + \dfrac{4}{3}$이다. $S'(a) = -\dfrac{2}{3}(a-1)(a^2-2a-2)$. $0 < a < 2$에서 $a=1$일 때 $S(a)$는 최소가 된다.

\vspace{0.5cm}
\hrule
\vspace{0.5cm}

\subsection*{필수 예제 12-4}
$x \ge 1$에서 정의된 연속함수 $y=f(x)$와 그 역함수 $y=g(x)$의 그래프에 대하여 정적분 $\int_{1}^{5} f(x) dx + \int_{1}^{4} g(x) dx$의 값을 구하여라. (단, $f(1)=1, f(5)=4$)

\begin{studybox}
역함수의 그래프는 직선 $y=x$에 대하여 대칭임을 이용한다. 역함수의 정적분은 원래 함수의 그래프에서 $y$축 방향의 넓이로 해석할 수 있다.
\end{studybox}

\textbf{모범답안}\\
$\int_1^4 g(x) dx$는 $y=g(x)$와 $x$축 사이의 넓이이며, 이는 $y=f(x)$와 $y$축 사이의 넓이 $\int_1^4 g(y) dy$와 같다. 따라서 두 적분의 합은 직사각형의 넓이의 차로 나타낼 수 있다.
\[ \int_1^5 f(x) dx + \int_1^4 g(x) dx = (5 \times 4) - (1 \times 1) = 19 \]

\vspace{0.5cm}
\hrule
\vspace{0.5cm}

\subsection*{필수 예제 12-5}
다음 직선과 곡선 또는 곡선과 곡선으로 둘러싸인 도형의 넓이를 구하여라.
\begin{enumerate}[label=(\arabic*)]
    \item $y=x+1$, $y=x^2-1$
    \item $y=x(x-1)(x-2)$, $y=x(x-1)$
\end{enumerate}

\begin{studybox}
두 곡선의 교점을 구하여 적분 구간을 정하고, '위의 식 - 아래 식'을 적분한다.
\end{studybox}

\textbf{모범답안}\\
(1) 교점: $x+1=x^2-1 \Rightarrow x=-1, 2$. $[-1, 2]$에서 직선이 곡선보다 위에 있으므로
\[ S = \int_{-1}^2 \{(x+1)-(x^2-1)\} dx = \left[ -\dfrac{1}{3}x^3 + \dfrac{1}{2}x^2 + 2x \right]_{-1}^2 = \dfrac{9}{2} \]
(2) 교점: $x(x-1)(x-2)=x(x-1) \Rightarrow x=0, 1, 3$.
\[ S = \int_0^1 (x^3-4x^2+3x) dx + \int_1^3 (-x^3+4x^2-3x) dx = \dfrac{5}{12} + \dfrac{8}{3} = \dfrac{37}{12} \]

\vspace{0.5cm}
\hrule
\vspace{0.5cm}

\subsection*{필수 예제 12-6}
다음 직선과 곡선 또는 곡선과 곡선으로 둘러싸인 도형의 넓이 $S$를 구하여라.
\begin{enumerate}[label=(\arabic*)]
    \item $y=x-2$, $y^2+2y=x$
    \item $x=(y-2)^2+1$, $(x-1)^2+(y-1)^2=1$ (단, $x \ge 1$)
\end{enumerate}

\begin{studybox}
$x$에 관하여 적분하기 복잡할 때는 식을 $x=g(y)$ 꼴로 고쳐서 $y$에 관하여 적분한다.
\end{studybox}

\textbf{모범답안}\\
(1) 교점의 $y$좌표: $y^2+2y=y+2 \Rightarrow y=-2, 1$. 구간 $[-2, 1]$에서 직선 $x=y+2$가 포물선 $x=y^2+2y$보다 오른쪽에 있다.
\[ S = \int_{-2}^1 \{(y+2)-(y^2+2y)\} dy = \left[ -\dfrac{1}{3}y^3 - \dfrac{1}{2}y^2 + 2y \right]_{-2}^1 = \dfrac{9}{2} \]
(2) 교점: $(y-2)^4+(y-1)^2=1 \Rightarrow y=1, 2$. 평행이동을 이용하여 구하면
\[ S = \dfrac{\pi}{4} - \int_0^1 (y-1)^2 dy = \dfrac{\pi}{4} - \left[ \dfrac{1}{3}(y-1)^3 \right]_0^1 = \dfrac{\pi}{4} - \dfrac{1}{3} \]

\vspace{0.5cm}
\hrule
\vspace{0.5cm}

\subsection*{필수 예제 12-7}
다음 두 곡선 $y=x(a-x)$...(1), $y=x^2(a-x)$...(2)에 대하여 물음에 답하여라.
\begin{enumerate}[label=(\arabic*)]
    \item 곡선 (1)과 $x$축으로 둘러싸인 넓이를 곡선 (2)가 이등분할 때, 상수 $a$의 값을 구하여라. (단, $0 < a \le 1$)
    \item 곡선 (1)과 (2)로 둘러싸인 두 부분의 넓이가 같을 때, 상수 $a$의 값을 구하여라. (단, $a > 1$)
\end{enumerate}

\textbf{모범답안}\\
교점은 $x=0, 1, a$이다.\\
(1) $\int_0^a x(a-x) dx = 2 \int_0^a x^2(a-x) dx \Rightarrow \dfrac{a^3}{6} = 2 \dfrac{a^4}{12} = \dfrac{a^4}{6}$. $a \ne 0$이므로 $a=1$.\\
(2) $\int_0^a \{x(a-x) - x^2(a-x)\} dx = 0$ 이어야 하므로 $\left[ \dfrac{a}{2}x^2 - \dfrac{1+a}{3}x^3 + \dfrac{1}{4}x^4 \right]_0^a = 0 \Rightarrow a=2$.

\vspace{0.5cm}
\hrule
\vspace{0.5cm}

\subsection*{필수 예제 12-8}
함수 $f(x)=x^3+3x^2+4x+1$의 역함수를 $g$라고 할 때, 두 곡선 $y=f(x), y=g(x)$와 직선 $y=-x+1$로 둘러싸인 도형의 넓이를 구하여라.

\begin{studybox}
$f(x)$가 증가함수임을 확인하고, 교점이 $y=x$ 위에 있음을 이용한다. 대칭성에 의해 한쪽 넓이를 구해 2배 한다.
\end{studybox}

\textbf{모범답안}\\
$f'(x)=3(x+1)^2+1>0$이므로 증가함수이다. $f(x)=x$의 교점은 $x=-1$이다. $y=-x+1$과 $y=x$의 교점은 $(1/2, 1/2)$이다.
\[ S = 2 \int_{-1}^0 (f(x)-x) dx + 2 \times (\triangle \text{넓이}) = 2 \left[ \dfrac{1}{4}x^4 + x^3 + \dfrac{3}{2}x^2 + x \right]_{-1}^0 + \dfrac{1}{2} = 1 \]

\vspace{0.5cm}
\hrule
\vspace{0.5cm}

\subsection*{필수 예제 12-9}
점 $P(-3, 6)$을 지나고 곡선 $y=x^3-5x^2+x+9$에 접하는 직선의 접점 중 $x$좌표가 음수가 아닌 것을 각각 $Q, R$라고 하자. 이때, 곡선과 선분 $PQ, PR$로 둘러싸인 도형의 넓이를 구하여라.

\textbf{모범답안}\\
접점의 $x$좌표를 $a$라 하면 접선의 방정식은 $y-(a^3-5a^2+a+9)=(3a^2-10a+1)(x-a)$이다. $P(-3, 6)$을 대입하면 $a(a-3)(a+5)=0$. 음수가 아닌 $a=0, 3$. 접선은 $y=x+9, y=-2x$.
\[ S = \triangle \text{넓이} + \int_0^3 \{x^3-5x^2+x+9-(-2x)\} dx = \dfrac{27}{2} + \dfrac{63}{4} = \dfrac{117}{4} \]

\vspace{0.5cm}
\hrule
\vspace{0.5cm}

\subsection*{필수 예제 12-10}
두 곡선 $y=ax^3$과 $y=bx^2+c$가 점 $(2, 8)$에서 접할 때, 다음 물음에 답하여라.
\begin{enumerate}[label=(\arabic*)]
    \item 상수 $a, b, c$의 값을 구하여라.
    \item 두 곡선으로 둘러싸인 도형의 넓이를 구하여라.
\end{enumerate}

\textbf{모범답안}\\
(1) 공통 접점을 지나므로 $8a=8 \Rightarrow a=1$, $4b+c=8$. 미분계수가 같으므로 $3a(2)^2 = 2b(2) \Rightarrow 12=4b \Rightarrow b=3$. 따라서 $c=-4$.\\
(2) $x^3=3x^2-4$의 교점은 $x=-1, 2$(중근). $S = \int_{-1}^2 \{x^3-(3x^2-4)\} dx = \left[ \dfrac{1}{4}x^4 - x^3 + 4x \right]_{-1}^2 = \dfrac{27}{4}$.

\vspace{0.5cm}
\hrule
\vspace{0.5cm}

\subsection*{필수 예제 12-11}
포물선 $y=-x^2+2x$ 위의 점 $P$에서 그은 접선과 포물선 $y=x^2$으로 둘러싸인 도형의 넓이가 최소가 되는 점 $P$의 좌표를 구하여라.

\begin{studybox}
점 $P$의 좌표를 $(a, -a^2+2a)$로 놓고 접선을 구한 뒤, 교점 $\alpha, \beta$와 넓이 공식 $S = \dfrac{1}{6}(\beta-\alpha)^3$을 이용한다.
\end{studybox}

\textbf{모범답안}\\
접선: $y=-2(a-1)x+a^2$. $x^2 = -2(a-1)x+a^2$의 두 근 $\alpha, \beta$에 대하여 $\beta-\alpha = 2\sqrt{2a^2-2a+1}$이다. $S(a) = \dfrac{4}{3}(2a^2-2a+1)^{3/2}$이므로 $a=1/2$일 때 최소가 된다. 점 $P(1/2, 3/4)$.

\end{document}