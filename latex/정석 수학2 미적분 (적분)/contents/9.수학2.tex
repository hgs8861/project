\documentclass[a4paper,11pt]{article}
\usepackage{kotex} % 한글 지원
\usepackage{amsmath, amssymb, amsthm} % 수식 패키지
\usepackage{geometry}
\usepackage{tcolorbox} % 박스 디자인 패키지
\usepackage{enumitem} % 리스트 간격 조절
\tcbuselibrary{skins, breakable}

% --- 여백 및 간격 설정 ---
\geometry{left=2.5cm, right=2.5cm, top=2.5cm, bottom=2.5cm}
\linespread{1.3} % 줄 간격 1.3배
\setlength{\parskip}{1em} % 문단 간격
\setlist[enumerate]{itemsep=5pt, topsep=0pt} % 리스트 항목 간격

% --- 사용자 정의 박스 스타일 (파랑/주황/회색) ---

% 1. 기본정석 (개념): 파랑 계열
\newtcolorbox{conceptbox}[1][]{
  colback=blue!5,
  colframe=blue!60,
  coltitle=white,
  fonttitle=\bfseries,
  title={#1},
  arc=1mm,
  boxrule=0.5mm,
  breakable,
  parskip=1em
}

% 2. Advice (조언): 주황 계열
\newtcolorbox{advicebox}{
  colback=orange!5,
  colframe=orange!60,
  title={Advice},
  fonttitle=\bfseries,
  coltitle=white,
  arc=1mm,
  boxrule=0.5mm,
  breakable,
  parskip=1em
}

% 3. 정석연구 (풀이/심화): 회색 계열
\newtcolorbox{studybox}{
  colback=gray!10,
  colframe=gray!60,
  title={정석연구},
  fonttitle=\bfseries,
  coltitle=white,
  arc=1mm,
  boxrule=0.5mm,
  breakable,
  parskip=1em
}
% ---------------------------------------------

\title{\textbf{수학 II - 제9장 부정적분}}
\author{현경서T}
\date{}

\begin{document}
\maketitle

\section{부정적분의 정의와 계산}

\subsection{부정적분의 정의}

\begin{conceptbox}[기본정석: 부정적분(원시함수)의 정의]
함수 $f(x)$가 주어져 있을 때, $F'(x)=f(x)$인 함수 $F(x)$를 $f(x)$의 \textbf{부정적분} 또는 \textbf{원시함수}라고 한다.

$F(x)$가 함수 $f(x)$의 부정적분의 하나일 때, $f(x)$의 모든 부정적분은 $F(x)+C$의 꼴로 나타내어지며, 이것을 다음과 같이 나타낸다.
\[ \int f(x)dx = F(x) + C \quad (\text{단, } C\text{는 상수}) \]
여기에서 $C$를 \textbf{적분상수}, 함수 $f(x)$를 \textbf{피적분함수}, $x$를 \textbf{적분변수}라 하고, $f(x)$의 부정적분을 구하는 것을 $f(x)$를 \textbf{적분한다}라고 한다.
\end{conceptbox}

\subsubsection*{부정적분과 도함수}
부정적분과 미분은 서로 역연산의 관계에 있다.
\begin{enumerate}
    \item $\dfrac{d}{dx}\left(\int f(x)dx\right) = f(x)$
    \item $\int \left(\dfrac{d}{dx}f(x)\right)dx = f(x) + C$
\end{enumerate}


\subsection{부정적분의 계산}

\begin{conceptbox}[기본정석: 부정적분의 기본 공식]
\begin{enumerate}
    \item $\int k \, dx = kx + C$ (단, $k$는 상수)
    \item $\int x^n \, dx = \dfrac{1}{n+1}x^{n+1} + C$ (단, $n$은 자연수)
    \item $\int k f(x) \, dx = k \int f(x) \, dx$ (단, $k \ne 0$인 상수)
    \item $\int \{f(x) \pm g(x)\} \, dx = \int f(x) \, dx \pm \int g(x) \, dx$
    \item $\int (ax+b)^n \, dx = \dfrac{1}{a(n+1)}(ax+b)^{n+1} + C$ (단, $a \ne 0, n$은 자연수)
\end{enumerate}
\end{conceptbox}

적분은 미분의 역 연산 이므로 우변을 미분한 것이  좌변의 피적분 함수와 같음.
\newpage

\section{필수 예제 및 유제}

\subsection*{[필수 예제 9-1] 부정적분의 계산}
다음 부정적분을 구하여라.
\begin{enumerate}
    \item $\int (x^2+\sqrt{2}x+1)(x^2-\sqrt{2}x+1) dx$
    \item $\int \dfrac{x^4+x^2+1}{x^2-x+1} dx$
    \item $\int \dfrac{y^3}{y+1} dy + \int \dfrac{1}{y+1} dy$
    \item $\int (x+1)^3 dx - \int (x-1)^3 dx$
\end{enumerate}

\begin{studybox}
(1) 적분에서는 다음에 주의해야 한다.
\[ \int f(x)g(x)dx \ne \left( \int f(x)dx \right) \left( \int g(x)dx \right) \]
따라서 피적분함수를 전개한 다음 부정적분을 구한다.\\
(2) 피적분함수를 약분하여 다항함수로 고칠 수 있는지 확인한다. \\
(3), (4) 각 함수를 바로 적분하는 것은 복잡하다.
\[ \int f(x)dx \pm \int g(x)dx = \int \{ f(x) \pm g(x) \} dx \quad (\text{복부호동순}) \]
를 이용하여 먼저 피적분함수를 간단히 해 보자.
\end{studybox}

\textbf{모범답안}
\begin{enumerate}
    \item $(\text{준식}) = \int \{ (x^2+1)+\sqrt{2}x \} \{ (x^2+1)-\sqrt{2}x \} dx = \int \{ (x^2+1)^2 - 2x^2 \} dx$
    \[ = \int (x^4+1) dx = \dfrac{1}{5}x^5 + x + C \]
    \item $(\text{준식}) = \int \dfrac{(x^2+x+1)(x^2-x+1)}{x^2-x+1} dx = \int (x^2+x+1) dx$
    \[ = \dfrac{1}{3}x^3 + \dfrac{1}{2}x^2 + x + C \]
    \item $(\text{준식}) = \int \dfrac{y^3+1}{y+1} dy = \int \dfrac{(y+1)(y^2-y+1)}{y+1} dy = \int (y^2-y+1) dy$
    \[ = \dfrac{1}{3}y^3 - \dfrac{1}{2}y^2 + y + C \]
    \item $(\text{준식}) = \int \{ (x+1)^3 - (x-1)^3 \} dx = \int (6x^2+2) dx$
    \[ = 2x^3 + 2x + C \]
\end{enumerate}

\vspace{0.5cm}
\hrule
\vspace{0.5cm}

\subsection*{[필수 예제 9-2] 함숫값과 부정적분}
다음 물음에 답하여라.
\begin{enumerate}
    \item $f'(x)=2+4x+3x^2$ 이고 $f(0)=3$ 인 함수 $f(x)$ 를 구하여라.
    \item 함수 $y=3x^2-ax$ 의 부정적분 중에서 $x=0$ 일 때 함숫값이 1이고, $x=2$ 일 때 함숫값이 5인 것을 구하여라. 단, $a$는 상수이다.
    \item 두 점 $(0, -2), (1, 0)$ 을 지나는 곡선 $y=f(x)$ 위의 점 $(x, y)$ 에서의 접선의 기울기가 $3x^2-6x+4$ 에 정비례할 때, $f(x)$ 를 구하여라.
\end{enumerate}

\begin{studybox}
도함수 $f'(x)$ 로부터 원시함수 $f(x)$ 를 구할 때에는 다음을 이용한다.
\[ f(x) = \int f'(x) dx \]
이때 나타나는 적분상수는 주어진 조건을 써서 알맞게 정한다.
\end{studybox}

\textbf{모범답안}
\begin{enumerate}
    \item $f(x) = \int (2+4x+3x^2) dx = 2x + 2x^2 + x^3 + C$
    $f(0)=3$ 이므로 $C=3$
    $\therefore f(x) = x^3 + 2x^2 + 2x + 3$
    \item $f(x) = \int (3x^2-ax) dx$ 라고 하면 $f(x) = x^3 - \dfrac{1}{2}ax^2 + C$
    $f(0)=1, f(2)=5$ 이므로 $C=1, \quad 8-2a+C=5 \implies a=2$
    $\therefore f(x) = x^3 - x^2 + 1$
    \item 문제의 조건으로부터 $f'(x) = k(3x^2-6x+4) (k \ne 0)$ 로 놓으면
    $f(x) = \int k(3x^2-6x+4) dx = k(x^3-3x^2+4x) + C$
    $f(0)=-2, f(1)=0$ 이므로 $C=-2, \quad 2k+C=0 \implies k=1$
    $\therefore f(x) = x^3 - 3x^2 + 4x - 2$
\end{enumerate}

\vspace{0.5cm}
\hrule
\vspace{0.5cm}

\subsection*{[필수 예제 9-3] 도함수의 그래프와 부정적분}
사차함수 $f(x)$ 의 도함수 $y=f'(x)$ 의 그래프가 오른쪽 그림(교재 참조)과 같다. $f(x)$ 의 극댓값이 0이고, 극솟값이 -16일 때, 함수 $f(x)$ 를 구하여라.
(단, 그래프는 $x=-2, 0, 2$ 에서 $x$축과 만나고 원점 대칭인 삼차함수 형태임)

\begin{studybox}
주어진 그래프에서 $f'(x)=0$ 의 해는 $x=-2, 0, 2$ 이다.
따라서 증감표를 작성해 보면 $f(x)$ 는 $x=0$ 에서 극대, $x=\pm 2$ 에서 극소임을 알 수 있다.
주어진 그래프로부터 먼저 $f'(x)$ 의 꼴을 정하고 $f(x) = \int f'(x) dx$ 와 주어진 조건을 이용하여 $f(x)$ 를 구해 보아라.
\end{studybox}

\textbf{모범답안}
$f'(x)$ 는 삼차함수이고 $f'(x)=0$ 의 해가 $x=-2, 0, 2$ 이므로
\[ f'(x) = ax(x-2)(x+2) = a(x^3-4x) \quad (a>0) \]
로 놓으면
\[ f(x) = \int f'(x) dx = \int a(x^3-4x) dx = \dfrac{1}{4}ax^4 - 2ax^2 + C \quad \cdots \text{\textcircled{1}} \]
$f'(x)$ 의 부호를 조사하면 함수 $f(x)$ 는 $x=0$ 에서 극대이고, $x=\pm 2$ 에서 극소이다.
극댓값이 0이므로 $f(0)=0 \implies C=0$
극솟값이 -16이므로 $f(-2)=-16$ 또는 $f(2)=-16$
그런데 $f(-2)=f(2)=4a-8a+C$ 이므로 $-4a+C=-16$
\textcircled{1}에 대입하면 $f(x) = x^4 - 8x^2$

\vspace{0.5cm}
\hrule
\vspace{0.5cm}

\subsection*{[필수 예제 9-4] 도함수가 주어진 경우}
두 다항함수 $f(x), g(x)$가
\[ \dfrac{d}{dx}\{f(x)+g(x)\} = 2x+1, \quad \dfrac{d}{dx}\{f(x)g(x)\} = 3x^2-2x+2 \]
를 만족시킨다. $f(0)=2, g(0)=-1$일 때, $f(x), g(x)$를 구하여라.

\begin{studybox}
$\dfrac{d}{dx}F(x)=f(x) \iff F(x) = \int f(x)dx$를 이용하여 먼저 $f(x)+g(x)$와 $f(x)g(x)$를 구한다.
\end{studybox}

\textbf{모범답안}

$f(x)+g(x) = \int (2x+1)dx = x^2+x+C_1$.
$x=0$을 대입하면 $C_1 = 2+(-1)=1$.
$\therefore f(x)+g(x) = x^2+x+1$

$f(x)g(x) = \int (3x^2-2x+2)dx = x^3-x^2+2x+C_2$.
$x=0$을 대입하면 $C_2 = 2 \times (-1) = -2$.
$\therefore f(x)g(x) = x^3-x^2+2x-2 = (x-1)(x^2+2)$

$f(x)+g(x) = (x-1) + (x^2+2)$이고 $f(0)=2, g(0)=-1$이므로
$f(x) = x^2+2, \quad g(x) = x-1$

\end{document}