\documentclass[a4paper,11pt]{article}
\usepackage{kotex}
\usepackage{amsmath, amssymb, amsthm}
\usepackage{geometry}
\usepackage{tcolorbox}
\usepackage{enumitem}
\tcbuselibrary{skins, breakable}

% --- 여백 및 간격 설정 ---
\geometry{left=2.5cm, right=2.5cm, top=2.5cm, bottom=2.5cm}
\linespread{1.4}
\setlength{\parskip}{1em}
\setlist[enumerate]{itemsep=5pt, topsep=0pt}

% --- 사용자 정의 박스 스타일 ---
\newtcolorbox{conceptbox}[1][]{
  colback=blue!5, colframe=blue!60, coltitle=white, fonttitle=\bfseries,
  title={#1}, arc=1mm, boxrule=0.5mm, breakable, parskip=1em
}
\newtcolorbox{advicebox}{
  colback=orange!5, colframe=orange!60, title={Advice}, fonttitle=\bfseries,
  coltitle=white, arc=1mm, boxrule=0.5mm, breakable, parskip=1em
}
\newtcolorbox{studybox}{
  colback=gray!10, colframe=gray!60, title={정석연구}, fonttitle=\bfseries,
  coltitle=white, arc=1mm, boxrule=0.5mm, breakable, parskip=1em
}

\title{\textbf{미적분 - 제16장 여러 가지 정적분에 관한 문제}}
\author{수학의 정석}
\date{}

\begin{document}

\maketitle

\section*{§1. 구분구적법}

\begin{conceptbox}[기본 정석]
다음과 같은 방법으로 평면도형의 넓이나 입체도형의 부피를 구하는 것을 \textbf{구분구적법}이라고 한다. 
\begin{enumerate}[label=(\text{\roman*})]
    \item 주어진 도형을 충분히 작은 $n$ 개의 기본 도형으로 나눈다. 
    \item 기본 도형들의 넓이의 합 $S_n$ 또는 부피의 합 $V_n$ 을 구한다. 
    \item $\displaystyle \lim_{n \to \infty} S_n$ 또는 $\displaystyle \lim_{n \to \infty} V_n$ 을 구한다. 
\end{enumerate}
\end{conceptbox}

\begin{advicebox}
곡선으로 둘러싸인 도형의 넓이를 구할 때에는 주어진 도형을 다각형으로 근사시키는 방법을 생각해야 한다. 반지름의 길이가 $r$ 인 원의 넓이 $S$ 는 원에 내접하는 정$n$각형의 넓이 $S_n$ 에 대하여 $S = \displaystyle \lim_{n \to \infty} S_n = \pi r^2$ 임을 알 수 있다. 
\end{advicebox}

\section*{§2. 정적분과 급수}

\begin{conceptbox}[기본 정석]
함수 $f(x)$ 가 닫힌구간 $[a, b]$ 에서 연속일 때, 이 구간을 $n$ 등분하여 양 끝점과 각 분점의 $x$ 좌표를 차례로 $a=x_0, x_1, x_2, \dots, x_n=b$ 라 하고, 각 소구간의 길이를 $\Delta x = \dfrac{b-a}{n}$ 라고 하면 
\[ \int_{a}^{b} f(x) dx = \lim_{n \to \infty} \sum_{k=1}^{n} f(x_k) \Delta x = \lim_{n \to \infty} \sum_{k=1}^{n} f\left(a + \dfrac{b-a}{n}k\right) \dfrac{b-a}{n} \]
\end{conceptbox}


\begin{advicebox}
\textbf{급수를 정적분으로 고치는 방법} 
일반적으로 $\displaystyle \lim_{n \to \infty} \sum_{k=1}^{n} f\left(a + \dfrac{p}{n}k\right) \dfrac{p}{n}$ 꼴의 급수는 다음과 같이 고칠 수 있다.
\begin{enumerate}[label=\text{\arabic*°}]
    \item $a + \dfrac{p}{n}k = x$ 로 놓으면 $\dfrac{p}{n} = dx$, 구간은 $[a, a+p] \implies \displaystyle \int_{a}^{a+p} f(x) dx$
    \item $\dfrac{p}{n}k = x$ 로 놓으면 $\dfrac{p}{n} = dx$, 구간은 $[0, p] \implies \displaystyle \int_{0}^{p} f(a+x) dx$
    \item $\dfrac{k}{n} = x$ 로 놓으면 $\dfrac{1}{n} = dx$, 구간은 $[0, 1] \implies \displaystyle p \int_{0}^{1} f(a+px) dx$
\end{enumerate}
\end{advicebox}

\section*{§3. 정적분으로 정의된 함수}

\begin{conceptbox}[기본 정석]
함수 $f(x)$ 가 연속함수일 때 
\begin{enumerate}
    \item $\displaystyle \dfrac{d}{dx} \int_{a}^{x} f(t) dt = f(x)$ (단, $a$ 는 상수) 
    \item $\displaystyle \dfrac{d}{dx} \int_{h(x)}^{g(x)} f(t) dt = f(g(x))g^{\prime}(x) - f(h(x))h^{\prime}(x)$ 
    \item $\displaystyle \lim_{x \to a} \dfrac{1}{x-a} \int_{a}^{x} f(t) dt = f(a)$ 
\end{enumerate}
\end{conceptbox}


\section*{§4. 정적분과 부등식}

\begin{conceptbox}[기본 정석]
\begin{enumerate}
    \item 함수 $f(x)$ 가 구간 $[a, b]$ 에서 연속이고 $f(x) \ge 0$ 일 때, 이 구간에 속하는 $\alpha$ 에 대하여 $f(\alpha) > 0$ 이면 $\displaystyle \int_{a}^{b} f(x) dx > 0$ 
    \item 두 함수 $f(x), g(x)$ 가 구간 $[a, b]$ 에서 연속이고 $f(x) \ge g(x)$ 일 때, 이 구간에 속하는 $\alpha$ 에 대하여 $f(\alpha) > g(\alpha)$ 이면 $\displaystyle \int_{a}^{b} f(x) dx > \int_{a}^{b} g(x) dx$ 
    \item \textbf{적분의 평균값 정리}: 함수 $f(x)$ 가 구간 $[a, b]$ 에서 연속이면 $\displaystyle \dfrac{1}{b-a} \int_{a}^{b} f(x) dx = f(c)$ 를 만족시키는 $c$ 가 구간 $(a, b)$ 에 적어도 하나 존재한다. 
\end{enumerate}
\end{conceptbox}

\newpage

\section*{필수 예제}

\subsection*{필수 예제 16-1}
곡선 $y=x^{3}$ 과 $x$ 축 및 직선 $x=1$ 로 둘러싸인 도형의 넓이를 구분구적법으로 구하여라. 

\begin{studybox}
구간 $[0, 1]$ 을 $n$ 등분하여 각 소구간의 오른쪽 끝 점의 함숫값을 세로의 길이로 하는 직사각형들의 넓이의 합 $T_n$ 의 극한값을 구한다. 
\end{studybox}

\textbf{모범답안} \\
$T_{n}=(\frac{1}{n})^{3}\frac{1}{n}+(\frac{2}{n})^{3}\frac{1}{n}+\cdot\cdot\cdot+(\frac{n}{n})^{3}\frac{1}{n} = \frac{1}{n^{4}}(1^{3}+2^{3}+\cdot\cdot\cdot+n^{3}) = \frac{1}{n^4}\{\frac{n(n+1)}{2}\}^2 = \frac{1}{4}(1+\frac{1}{n})^2$ \\
$\therefore S = \displaystyle \lim_{n \to \infty} T_n = \mathbf{\dfrac{1}{4}}$ 

\subsection*{필수 예제 16-2}
밑면의 반지름의 길이가 $r$ 이고 높이가 $h$ 인 원뿔의 부피 $V$ 를 구분구적법으로 구하여라. 

\textbf{모범답안} \\
원뿔을 밑면에 평행한 평면으로 $n$ 등분하여 $n-1$ 개의 원기둥을 만들면 부피의 합 $V_n$ 은 \\
$V_{n}=\pi\times\frac{r^{2}h}{n^{3}}\{1^{2}+2^{2}+\cdot\cdot\cdot+(n-1)^{2}\} = \frac{\pi r^{2}h}{n^{3}}\times\frac{(n-1)n(2n-1)}{6}$ \\
$\therefore V = \displaystyle \lim_{n \to \infty} V_n = \mathbf{\dfrac{1}{3} \pi r^2 h}$ 

\subsection*{필수 예제 16-3}
정적분을 이용하여 다음 급수의 합을 구하여라. 
\begin{enumerate}[label=(\arabic*)]
    \item $\displaystyle \lim_{n \to \infty} \dfrac{(n+1)^3 + (n+2)^3 + \dots + (2n)^3}{1^3 + 2^3 + \dots + n^3}$
    \item $\displaystyle \lim_{n \to \infty} \sum_{k=1}^{n} \dfrac{k}{n^2} \cos \dfrac{\pi k^2}{2n^2}$
    \item $\displaystyle \lim_{n \to \infty} \dfrac{1}{n^3} \sum_{k=1}^{n} k^2 e^{k/n}$
\end{enumerate}

\textbf{모범답안} \\
(1) $\displaystyle \dfrac{\int_0^1 (1+x)^3 dx}{\int_0^1 x^3 dx} = \mathbf{15}$ \quad (2) $\displaystyle \int_0^1 x \cos(\frac{\pi}{2}x^2) dx = \mathbf{\dfrac{1}{\pi}}$ \quad (3) $\displaystyle \int_0^1 x^2 e^x dx = \mathbf{e-2}$ 

\subsection*{필수 예제 16-4}
정적분을 이용하여 다음 급수의 합을 구하여라. 
\begin{enumerate}[label=(\arabic*)]
    \item $\displaystyle \lim_{n \to \infty} \dfrac{\pi}{n^2} (\cos\dfrac{\pi}{n} + 2\cos\dfrac{2\pi}{n} + \dots + n\cos\dfrac{n\pi}{n})$
    \item $\displaystyle \lim_{n \to \infty} (\ln \sqrt[n]{\dfrac{n+1}{n}} + \ln \sqrt[n]{\dfrac{n+2}{n}} + \dots + \ln \sqrt[n]{\dfrac{n+n}{n}})$
\end{enumerate}

\textbf{모범답안} \\
(1) $\displaystyle \pi \int_0^1 x \cos \pi x dx = \mathbf{-\dfrac{2}{\pi}}$ \quad (2) $\displaystyle \int_0^1 \ln(1+x) dx = \mathbf{2 \ln 2 - 1}$ 

\subsection*{필수 예제 16-5}
정적분을 이용하여 다음 급수의 합을 구하여라. 
\begin{enumerate}[label=(\arabic*)]
    \item $\displaystyle \lim_{n \to \infty} \dfrac{1}{n^3} \{ \sqrt{n^2-1^2} + 2\sqrt{n^2-2^2} + \dots + (n-1)\sqrt{n^2-(n-1)^2} \}$
    \item $\displaystyle \lim_{n \to \infty} ( \dfrac{n+2}{n^2+1} + \dfrac{n+4}{n^2+4} + \dots + \dfrac{n+2n}{n^2+k^2} )$
\end{enumerate}

\textbf{모범답안} \\
(1) $\displaystyle \int_0^1 x \sqrt{1-x^2} dx = \mathbf{\dfrac{1}{3}}$ \quad (2) $\displaystyle \int_0^1 \dfrac{1+2x}{1+x^2} dx = \mathbf{\dfrac{\pi}{4} + \ln 2}$ 

\subsection*{필수 예제 16-6}
$n$ 이 2 이상의 자연수일 때, 다음 부등식을 증명하여라. 
\[ \ln(n+1) < 1 + \dfrac{1}{2} + \dfrac{1}{3} + \dots + \dfrac{1}{n} < 1 + \ln n \]

\textbf{모범답안} \\
$y=1/x$ 의 그래프에서 $S = 1 + \dots + 1/n$ 이라 하면, 영역의 넓이 비교에 의해 $S > \int_1^{n+1} \frac{1}{x} dx = \ln(n+1)$ 이고, $S-1 < \int_1^n \frac{1}{x} dx = \ln n$ 이므로 성립한다. 

\subsection*{필수 예제 16-7}
다음 극한값을 구하여라. 
\begin{enumerate}[label=(\arabic*)]
    \item $\displaystyle \lim_{x \to 1} \dfrac{1}{x-1} \int_{1}^{x} (t^5+3t^3+2t) dt$
    \item $\displaystyle \lim_{x \to 1} \dfrac{1}{x^3-1} \int_{1}^{x^2} t^3 e^t dt$
\end{enumerate}

\textbf{모범답안} \\
(1) $F^{\prime}(1) = \mathbf{6}$ \quad (2) $F^{\prime}(1) \times \displaystyle \lim_{x \to 1} \dfrac{x^2-1}{x^3-1} = e \times \dfrac{2}{3} = \mathbf{\dfrac{2}{3}e}$ 

\subsection*{필수 예제 16-8}
다음 극한값을 구하여라. 
\begin{enumerate}[label=(\arabic*)]
    \item $\displaystyle \lim_{h \to 0} \dfrac{1}{h} \int_{1-h}^{1+h} \dfrac{\cos^3 \pi x}{1 + \sin \pi x} dx$
    \item $\displaystyle \lim_{t \to \infty} t \int_{0}^{\frac{2}{t}} \dfrac{|x-1|}{x^2+2} dx$
\end{enumerate}

\textbf{모범답안} \\
(1) $2F^{\prime}(1) = \mathbf{-2}$ \quad (2) $\displaystyle \lim_{h \to 0+} \dfrac{1}{h} \int_0^{2h} f(x) dx = 2f(0) = \mathbf{1}$ 

\subsection*{필수 예제 16-9}
연속함수 $f(x)$ 가 다음 등식을 만족시킬 때, 상수 $a$ 의 값과 $f(x)$ 를 구하여라. 
\begin{enumerate}[label=(\arabic*)]
    \item $\displaystyle \int_{\ln 3}^{x} e^t f(t) dt = e^{2x} - ae^x + 3$
    \item $\displaystyle \int_{a}^{\ln x} f(t) dt = x^2 - x$
\end{enumerate}

\textbf{모범답안} \\
(1) $a=4, f(x)=\mathbf{2e^x-4}$ \quad (2) $a=0, f(x)=\mathbf{2e^{2x}-e^x}$ 

\subsection*{필수 예제 16-10}
$-\frac{\pi}{2} < x < \frac{\pi}{2}$ 에서 정의된 함수 $f(x)$ 에 대하여 $f^{\prime}(x)$ 가 연속함수이고 $f(x) = \tan x - x - \displaystyle \int_{0}^{x} f^{\prime}(u) \tan^2 u du$ 일 때, $f^{\prime}(x)$ 와 $f(x)$ 를 구하여라. 

\textbf{모범답안} \\
양변을 미분하면 $f^{\prime}(x) = \sec^2 x - 1 - f^{\prime}(x) \tan^2 x \implies f^{\prime}(x) = \mathbf{\sin^2 x}$, 이를 적분하고 $f(0)=0$ 을 대입하면 $f(x) = \mathbf{\dfrac{1}{2}x - \dfrac{1}{4}\sin 2x}$ 

\subsection*{필수 예제 16-11}
함수 $f(x) = e^x(ax+b)$ 가 모든 실수 $x$ 에 대하여 $f(x) = \displaystyle \int_{0}^{x} (x-t)f^{\prime}(t) dt + e^x + x$ 를 만족시킬 때, 상수 $a, b$ 의 값을 구하여라. 

\textbf{모범답안} \\
양변을 두 번 미분하여 정리하면 $f^{\prime\prime}(x) = f^{\prime}(x) + e^x$. 대입하여 계수를 비교하면 $\mathbf{a=1, b=1}$ 

\subsection*{필수 예제 16-12}
다음 물음에 답하여라. 
\begin{enumerate}[label=(\arabic*)]
    \item $f(x) = \displaystyle \int_{0}^{x} (1+\cos t) \sin t dt$ ($-2\pi < x < 2\pi$) 의 극값을 구하여라.
    \item $f(x) = \displaystyle \int_{x}^{x+1} e^{t^3-7t} dt$ 가 극대가 되는 $x$ 의 값을 구하여라.
\end{enumerate}

\textbf{모범답안} \\
(1) 극댓값 $f(-\pi)=f(\pi)=\mathbf{2}$, 극솟값 $f(0)=\mathbf{0}$ \quad (2) $f'(x)=0$ 에서 $\mathbf{x=-2}$ 

\subsection*{필수 예제 16-13}
$n$ 이 자연수일 때, 다음 부등식을 증명하여라. 
\begin{enumerate}[label=(\arabic*)]
    \item $\dfrac{1}{2(n+1)} < \displaystyle \int_{0}^{1} \dfrac{x^n}{1+x^2} dx < \dfrac{1}{n+1}$
    \item $\dfrac{1}{3} < \displaystyle \int_{0}^{1} x^{( \sin x + \cos x )^2} dx < \dfrac{1}{2}$
\end{enumerate}

\textbf{모범답안} \\
(1) $0 \le x \le 1$ 에서 $\frac{1}{2} \le \frac{1}{1+x^2} \le 1$ 이므로 성립. (2) $1 \le (\sin x + \cos x)^2 \le 2$ 이므로 성립. 

\subsection*{필수 예제 16-14}
함수 $f(x)$ 가 구간 $[a, b]$ 에서 연속이면 $\dfrac{1}{b-a} \int_a^b f(x) dx = f(c)$ 를 만족시키는 $c$ 가 구간 $(a, b)$ 에 적어도 하나 존재함을 증명하여라. 

\textbf{모범답안} \\
최대·최소 정리에 의해 $m \le f(x) \le M$ 이므로 $m(b-a) \le \int_a^b f(x) dx \le M(b-a)$ 가 성립하며, 사잇값의 정리에 의해 만족하는 $c$ 가 존재한다. 

\end{document}