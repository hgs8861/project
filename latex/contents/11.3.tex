\documentclass{article}
\usepackage[a4paper,left=3cm, right=3cm, top=2cm, bottom=2cm]{geometry}
\usepackage{amsmath}
\usepackage{graphicx}
\usepackage{caption}
\usepackage{setspace}
\usepackage{xcolor}
\usepackage{titlesec}
\usepackage{amssymb}
\usepackage{tcolorbox}
\usepackage{wrapfig}

\graphicspath{{graph/}}
\title{11.3 The Integral Test and Estimates of Sums}
\date{}
\author{}
\setstretch{1.3} 

% \subsection* 형식 지정 (번호 없음)
\titleformat{name=\section, numberless}
  {\normalfont\large\bfseries\color{blue}}
  {}
  {0pt}
  {}
\geometry{a4paper, margin=1in}

\begin{document}
\maketitle

In general, it is difficult to find the exact sum of a series. In this section, we develop a test that enables us to determine whether a series is convergent or divergent without explicitly finding its sum.

\section*{The Integral Test}

\begin{tcolorbox}[
    colback=white,
    colframe=orange!80!white,
    title=The Integral Test,
    boxrule=0.5mm,
    arc=3mm
    ]
    Suppose \(f\) is a continuous, positive, decreasing function on \([1, \infty)\) and let \(a_n = f(n)\). Then the series \( \sum_{n=1}^{\infty} a_n \) is convergent if and only if the improper integral \( \int_1^\infty f(x) \,dx \) is convergent. In other words:
    \begin{itemize}
        \item[(i)] If \( \int_1^\infty f(x) \,dx \) is \textbf{convergent}, then \( \sum_{n=1}^{\infty} a_n \) is \textbf{convergent}.
        \item[(ii)] If \( \int_1^\infty f(x) \,dx \) is \textbf{divergent}, then \( \sum_{n=1}^{\infty} a_n \) is \textbf{divergent}.
    \end{itemize}
    \textbf{Note:} It is not necessary that \(f\) be always decreasing. It is sufficient if \(f\) is ultimately decreasing, i.e., decreasing for \(x\) larger than some number \(N\).
\end{tcolorbox}

\subsection*{EXAMPLE 1}
Test the series \( \sum_{n=1}^{\infty} \dfrac{1}{n^2+1} \) for convergence or divergence.

\textbf{SOLUTION:}
The function \(f(x) = \dfrac{1}{x^2+1}\) is continuous, positive, and decreasing on \([1, \infty)\) because \(x^2+1\) is an increasing function. We evaluate the integral:
\begin{align*}
    \int_1^\infty \dfrac{1}{x^2+1} \,dx &= \lim_{t\to\infty} \int_1^t \dfrac{1}{x^2+1} \,dx \\
    &= \lim_{t\to\infty} \left[ \tan^{-1}(x) \right]_1^t \\
    &= \lim_{t\to\infty} (\tan^{-1}(t) - \tan^{-1}(1)) = \dfrac{\pi}{2} - \dfrac{\pi}{4} = \dfrac{\pi}{4}
\end{align*}
Since this improper integral is convergent, the series is \textbf{convergent} by the Integral Test.

\section*{The p-series}
The series \( \sum_{n=1}^{\infty} \dfrac{1}{n^p} \) is called the \textbf{p-series}.

\begin{tcolorbox}[
    colback=white,
    colframe=orange!80!white,
    title=Convergence of a p-series,
    boxrule=0.5mm,
    arc=3mm
    ]
    The p-series \( \sum_{n=1}^{\infty} \dfrac{1}{n^p} \) is \textbf{convergent} if \(p > 1\) and \textbf{divergent} if \(p \le 1\).
\end{tcolorbox}

\subsection*{EXAMPLE 2}
(a) The series \( \sum_{n=1}^{\infty} \dfrac{1}{n^3} \) is convergent because it is a p-series with \(p = 3 > 1\). \\
(b) The series \( \sum_{n=1}^{\infty} \dfrac{1}{n^{1/3}} \) is divergent because it is a p-series with \(p = 1/3 < 1\).

\subsection*{EXAMPLE 4}
Determine whether the series \( \sum_{n=1}^{\infty} \dfrac{\ln n}{n} \) converges or diverges.

\textbf{SOLUTION:}
The function \(f(x) = \dfrac{\ln x}{x}\) is positive and continuous for \(x>1\). To check if it's decreasing, we find the derivative:
\[ f'(x) = \dfrac{x(1/x) - (\ln x)(1)}{x^2} = \dfrac{1 - \ln x}{x^2} \]
\(f'(x) < 0\) when \(1 - \ln x < 0\), or \(\ln x > 1\), which means \(x > e\). Thus, \(f\) is decreasing for \(x > e\).

Applying the Integral Test:
\[ \int_1^\infty \dfrac{\ln x}{x} \,dx = \lim_{t\to\infty} \left[ \dfrac{(\ln x)^2}{2} \right]_1^t = \lim_{t\to\infty} \dfrac{(\ln t)^2}{2} = \infty \]
The integral is divergent, so the series is \textbf{divergent}.

\section*{Estimating the Sum of a Series}
For a convergent series \( \sum a_n \), we can approximate its sum \(s\) with a partial sum \(s_n\). The error in this approximation is the remainder, \(R_n = s - s_n\).

\begin{tcolorbox}[
    colback=white,
    colframe=orange!80!white,
    title=Remainder Estimate for the Integral Test,
    boxrule=0.5mm,
    arc=3mm
    ]
    Suppose \(f(k) = a_k\), where \(f\) is a continuous, positive, decreasing function for \(x \ge n\) and \( \sum a_n \) is convergent. If \(R_n = s - s_n\), then:
    \[ \int_{n+1}^\infty f(x) \,dx \le R_n \le \int_n^\infty f(x) \,dx \]
\end{tcolorbox}

\subsection*{EXAMPLE 5}
(a) Approximate the sum of the series \( \sum 1/n^3 \) by using the sum of the first 10 terms. Estimate the error involved in this approximation. \\
(b) How many terms are required to ensure that the sum is accurate to within 0.0005?

\textbf{SOLUTION:}
In both parts, we need to know \( \int_n^\infty \dfrac{1}{x^3} \,dx = \lim_{t\to\infty} \left[ -\dfrac{1}{2x^2} \right]_n^t = \dfrac{1}{2n^2} \).
\begin{itemize}
    \item[(a)] \( s_{10} = 1 + \dfrac{1}{2^3} + \dots + \dfrac{1}{10^3} \approx 1.1975 \). The remainder \(R_{10}\) satisfies:
    \[ R_{10} \le \int_{10}^\infty \dfrac{1}{x^3} \,dx = \dfrac{1}{2(10^2)} = \dfrac{1}{200} = 0.005 \]
    The error is at most 0.005.
    \item[(b)] We require \(R_n < 0.0005\). Since \(R_n \le \int_n^\infty f(x) dx\), we need:
    \[ \dfrac{1}{2n^2} < 0.0005 \Rightarrow 2n^2 > \dfrac{1}{0.0005} = 2000 \Rightarrow n^2 > 1000 \Rightarrow n > \sqrt{1000} \approx 31.6 \]
    We need 32 terms to ensure accuracy to within 0.0005.
\end{itemize}

By adding the remainder estimate to the partial sum, we can get an improved estimate for the total sum \(s\):
\[ s_n + \int_{n+1}^\infty f(x) \,dx \le s \le s_n + \int_n^\infty f(x) \,dx \]

\end{document}