\documentclass{article}
\usepackage[a4paper,left=3cm, right=3cm, top=2cm, bottom=2cm]{geometry}
\usepackage{amsmath}
\usepackage{graphicx}
\usepackage{caption}
\usepackage{setspace}
\usepackage{xcolor}
\usepackage{titlesec}
\usepackage{amssymb}
\usepackage{tcolorbox}
\usepackage{wrapfig}
\usepackage{amsthm} % For proofs
\usepackage{amsfonts} % For \mathbb{R}

\graphicspath{{graph/}}
\title{11.10 Taylor and Maclaurin Series}
\date{}
\author{}
\setstretch{1.3}

% \subsection* format (no number)
\titleformat{name=\section, numberless}
  {\normalfont\large\bfseries\color{blue}}
  {}
  {0pt}
  {}
\geometry{a4paper, margin=1in}

% Proof environment style
\newtheoremstyle{mystyle}% name
  {}% Space above
  {}% Space below
  {\itshape}% Body font
  {}% Indent amount
  {\bfseries}% Theorem head font
  {.}% Punctuation after theorem head
  {.5em}% Space after theorem head
  {}% Theorem head spec (can be left empty, meaning `normal')
\theoremstyle{mystyle}
\newtheorem*{theorem}{Theorem} % Use unnumbered theorem environment

\begin{document}
\maketitle

This section investigates which functions have power series representations and how to find such representations. It focuses on Taylor series and Maclaurin series.

\section*{Definitions of Taylor Series and Maclaurin Series}
Suppose a function \(f\) can be represented by a power series centered at \(a\):
\begin{equation*}
 f(x) = c_0 + c_1(x-a) + c_2(x-a)^2 + c_3(x-a)^3 + \cdots = \sum_{n=0}^{\infty} c_n (x-a)^n \quad |x-a| < R \label{eq:power_series_def}
\end{equation*}
By evaluating \(f(a)\) and its successive derivatives \(f'(a), f''(a), f'''(a), \ldots\), we can determine the coefficients \(c_n\).
\begin{align*}
    f(a) &= c_0 \\
    f'(a) &= c_1 \\
    f''(a) &= 2c_2 \\
    f'''(a) &= 2 \cdot 3 c_3 = 3! c_3 \\
    &\vdots \\
    f^{(n)}(a) &= n! c_n
\end{align*}
Solving for \(c_n\) gives the following theorem:

\begin{tcolorbox}[
    colback=white,
    colframe=orange!80!white,
    title=Theorem 5,
    boxrule=0.5mm,
    arc=3mm
    ]
    If \(f\) has a power series representation (expansion) at \(a\), that is, if
    \[ f(x) = \sum_{n=0}^{\infty} c_n (x-a)^n \quad |x-a| < R \]
    then its coefficients are given by the formula
    \[ c_n = \dfrac{f^{(n)}(a)}{n!} \]
\end{tcolorbox}

Substituting this formula back into the series gives the \textbf{Taylor series} of the function \(f\) at \(a\):
\begin{equation*}
 f(x) = \sum_{n=0}^{\infty} \dfrac{f^{(n)}(a)}{n!} (x-a)^n = f(a) + \dfrac{f'(a)}{1!}(x-a) + \dfrac{f''(a)}{2!}(x-a)^2 + \dfrac{f'''(a)}{3!}(x-a)^3 + \cdots \label{eq:taylor_series}
\end{equation*}

For the special case \(a=0\), the Taylor series becomes the \textbf{Maclaurin series}:
\begin{equation*}
 f(x) = \sum_{n=0}^{\infty} \dfrac{f^{(n)}(0)}{n!} x^n = f(0) + \dfrac{f'(0)}{1!}x + \dfrac{f''(0)}{2!}x^2 + \cdots \label{eq:maclaurin_series}
\end{equation*}
\textbf{Note 1:} Having a Taylor series doesn't guarantee the series sums to \(f(x)\).\\
\textbf{Note 2:} A power series representation at \(a\) is unique; it must be the Taylor series. Representations found in Section 11.9 are Taylor series.

\subsection*{EXAMPLE 1}
Confirm the Maclaurin series for \(f(x) = \dfrac{1}{1-x}\).\\
\textbf{SOLUTION:}
We found \(f^{(n)}(0) = n!\). Therefore,
\[ c_n = \dfrac{f^{(n)}(0)}{n!} = \dfrac{n!}{n!} = 1 \]
The Maclaurin series is \( \sum_{n=0}^{\infty} x^n = 1 + x + x^2 + \cdots \), which matches the geometric series representation, valid for \(|x|<1\).

\subsection*{EXAMPLE 2}
Find the Maclaurin series for \(f(x) = e^x\) and its radius of convergence.\\
\textbf{SOLUTION:}
Since \(f^{(n)}(x) = e^x\) for all \(n\), we have \(f^{(n)}(0) = e^0 = 1\) for all \(n\). The Maclaurin series is:
\[ \sum_{n=0}^{\infty} \dfrac{f^{(n)}(0)}{n!} x^n = \sum_{n=0}^{\infty} \dfrac{x^n}{n!} = 1 + \dfrac{x}{1!} + \dfrac{x^2}{2!} + \dfrac{x^3}{3!} + \cdots \]
Using the Ratio Test:
\[ \left| \dfrac{a_{n+1}}{a_n} \right| = \left| \dfrac{x^{n+1}}{(n+1)!} \cdot \dfrac{n!}{x^n} \right| = \dfrac{|x|}{n+1} \to 0 \text{ as } n \to \infty \]
The series converges for all \(x\), so the radius of convergence is \(R = \infty\).

\section*{When Is a Function Represented by Its Taylor Series?}
Let \(T_n(x)\) be the \(n\)-th degree Taylor polynomial of \(f\) at \(a\):
\[ T_n(x) = \sum_{i=0}^{n} \dfrac{f^{(i)}(a)}{i!} (x-a)^i \]
The remainder is defined as \(R_n(x) = f(x) - T_n(x)\). A function \(f\) is equal to the sum of its Taylor series if \(f(x) = \lim_{n\to\infty} T_n(x)\).

\begin{tcolorbox}[ colback=white,colframe=orange!80!white, title=Theorem 5,boxrule=0.5mm, arc=3mm]
    If \(f(x) = T_n(x) + R_n(x)\), where \(T_n\) is the \(n\)-th degree Taylor polynomial of \(f\) at \(a\), and if
    \[ \lim_{n\to\infty} R_n(x) = 0 \]
    for \(|x-a| < R\), then \(f\) is equal to the sum of its Taylor series on the interval \(|x-a| < R\).
\end{tcolorbox}

To show \(\lim_{n\to\infty} R_n(x) = 0\), we often use Taylor's Inequality.

\begin{tcolorbox}[
    colback=white,
    colframe=orange!80!white,
    title=Theorem 5,
    boxrule=0.5mm,
    arc=3mm
    ]
    If \(|f^{(n+1)}(x)| \le M\) for \(|x-a| \le d\), then the remainder \(R_n(x)\) of the Taylor series satisfies the inequality
    \[ |R_n(x)| \le \dfrac{M}{(n+1)!} |x-a|^{n+1} \quad \text{for } |x-a| \le d \]
\end{tcolorbox}

A useful limit for applying Taylor's Inequality:
\begin{equation*}
 \lim_{n\to\infty} \dfrac{x^n}{n!} = 0 \quad \text{for every real number } x \label{eq:limit_factorial}
\end{equation*}

\subsection*{EXAMPLE 3}
Prove that \(e^x\) is equal to the sum of its Maclaurin series.\\
\textbf{SOLUTION:}
Let \(f(x) = e^x\). Then \(f^{(n+1)}(x) = e^x\). For \(|x| \le d\), \(|f^{(n+1)}(x)| = e^x \le e^d\). Take \(M=e^d\). Taylor's Inequality gives:
\[ |R_n(x)| \le \dfrac{e^d}{(n+1)!} |x|^{n+1} \quad \text{for } |x| \le d \]
Using Equation (\ref{eq:limit_factorial}),
\[ \lim_{n\to\infty} \dfrac{e^d}{(n+1)!} |x|^{n+1} = e^d \lim_{n\to\infty} \dfrac{|x|^{n+1}}{(n+1)!} = 0 \]
By the Squeeze Theorem, \(\lim_{n\to\infty} |R_n(x)| = 0\), so \(\lim_{n\to\infty} R_n(x) = 0\) for all \(x\). By Theorem 8, \(e^x\) equals the sum of its Maclaurin series.
\begin{equation*}
 e^x = \sum_{n=0}^{\infty} \dfrac{x^n}{n!} \quad \text{for all } x \label{eq:ex_maclaurin}
\end{equation*}
Setting \(x=1\), we get
\begin{equation*}
 e = \sum_{n=0}^{\infty} \dfrac{1}{n!} = 1 + \dfrac{1}{1!} + \dfrac{1}{2!} + \dfrac{1}{3!} + \cdots \label{eq:e_series}
\end{equation*}

\subsection*{EXAMPLE 4}
Find the Taylor series for \(f(x) = e^x\) at \(a=2\).\\
\textbf{SOLUTION:}
We have \(f^{(n)}(2) = e^2\) for all \(n\). The Taylor series is:
\[ \sum_{n=0}^{\infty} \dfrac{f^{(n)}(2)}{n!} (x-2)^n = \sum_{n=0}^{\infty} \dfrac{e^2}{n!} (x-2)^n \]
It can be shown that \(\lim_{n\to\infty} R_n(x) = 0\), so
\begin{equation*}
 e^x = \sum_{n=0}^{\infty} \dfrac{e^2}{n!} (x-2)^n \quad \text{for all } x \label{eq:ex_taylor_a2}
\end{equation*}

\section*{Taylor Series of Important Functions}

\subsection*{EXAMPLE 5}
Find the Maclaurin series for \(\sin x\) and prove that it represents \(\sin x\) for all \(x\).\\
\textbf{SOLUTION:}
Let \(f(x) = \sin x\). The derivatives at 0 follow a pattern: \(0, 1, 0, -1, 0, 1, 0, -1, \ldots\).
The Maclaurin series is:
\[ x - \dfrac{x^3}{3!} + \dfrac{x^5}{5!} - \dfrac{x^7}{7!} + \cdots = \sum_{n=0}^{\infty} (-1)^n \dfrac{x^{2n+1}}{(2n+1)!} \]
Since \(|f^{(n+1)}(x)| = |\pm \sin x|\) or \(|\pm \cos x|\), we have \(|f^{(n+1)}(x)| \le 1\) for all \(x\). Take \(M=1\) in Taylor's Inequality:
\[ |R_n(x)| \le \dfrac{1}{(n+1)!} |x|^{n+1} \]
Using Equation (\ref{eq:limit_factorial}), \(\lim_{n\to\infty} \dfrac{|x|^{n+1}}{(n+1)!} = 0\). Thus \(\lim_{n\to\infty} R_n(x) = 0\) for all \(x\).
\begin{equation*}
 \sin x = x - \dfrac{x^3}{3!} + \dfrac{x^5}{5!} - \dfrac{x^7}{7!} + \cdots = \sum_{n=0}^{\infty} (-1)^n \dfrac{x^{2n+1}}{(2n+1)!} \quad \text{for all } x \label{eq:sinx_maclaurin}
\end{equation*}

\subsection*{EXAMPLE 6}
Find the Maclaurin series for \(\cos x\).\\
\textbf{SOLUTION:}
Differentiate the Maclaurin series for \(\sin x\) (Equation \ref{eq:sinx_maclaurin}) term by term:
\begin{align*}
 \cos x = \dfrac{d}{dx}(\sin x) &= \dfrac{d}{dx} \left( x - \dfrac{x^3}{3!} + \dfrac{x^5}{5!} - \dfrac{x^7}{7!} + \cdots \right) \\
 &= 1 - \dfrac{3x^2}{3!} + \dfrac{5x^4}{5!} - \dfrac{7x^6}{7!} + \cdots \\
 &= 1 - \dfrac{x^2}{2!} + \dfrac{x^4}{4!} - \dfrac{x^6}{6!} + \cdots = \sum_{n=0}^{\infty} (-1)^n \dfrac{x^{2n}}{(2n)!}
\end{align*}
The radius of convergence is \(R = \infty\).
\begin{equation*}
 \cos x = 1 - \dfrac{x^2}{2!} + \dfrac{x^4}{4!} - \dfrac{x^6}{6!} + \cdots = \sum_{n=0}^{\infty} (-1)^n \dfrac{x^{2n}}{(2n)!} \quad \text{for all } x \label{eq:cosx_maclaurin}
\end{equation*}

\subsection*{EXAMPLE 7}
Represent \(f(x) = \sin x\) as the sum of its Taylor series centered at \(\pi/3\).\\
\textbf{SOLUTION:}
Calculate derivatives at \(a=\pi/3\): \(f(\pi/3)=\sqrt{3}/2\), \(f'(\pi/3)=1/2\), \(f''(\pi/3)=-\sqrt{3}/2\), \(f'''(\pi/3)=-1/2\), etc. The Taylor series is:
\[ \dfrac{\sqrt{3}}{2} + \dfrac{1}{2 \cdot 1!}\left(x-\dfrac{\pi}{3}\right) - \dfrac{\sqrt{3}}{2 \cdot 2!}\left(x-\dfrac{\pi}{3}\right)^2 - \dfrac{1}{2 \cdot 3!}\left(x-\dfrac{\pi}{3}\right)^3 + \cdots \]
This can be written as:
\[ \sin x = \sum_{n=0}^{\infty} \dfrac{(-1)^n \sqrt{3}}{2(2n)!} \left(x-\dfrac{\pi}{3}\right)^{2n} + \sum_{n=0}^{\infty} \dfrac{(-1)^n}{2(2n+1)!} \left(x-\dfrac{\pi}{3}\right)^{2n+1} \]
This representation is valid for all \(x\).

\subsection*{EXAMPLE 8}
Find the Maclaurin series for \(f(x) = (1+x)^k\), where \(k\) is any real number.\\
\textbf{SOLUTION:}
Calculate derivatives: \(f(0)=1\), \(f'(0)=k\), \(f''(0)=k(k-1)\), \(f'''(0)=k(k-1)(k-2)\), ..., \(f^{(n)}(0)=k(k-1)\cdots(k-n+1)\).
The Maclaurin series is:
\[ \sum_{n=0}^{\infty} \dfrac{k(k-1)\cdots(k-n+1)}{n!} x^n \]
Using binomial coefficient notation \( \binom{k}{n} = \dfrac{k(k-1)\cdots(k-n+1)}{n!} \) for \(n \ge 1\) and \( \binom{k}{0}=1 \), the series is \( \sum_{n=0}^{\infty} \binom{k}{n} x^n \).
The Ratio Test gives:
\[ \left| \dfrac{a_{n+1}}{a_n} \right| = \left| \dfrac{\binom{k}{n+1} x^{n+1}}{\binom{k}{n} x^n} \right| = \left| \dfrac{k-n}{n+1} x \right| \to |x| \text{ as } n \to \infty \]
The series converges if \(|x|<1\) and diverges if \(|x|>1\). Radius of convergence is \(R=1\).

\begin{tcolorbox}[
    colback=white,
    colframe=orange!80!white,
    title=Theorem 5,
    boxrule=0.5mm,
    arc=3mm
    ]
    If \(k\) is any real number and \(|x| < 1\), then
    \[ (1+x)^k = \sum_{n=0}^{\infty} \binom{k}{n} x^n = 1 + kx + \dfrac{k(k-1)}{2!}x^2 + \dfrac{k(k-1)(k-2)}{3!}x^3 + \cdots \]
\end{tcolorbox}
Convergence at endpoints \(x=\pm 1\) depends on \(k\).

\subsection*{EXAMPLE 9}
Find the Maclaurin series for \(f(x) = \dfrac{1}{\sqrt{4-x}}\) and its radius of convergence.\\
\textbf{SOLUTION:}
Rewrite \(f(x)\):
\[ \dfrac{1}{\sqrt{4-x}} = \dfrac{1}{\sqrt{4(1-x/4)}} = \dfrac{1}{2\sqrt{1-x/4}} = \dfrac{1}{2} \left(1 + \left(-\dfrac{x}{4}\right)\right)^{-1/2} \]
Use the binomial series with \(k=-1/2\) and \(x\) replaced by \(-x/4\):
\begin{align*}
 \dfrac{1}{\sqrt{4-x}} &= \dfrac{1}{2} \sum_{n=0}^{\infty} \binom{-1/2}{n} \left(-\dfrac{x}{4}\right)^n \\
 &= \dfrac{1}{2} \left[ 1 + \left(-\dfrac{1}{2}\right)\left(-\dfrac{x}{4}\right) + \dfrac{(- \dfrac{1}{2})(- \dfrac{3}{2})}{2!}\left(-\dfrac{x}{4}\right)^2 + \cdots \right] \\
 &= \dfrac{1}{2} \left[ 1 + \dfrac{1}{8}x + \dfrac{1 \cdot 3}{2! 8^2} x^2 + \dfrac{1 \cdot 3 \cdot 5}{3! 8^3} x^3 + \cdots + \dfrac{1 \cdot 3 \cdot 5 \cdots (2n-1)}{n! 8^n} x^n + \cdots \right]
\end{align*}
The series converges for \(|-x/4| < 1\), which means \(|x|<4\). The radius of convergence is \(R=4\).

\begin{tcolorbox}[
    colback=white,
    colframe=orange!80!white,
    title=Theorem 5,
    boxrule=0.5mm,
    arc=3mm
    ]
\begin{align*}
    \dfrac{1}{1-x} &= \sum_{n=0}^{\infty} x^n = 1 + x + x^2 + x^3 + \cdots & R &= 1 \\[.5em] % Add some vertical space
    e^x &= \sum_{n=0}^{\infty} \dfrac{x^n}{n!} = 1 + \dfrac{x}{1!} + \dfrac{x^2}{2!} + \dfrac{x^3}{3!} + \cdots & R &= \infty \\[.5em]
    \sin x &= \sum_{n=0}^{\infty} (-1)^n \dfrac{x^{2n+1}}{(2n+1)!} = x - \dfrac{x^3}{3!} + \dfrac{x^5}{5!} - \dfrac{x^7}{7!} + \cdots & R &= \infty \\[.5em]
    \cos x &= \sum_{n=0}^{\infty} (-1)^n \dfrac{x^{2n}}{(2n)!} = 1 - \dfrac{x^2}{2!} + \dfrac{x^4}{4!} - \dfrac{x^6}{6!} + \cdots & R &= \infty \\[.5em]
    \tan^{-1} x &= \sum_{n=0}^{\infty} (-1)^n \dfrac{x^{2n+1}}{2n+1} = x - \dfrac{x^3}{3} + \dfrac{x^5}{5} - \dfrac{x^7}{7} + \cdots & R &= 1 \\[.5em]
    \ln(1+x) &= \sum_{n=1}^{\infty} (-1)^{n-1} \dfrac{x^n}{n} = x - \dfrac{x^2}{2} + \dfrac{x^3}{3} - \dfrac{x^4}{4} + \cdots & R &= 1 \\[.5em]
    (1+x)^k &= \sum_{n=0}^{\infty} \binom{k}{n} x^n = 1 + kx + \dfrac{k(k-1)}{2!}x^2 + \dfrac{k(k-1)(k-2)}{3!}x^3 + \cdots & R &= 1
\end{align*}
\end{tcolorbox}

\section*{New Taylor Series from Old}
We can find Taylor series for new functions by manipulating known series (from Table 1) via substitution, multiplication, division, addition, subtraction, differentiation, and integration.

\subsection*{EXAMPLE 10}
Find the Maclaurin series for (a) \(f(x) = x \cos x\) and (b) \(f(x) = \ln(1+3x^2)\).\\
\textbf{SOLUTION:}
(a) Multiply the series for \(\cos x\) by \(x\):
\[ x \cos x = x \sum_{n=0}^{\infty} (-1)^n \dfrac{x^{2n}}{(2n)!} = \sum_{n=0}^{\infty} (-1)^n \dfrac{x^{2n+1}}{(2n)!} \quad (R=\infty) \]
(b) Replace \(x\) by \(3x^2\) in the series for \(\ln(1+x)\):
\[ \ln(1+3x^2) = \sum_{n=1}^{\infty} (-1)^{n-1} \dfrac{(3x^2)^n}{n} = \sum_{n=1}^{\infty} (-1)^{n-1} \dfrac{3^n x^{2n}}{n} \]
Converges for \(|3x^2| < 1\), i.e., \(|x| < 1/\sqrt{3}\). \(R=1/\sqrt{3}\).

\subsection*{EXAMPLE 11}
Find the function represented by the power series \( \sum_{n=0}^{\infty} \dfrac{(-1)^n 2^n x^n}{n!} \).\\
\textbf{SOLUTION:}
Rewrite the series as \( \sum_{n=0}^{\infty} \dfrac{(-2x)^n}{n!} \). This is the Maclaurin series for \(e^u\) with \(u=-2x\). So the function is \(e^{-2x}\).

\subsection*{EXAMPLE 12}
Find the sum of the series \( \dfrac{1}{1 \cdot 2} - \dfrac{1}{2 \cdot 2^2} + \dfrac{1}{3 \cdot 2^3} - \dfrac{1}{4 \cdot 2^4} + \cdots \).\\
\textbf{SOLUTION:}
The series is \( \sum_{n=1}^{\infty} (-1)^{n-1} \dfrac{1}{n 2^n} = \sum_{n=1}^{\infty} (-1)^{n-1} \dfrac{(1/2)^n}{n} \).
This matches the series for \(\ln(1+x)\) with \(x=1/2\).
The sum is \(\ln(1 + 1/2) = \ln(3/2)\).

\subsection*{EXAMPLE 13}
(a) Evaluate \( \int e^{-x^2} \,dx \) as an infinite series. \\
(b) Evaluate \( \int_0^1 e^{-x^2} \,dx \) correct to within an error of 0.001.\\
\textbf{SOLUTION:}
(a) Replace \(x\) with \(-x^2\) in the series for \(e^x\):
\[ e^{-x^2} = \sum_{n=0}^{\infty} \dfrac{(-x^2)^n}{n!} = \sum_{n=0}^{\infty} (-1)^n \dfrac{x^{2n}}{n!} = 1 - \dfrac{x^2}{1!} + \dfrac{x^4}{2!} - \dfrac{x^6}{3!} + \cdots \]
Integrate term by term:
\[ \int e^{-x^2} \,dx = C + \sum_{n=0}^{\infty} (-1)^n \dfrac{x^{2n+1}}{(2n+1)n!} = C + x - \dfrac{x^3}{3 \cdot 1!} + \dfrac{x^5}{5 \cdot 2!} - \dfrac{x^7}{7 \cdot 3!} + \cdots \]
(b) Evaluate the definite integral using the series from part (a) with C=0:
\[ \int_0^1 e^{-x^2} \,dx = \left[ x - \dfrac{x^3}{3} + \dfrac{x^5}{10} - \dfrac{x^7}{42} + \dfrac{x^9}{216} - \cdots \right]_0^1 \]
\[ = 1 - \dfrac{1}{3} + \dfrac{1}{10} - \dfrac{1}{42} + \dfrac{1}{216} - \cdots \approx 1 - 0.33333 + 0.10000 - 0.02381 + 0.00463 - \cdots \]
This is an alternating series. The first few terms are \(1, -0.33333, 0.10000, -0.02381, 0.00463, -0.00059, \ldots\).
The term \(b_5 = 1/(11 \cdot 5!) = 1/1320 \approx 0.00076\).
The term \(b_6 = 1/(13 \cdot 6!) = 1/9360 \approx 0.00011\).
The error using \(s_5\) is less than \(b_6 \approx 0.00011 < 0.001\).
\[ \int_0^1 e^{-x^2} \,dx \approx 1 - \dfrac{1}{3} + \dfrac{1}{10} - \dfrac{1}{42} + \dfrac{1}{216} - \dfrac{1}{1320} \approx 0.7467 \]
(Using \(s_4\) gives error less than \(b_5 \approx 0.00076 < 0.001\).)
\[ \int_0^1 e^{-x^2} \,dx \approx 1 - \dfrac{1}{3} + \dfrac{1}{10} - \dfrac{1}{42} + \dfrac{1}{216} \approx 0.7475 \]

\subsection*{EXAMPLE 14}
Evaluate \( \lim_{x\to 0} \dfrac{e^x - 1 - x}{x^2} \).\\
\textbf{SOLUTION:}
Use the Maclaurin series for \(e^x\):
\begin{align*}
 \dfrac{e^x - 1 - x}{x^2} &= \dfrac{(1 + x +  \dfrac{x^2}{2!} +  \dfrac{x^3}{3!} + \cdots) - 1 - x}{x^2} \\
 &= \dfrac{ \dfrac{x^2}{2!} +  \dfrac{x^3}{3!} +  \dfrac{x^4}{4!} + \cdots}{x^2} \\
 &= \dfrac{1}{2!} + \dfrac{x}{3!} + \dfrac{x^2}{4!} + \cdots
\end{align*}
Therefore,
\[ \lim_{x\to 0} \dfrac{e^x - 1 - x}{x^2} = \lim_{x\to 0} \left( \dfrac{1}{2!} + \dfrac{x}{3!} + \dfrac{x^2}{4!} + \cdots \right) = \dfrac{1}{2!} = \dfrac{1}{2} \]

\section*{Multiplication and Division of Power Series}
Power series can be multiplied and divided like polynomials.

\subsection*{EXAMPLE 15}
Find the first three nonzero terms in the Maclaurin series for (a) \(e^x \sin x\) and (b) \(\tan x\).\\
\textbf{SOLUTION:}
(a) Multiply the series for \(e^x\) and \(\sin x\):
\[ e^x \sin x = \left( 1 + x + \dfrac{x^2}{2!} + \dfrac{x^3}{3!} + \cdots \right) \left( x - \dfrac{x^3}{3!} + \dfrac{x^5}{5!} - \cdots \right) \]
\[ = (1+x+ \dfrac{1}{2}x^2+ \dfrac{1}{6}x^3+\cdots)(x- \dfrac{1}{6}x^3+\cdots) \]
\[ = 1(x- \dfrac{1}{6}x^3+\cdots) + x(x- \dfrac{1}{6}x^3+\cdots) +  \dfrac{1}{2}x^2(x- \dfrac{1}{6}x^3+\cdots) +  \dfrac{1}{6}x^3(x-\cdots) + \cdots \]
\[ = x -  \dfrac{1}{6}x^3 + x^2 -  \dfrac{1}{6}x^4 +  \dfrac{1}{2}x^3 -  \dfrac{1}{12}x^5 +  \dfrac{1}{6}x^4 + \cdots \]
\[ = x + x^2 + \left(- \dfrac{1}{6} +  \dfrac{1}{2}\right)x^3 + \left(- \dfrac{1}{6} +  \dfrac{1}{6}\right)x^4 + \cdots \]
\[ = x + x^2 + \dfrac{1}{3}x^3 + \cdots \]
(b) Divide the series for \(\sin x\) by the series for \(\cos x\) using long division:
\[ \tan x = \dfrac{\sin x}{\cos x} = \dfrac{x -  \dfrac{x^3}{6} +  \dfrac{x^5}{120} - \cdots}{1 -  \dfrac{x^2}{2} +  \dfrac{x^4}{24} - \cdots} \]
Performing the long division:
\[\begin{array}{r} % <- 여기에 {r} 추가 (오른쪽 정렬)
    x +  \dfrac{1}{3}x^3 +  \dfrac{2}{15}x^5 + \cdots \\
    1 -  \dfrac{1}{2}x^2 +  \dfrac{1}{24}x^4 - \cdots \overline{) x -  \dfrac{1}{6}x^3 +  \dfrac{1}{120}x^5 - \cdots} \\ % 긴 나눗셈 기호 \overline 사용
    -(x -  \dfrac{1}{2}x^3 +  \dfrac{1}{24}x^5 - \cdots) \\
    \hline % 가로줄 \hline 사용
     \dfrac{1}{3}x^3 -  \dfrac{1}{30}x^5 + \cdots \\
    -( \dfrac{1}{3}x^3 -  \dfrac{1}{6}x^5 + \cdots) \\
    \hline % 가로줄 \hline 사용
     \dfrac{2}{15}x^5 + \cdots
\end{array}\]
    \[ \tan x = x + \dfrac{1}{3}x^3 + \dfrac{2}{15}x^5 + \cdots \]

\end{document}