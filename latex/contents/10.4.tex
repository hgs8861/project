\documentclass{article}
\usepackage[a4paper,left=3cm, right=3cm, top=2cm, bottom=2cm]{geometry}
\usepackage{amsmath}
\usepackage{graphicx}
\usepackage{caption}
\usepackage{setspace}
\usepackage{xcolor}
\usepackage{titlesec}
\usepackage{amssymb}
\graphicspath{{graph/}}
\title{10.4 Calculus in Polar Coordinates}
\date{}
\author{}
\setstretch{1.2} 

% \subsection* 형식 지정 (번호 없음)
\titleformat{name=\section, numberless}
  {\normalfont\large\bfseries\color{blue}}
  {}
  {0pt}
  {}
\geometry{a4paper, margin=1in}

\begin{document}
\maketitle
This section applies calculus methods to polar curves, focusing on areas, arc lengths, and tangent slopes.

\section*{Area}
To develop the formula for the area A of a region whose boundary is given by a polar equation, we start with the formula for the area of a sector of a circle: $A = \frac{1}{2} r^2\theta$.

Consider a region bounded by a polar curve $r = f(\theta)$ and by rays $\theta = a$ and $\theta = b$. We divide the interval $[a,b]$ into n subintervals of equal width $\Delta\theta$. These rays divide the region into n smaller regions. The area $\Delta A_i$ of the i-th region is approximated by the area of a sector of a circle with radius $f(\theta_i^*)$ and central angle $\Delta\theta$:
\[
\Delta A_i \approx \frac{1}{2} [f(\theta_i^*)]^2 \Delta\theta
\]
Summing these approximations gives a Riemann sum. As $n \to \infty$, this sum approaches the definite integral.

\paragraph{Formula for Area in Polar Coordinates:}
\[
A = \frac{1}{2} \int_{a}^{b} [f(\theta)]^2 \, d\theta \quad \text{or} \quad A = \frac{1}{2} \int_{a}^{b} r^2 \, d\theta
\]

\subsubsection*{EXAMPLE 1}
Find the area enclosed by one loop of the four-leaved rose $r = \cos(2\theta)$.

\paragraph{Solution:} One loop of $r = \cos(2\theta)$ is traced from $\theta = -\pi/4$ to $\theta = \pi/4$. Using symmetry and the area formula:
\[
A = 2 \cdot \frac{1}{2} \int_{0}^{\pi/4} \cos^2(2\theta) \, d\theta
\]
Applying the half-angle identity $\cos^2(u) = \frac{1 + \cos(2u)}{2}$:
\[
A = \int_{0}^{\pi/4} \frac{1 + \cos(4\theta)}{2} \, d\theta = \frac{1}{2} \left[ \theta + \frac{1}{4}\sin(4\theta) \right]_{0}^{\pi/4} = \frac{1}{2} \left( \frac{\pi}{4} \right) = \frac{\pi}{8}
\]

\subsubsection*{EXAMPLE 2}
Find the area of the region inside the circle $r = 3\sin\theta$ and outside the cardioid $r = 1 + \sin\theta$.

\paragraph{Solution:} Intersection points are found by $3\sin\theta = 1 + \sin\theta \implies \sin\theta = 1/2$, yielding $\theta = \pi/6$ and $\theta = 5\pi/6$. The area is the difference of the two areas between these angles:
\begin{align*}
    A &= \frac{1}{2} \int_{\pi/6}^{5\pi/6} \left[ (3\sin\theta)^2 - (1 + \sin\theta)^2 \right] \, d\theta \\
    &= \frac{1}{2} \int_{\pi/6}^{5\pi/6} (9\sin^2\theta - (1 + 2\sin\theta + \sin^2\theta)) \, d\theta \\
    &= \frac{1}{2} \int_{\pi/6}^{5\pi/6} (8\sin^2\theta - 2\sin\theta - 1) \, d\theta
\end{align*}
Using $\sin^2\theta = \frac{1 - \cos(2\theta)}{2}$:
\begin{align*}
    A &= \frac{1}{2} \int_{\pi/6}^{5\pi/6} \left[ 8\left(\frac{1 - \cos(2\theta)}{2}\right) - 2\sin\theta - 1 \right] \, d\theta \\
    &= \frac{1}{2} \int_{\pi/6}^{5\pi/6} (3 - 4\cos(2\theta) - 2\sin\theta) \, d\theta \\
    &= \frac{1}{2} \left[ 3\theta - 2\sin(2\theta) + 2\cos\theta \right]_{\pi/6}^{5\pi/6} = \pi
\end{align*}

\paragraph{CAUTION:} Graphing polar curves is crucial for finding all intersection points. It is especially convenient to use a graphing calculator or computer to help with this task.

\subsubsection*{EXAMPLE 3}
Find all points of intersection of $r = \cos(2\theta)$ and $r = 1/2$.

\paragraph{Solution:} Equating the equations, $\cos(2\theta) = 1/2$, yields $2\theta = \pi/3, 5\pi/3, 7\pi/3, 11\pi/3, \dots$. Thus, for $0 \le \theta < 2\pi$, intersection points are $(1/2, \pi/6)$, $(1/2, 5\pi/6)$, $(1/2, 7\pi/6)$, and $(1/2, 11\pi/6)$.

\section*{Arc Length}
To find the length L of a polar curve $r = f(\theta)$ from $\theta = a$ to $\theta = b$, we use its parametric form $x = r\cos\theta$ and $y = r\sin\theta$. Differentiating with respect to $\theta$:
\begin{align*}
    \frac{dx}{d\theta} &= \frac{dr}{d\theta}\cos\theta - r\sin\theta \\
    \frac{dy}{d\theta} &= \frac{dr}{d\theta}\sin\theta + r\cos\theta
\end{align*}
Squaring and summing these gives $\left(\frac{dx}{d\theta}\right)^2 + \left(\frac{dy}{d\theta}\right)^2 = \left(\frac{dr}{d\theta}\right)^2 + r^2$.

\paragraph{Formula for Arc Length in Polar Coordinates:}
\[
L = \int_{a}^{b} \sqrt{r^2 + \left(\frac{dr}{d\theta}\right)^2} \, d\theta
\]

\subsubsection*{EXAMPLE 4}
Find the length of the cardioid $r = 1 + \sin\theta$.

\paragraph{Solution:} The cardioid is traced once for $0 \le \theta \le 2\pi$. With $r = 1 + \sin\theta$, we have $\frac{dr}{d\theta} = \cos\theta$.
\begin{align*}
    L &= \int_{0}^{2\pi} \sqrt{(1 + \sin\theta)^2 + (\cos\theta)^2} \, d\theta \\
    &= \int_{0}^{2\pi} \sqrt{1 + 2\sin\theta + \sin^2\theta + \cos^2\theta} \, d\theta \\
    &= \int_{0}^{2\pi} \sqrt{2 + 2\sin\theta} \, d\theta
\end{align*}
Using the identity $1+\sin\theta = 2\cos^2(\frac{\pi}{4}-\frac{\theta}{2})$ and careful evaluation of the absolute value, the length is 8.

\subsection*{Tangents}
To find the slope of the tangent line $dy/dx$ to a polar curve $r = f(\theta)$, we use its parametric form.

\paragraph{Formula for Slope of Tangent in Polar Coordinates:}
\[
\frac{dy}{dx} = \frac{\frac{dy}{d\theta}}{\frac{dx}{d\theta}} = \frac{\frac{dr}{d\theta}\sin\theta + r\cos\theta}{\frac{dr}{d\theta}\cos\theta - r\sin\theta}, \quad \text{provided } \frac{dx}{d\theta} \neq 0
\]
Horizontal tangents occur when $\frac{dy}{d\theta} = 0$ (and $\frac{dx}{d\theta} \neq 0$). Vertical tangents occur when $\frac{dx}{d\theta} = 0$ (and $\frac{dy}{d\theta} \neq 0$).

\paragraph{Tangents at the Pole:} If $r = 0$ at $\theta = \theta_0$ and $\frac{dr}{d\theta} \neq 0$, the tangent line at the pole is $\theta = \theta_0$.

\subsubsection*{EXAMPLE 5}
For the cardioid $r = 1 + \sin\theta$: (a) Find the tangent slope at $\theta = \pi/3$. (b) Find horizontal and vertical tangent points.

\paragraph{Solution:} Given $r = 1 + \sin\theta$, we have $\frac{dr}{d\theta} = \cos\theta$.
(a) At $\theta = \pi/3$, substituting values into the slope formula yields $\frac{dy}{dx} = -1$.

(b) Horizontal tangents ($\frac{dy}{d\theta} = 0$): $\cos\theta(1 + 2\sin\theta) = 0$. This gives $\theta = \pi/2, 3\pi/2, 7\pi/6, 11\pi/6$. The points are $(2, \pi/2)$, $(1/2, 7\pi/6)$, and $(1/2, 11\pi/6)$.

Vertical tangents ($\frac{dx}{d\theta} = 0$): $1 - \sin\theta - 2\sin^2\theta = 0 \implies (1+\sin\theta)(1-2\sin\theta)=0$. This gives $\theta = 3\pi/2, \pi/6, 5\pi/6$. The points are $(3/2, \pi/6)$ and $(3/2, 5\pi/6)$. At $\theta=3\pi/2$, both derivatives are zero, which corresponds to the cusp at the pole.

\end{document}
