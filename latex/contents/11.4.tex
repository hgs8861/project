\documentclass{article}
\usepackage[a4paper,left=3cm, right=3cm, top=2cm, bottom=2cm]{geometry}
\usepackage{amsmath}
\usepackage{graphicx}
\usepackage{caption}
\usepackage{setspace}
\usepackage{xcolor}
\usepackage{titlesec}
\usepackage{amssymb}
\usepackage{tcolorbox}
\usepackage{wrapfig}
\usepackage{amsthm} % For proofs

\graphicspath{{graph/}}
\title{11.4 The Comparison Tests}
\date{}
\author{}
\setstretch{1.3}

% \subsection* 형식 지정 (번호 없음)
\titleformat{name=\section, numberless}
  {\normalfont\large\bfseries\color{blue}}
  {}
  {0pt}
  {}
\geometry{a4paper, margin=1in}

% 증명 환경 스타일
\newtheoremstyle{mystyle}% name
  {}% Space above
  {}% Space below
  {\itshape}% Body font
  {}% Indent amount
  {\bfseries}% Theorem head font
  {.}% Punctuation after theorem head
  {.5em}% Space after theorem head
  {}% Theorem head spec (can be left empty, meaning `normal')
\theoremstyle{mystyle}

\begin{document}
\maketitle

In the comparison tests, the idea is to compare a given series with a series that is known to be convergent or divergent.

\section*{The Direct Comparison Test}

\begin{tcolorbox}[
    colback=white,
    colframe=orange!80!white,
    title=The Direct Comparison Test,
    boxrule=0.5mm,
    arc=3mm
    ]
    Suppose that \( \sum a_n \) and \( \sum b_n \) are series with positive terms.
    \begin{itemize}
        \item[(i)] If \( \sum b_n \) is \textbf{convergent} and \( a_n \le b_n \) for all \( n \), then \( \sum a_n \) is also \textbf{convergent}.
        \item[(ii)] If \( \sum b_n \) is \textbf{divergent} and \( a_n \ge b_n \) for all \( n \), then \( \sum a_n \) is also \textbf{divergent}.
    \end{itemize}
    \textbf{Standard series for use with the comparison tests:}
    \begin{itemize}
        \item A \textbf{p-series} \( \sum 1/n^p \) converges if \( p > 1 \) and diverges if \( p \le 1 \).
        \item A \textbf{geometric series} \( \sum ar^{n-1} \) converges if \( |r| < 1 \) and diverges if \( |r| \ge 1 \).
    \end{itemize}
\end{tcolorbox}

\subsection*{EXAMPLE 1}
Determine whether the series \( \sum_{n=1}^{\infty} \frac{5}{2n^2 + 4n + 3} \) converges or diverges.

\textbf{SOLUTION:}
For large $n$ the dominant term in the denominator is $2n^2$, so we compare the given series with the series $\sum 5/(2n^2)$. Observe that
\[ \frac{5}{2n^2 + 4n + 3} < \frac{5}{2n^2} \]
because the left side has a bigger denominator. We know that
\[ \sum_{n=1}^{\infty} \frac{5}{2n^2} = \frac{5}{2} \sum_{n=1}^{\infty} \frac{1}{n^2} \]
is convergent because it's a constant times a p-series with $p = 2 > 1$. Therefore the given series is \textbf{convergent} by the Direct Comparison Test.

\subsection*{EXAMPLE 2}
Test the series \( \sum_{k=1}^{\infty} \frac{\ln k}{k} \) for convergence or divergence.

\textbf{SOLUTION:}
We compare it with the harmonic series. Observe that $\ln k > 1$ for $k \ge 3$ and so
\[ \frac{\ln k}{k} > \frac{1}{k} \quad \text{for } k \ge 3 \]
We know that $\sum 1/k$ is divergent (p-series with $p=1$). Thus the given series is \textbf{divergent} by the Direct Comparison Test.

\section*{The Limit Comparison Test}
\begin{tcolorbox}[
    colback=white,
    colframe=orange!80!white,
    title=The Limit Comparison Test,
    boxrule=0.5mm,
    arc=3mm
    ]
    Suppose that \( \sum a_n \) and \( \sum b_n \) are series with positive terms. If
    \[ \lim_{n \to \infty} \frac{a_n}{b_n} = c \]
    where \( c \) is a finite number and \( c > 0 \), then either both series converge or both diverge.
\end{tcolorbox}

\subsection*{EXAMPLE 3}
Test the series \( \sum_{n=1}^{\infty} \frac{1}{2^n - 1} \) for convergence or divergence.

\textbf{SOLUTION:}
We use the Limit Comparison Test with $a_n = \frac{1}{2^n - 1}$ and $b_n = \frac{1}{2^n}$.
\[ \lim_{n \to \infty} \frac{a_n}{b_n} = \lim_{n \to \infty} \frac{1/(2^n - 1)}{1/2^n} = \lim_{n \to \infty} \frac{2^n}{2^n - 1} = \lim_{n \to \infty} \frac{1}{1 - 1/2^n} = 1 > 0 \]
Since this limit exists and $\sum 1/2^n$ is a convergent geometric series, the given series \textbf{converges} by the Limit Comparison Test.

\subsection*{EXAMPLE 4}
Determine whether the series \( \sum_{n=1}^{\infty} \frac{2n^2 + 3n}{\sqrt{5 + n^5}} \) converges or diverges.

\textbf{SOLUTION:}
The dominant part of the numerator is $2n^2$ and the dominant part of the denominator is $\sqrt{n^5} = n^{5/2}$. This suggests taking
\[ a_n = \frac{2n^2 + 3n}{\sqrt{5 + n^5}} \quad b_n = \frac{2n^2}{n^{5/2}} = \frac{2}{n^{1/2}} \]
\[ \lim_{n \to \infty} \frac{a_n}{b_n} = \lim_{n \to \infty} \frac{2n^2 + 3n}{\sqrt{5 + n^5}} \cdot \frac{n^{1/2}}{2} = \lim_{n \to \infty} \frac{2n^{5/2} + 3n^{3/2}}{2\sqrt{5 + n^5}} = \lim_{n \to \infty} \frac{2 + 3/n}{2\sqrt{5/n^5 + 1}} = \frac{2+0}{2\sqrt{0+1}} = 1 \]
Since $\sum b_n = 2\sum 1/n^{1/2}$ is divergent (p-series with $p = 1/2 < 1$), the given series \textbf{diverges} by the Limit Comparison Test.

\section*{Estimating Sums}
If we have used the Direct Comparison Test to show that a series \( \sum a_n \) converges by comparison with a series \( \sum b_n \), then we may be able to estimate the sum \( \sum a_n \) by comparing remainders. Let \(R_n = s - s_n\) and \(T_n = t - t_n\). Since \(a_n \le b_n\) for all \(n\), we have \(R_n \le T_n\).

\subsection*{EXAMPLE 5}
Use the sum of the first 100 terms to approximate the sum of the series \( \sum \frac{1}{n^3 + 1} \). Estimate the error involved in this approximation.

\textbf{SOLUTION:}
Since $\frac{1}{n^3 + 1} < \frac{1}{n^3}$, the given series is convergent by the Direct Comparison Test. The remainder $T_n$ for the comparison series $\sum 1/n^3$ was estimated using the Remainder Estimate for the Integral Test. We found that
\[ T_n \le \int_n^\infty \frac{1}{x^3} dx = \frac{1}{2n^2} \]
Therefore the remainder $R_n$ for the given series satisfies
\[ R_n \le T_n \le \frac{1}{2n^2} \]
With $n=100$ we have
\[ R_{100} \le \frac{1}{2(100)^2} = 0.00005 \]
Using a calculator, we find that
\[ \sum_{n=1}^{\infty} \frac{1}{n^3 + 1} \approx \sum_{n=1}^{100} \frac{1}{n^3 + 1} \approx 0.6864538 \]
with error less than 0.00005.

\end{document}
