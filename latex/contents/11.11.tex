\documentclass{article}
\usepackage[a4paper,left=3cm, right=3cm, top=2cm, bottom=2cm]{geometry}
\usepackage{amsmath}
\usepackage{graphicx}
\usepackage{caption}
\usepackage{setspace}
\usepackage{xcolor}
\usepackage{titlesec}
\usepackage{amssymb}
\usepackage{tcolorbox}
\usepackage{wrapfig}
\usepackage{amsthm} % For proofs
\usepackage{amsfonts} % For \mathbb{R}

\graphicspath{{graph/}}
\title{11.11 Applications of Taylor Polynomials}
\date{}
\author{}
\setstretch{1.3}

% \subsection* format (no number)
\titleformat{name=\section, numberless}
  {\normalfont\large\bfseries\color{blue}}
  {}
  {0pt}
  {}
\geometry{a4paper, margin=1in}

% Theorem environment style
\newtheoremstyle{mystyle}% name
  {}% Space above
  {}% Space below
  {\itshape}% Body font
  {}% Indent amount
  {\bfseries}% Theorem head font
  {.}% Punctuation after theorem head
  {.5em}% Space after theorem head
  {}% Theorem head spec (can be left empty, meaning `normal')
\theoremstyle{mystyle}
\newtheorem*{theorem}{Theorem} % Use unnumbered theorem environment

\begin{document}
\maketitle

\section*{Approximating Functions by Polynomials}
Suppose that \(f(x)\) is equal to the sum of its Taylor series at \(a\):
\[ f(x) = \sum_{n=0}^{\infty} \dfrac{f^{(n)}(a)}{n!} (x-a)^n \]
In Section 11.10 we introduced the notation \(T_n(x)\) for the \(n\)-th partial sum of this series, called the \textbf{\(n\)-th degree Taylor polynomial} of \(f\) at \(a\).
\[ T_n(x) = \sum_{i=0}^{n} \dfrac{f^{(i)}(a)}{i!} (x-a)^i = f(a) + \dfrac{f'(a)}{1!}(x-a) + \dfrac{f''(a)}{2!}(x-a)^2 + \cdots + \dfrac{f^{(n)}(a)}{n!}(x-a)^n \]
Since \(T_n(x) \to f(x)\) as \(n \to \infty\), \(T_n\) can be used as an approximation to \(f\): \(f(x) \approx T_n(x)\).

Notice that the first-degree Taylor polynomial
\[ T_1(x) = f(a) + f'(a)(x-a) \]
is the same as the linearization of \(f\) at \(a\) (from Section 2.9). The derivatives of \(T_n\) at \(a\) agree with those of \(f\) up to order \(n\).\\
To determine the accuracy of the approximation \(f(x) \approx T_n(x)\), we must estimate the absolute value of the remainder:
\[ |R_n(x)| = |f(x) - T_n(x)| \]
There are three possible methods for estimating the error:
\begin{itemize}
    \item[1.] Use a calculator or computer to graph \(|R_n(x)|\).
    \item[2.] If the series is an alternating series, use the Alternating Series Estimation Theorem.
    \item[3.] In all cases, use Taylor's Inequality.
\end{itemize}

\begin{tcolorbox}[
    colback=white,
    colframe=orange!80!white,
    title=Theorem 11.10.9: Taylor's Inequality,
    boxrule=0.5mm,
    arc=3mm
    ]
    If \(|f^{(n+1)}(x)| \le M\) for \(|x-a| \le d\), then the remainder \(R_n(x)\) of the Taylor series satisfies the inequality
    \[ |R_n(x)| \le \dfrac{M}{(n+1)!} |x-a|^{n+1} \quad \text{for } |x-a| \le d \]
\end{tcolorbox}

\subsection*{EXAMPLE 1}
(a) Approximate the function \(f(x) = \sqrt[3]{x}\) by a Taylor polynomial of degree 2 at \(a=8\). \\
(b) How accurate is this approximation when \(7 \le x \le 9\)?\\
\textbf{SOLUTION:}
(a) First, we compute the derivatives and evaluate them at \(a=8\):
\begin{align*}
    f(x) &= x^{1/3} & f(8) &= 2 \\
    f'(x) &= \frac{1}{3}x^{-2/3} & f'(8) &= \frac{1}{3 \cdot 8^{2/3}} = \frac{1}{12} \\
    f''(x) &= -\frac{2}{9}x^{-5/3} & f''(8) &= -\frac{2}{9 \cdot 8^{5/3}} = -\frac{2}{9 \cdot 32} = -\frac{1}{144} 
\end{align*}
The second-degree Taylor polynomial is:
\begin{align*}
    T_2(x) &= f(a) + \dfrac{f'(a)}{1!}(x-a) + \dfrac{f''(a)}{2!}(x-a)^2 \\
    &= 2 + \dfrac{1}{12}(x-8) + \dfrac{-1/144}{2}(x-8)^2 \\
    &= 2 + \dfrac{1}{12}(x-8) - \dfrac{1}{288}(x-8)^2
\end{align*}
The approximation is \(\sqrt[3]{x} \approx T_2(x) = 2 + \dfrac{1}{12}(x-8) - \dfrac{1}{288}(x-8)^2\).

(b) We use Taylor's Inequality with \(n=2\) and \(a=8\). The remainder \(R_2(x)\) satisfies:
\[ |R_2(x)| \le \dfrac{M}{3!} |x-8|^3 \]
where \(|f'''(x)| \le M\). For \(x\) in the interval \([7, 9]\), we must find the maximum value of \(|f'''(x)|\).
\[ |f'''(x)| = \left| \dfrac{10}{27}x^{-8/3} \right| = \dfrac{10}{27x^{8/3}} \]
This function is decreasing for \(x > 0\), so its maximum value on \([7, 9]\) occurs at \(x=7\).
\[ |f'''(x)| \le \dfrac{10}{27 \cdot 7^{8/3}} < 0.0021 \]
We can take \(M=0.0021\). Also, for \(7 \le x \le 9\), we have \(-1 \le x-8 \le 1\), so \(|x-8| \le 1\).
Using Taylor's Inequality:
\[ |R_2(x)| \le \dfrac{0.0021}{3!} (1)^3 = \dfrac{0.0021}{6} < 0.0004 \]
Thus, if \(7 \le x \le 9\), the approximation is accurate to within 0.0004.
\begin{figure}[htbp]
    \centering
    \includegraphics[width=0.33\textwidth]{graph77.png}
    \includegraphics[width=0.33\textwidth]{graph78.png}
\end{figure}


\subsection*{EXAMPLE 2}
(a) What is the maximum error possible in using the approximation
\[ \sin x \approx x - \dfrac{x^3}{3!} + \dfrac{x^5}{5!} \]
when \(-0.3 \le x \le 0.3\)? Use this approximation to find \(\sin 12^\circ\) correct to six decimal places. \\
(b) For what values of \(x\) is this approximation accurate to within 0.00005?\\
\textbf{SOLUTION:}
(a) The error in approximating \(\sin x\) by the first three terms (\(T_5(x)\)) is at most the absolute value of the first neglected term:
\[ |R_5(x)| \le \left| -\dfrac{x^7}{7!} \right| = \dfrac{|x|^7}{5040} \]
If \(-0.3 \le x \le 0.3\), then \(|x| \le 0.3\), so the error is smaller than:
\[ \dfrac{(0.3)^7}{5040} < 4.3 \times 10^{-8} \]
To find \(\sin 12^\circ\), we first convert to radians:
\[ 12^\circ = 12 \left( \dfrac{\pi}{180} \right) = \dfrac{\pi}{15} \text{ rad} \]
Using the approximation:
\[ \sin\left(\dfrac{\pi}{15}\right) \approx \left(\dfrac{\pi}{15}\right) - \dfrac{(\pi/15)^3}{3!} + \dfrac{(\pi/15)^5}{5!} \approx 0.20791169 \]
The error is less than \(4.3 \times 10^{-8}\), so, correct to six decimal places, \(\sin 12^\circ \approx 0.207912\).

(b) We want the error to be smaller than 0.00005:
\[ \dfrac{|x|^7}{5040} < 0.00005 \]
Solving this inequality for \(|x|\):
\[ |x|^7 < 5040(0.00005) = 0.252 \]
\[ |x| < (0.252)^{1/7} \approx 0.821 \]
So the approximation is accurate to within 0.00005 when \(|x| < 0.82\).

(Using Taylor's Inequality: The approximation is \(T_6(x) = T_5(x)\). The next derivative is \(f^{(7)}(x) = -\cos x\). Thus \(|f^{(7)}(x)| \le 1\) for all \(x\). We take \(M=1\).
\[ |R_6(x)| \le \dfrac{M}{(6+1)!} |x|^{6+1} = \dfrac{|x|^7}{7!} = \dfrac{|x|^7}{5040} \]
This gives the same estimate. Graphical methods (Figures 4 and 5 in the text) also confirm this result.)

\begin{figure}[htbp]

    \centering
    \includegraphics[width=0.33\textwidth]{graph79.png}
    \includegraphics[width=0.33\textwidth]{graph80.png}
\end{figure}

\end{document}