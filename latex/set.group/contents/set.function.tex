\documentclass{article}
\usepackage[a4paper,left=3cm, right=3cm, top=2cm, bottom=2cm]{geometry}
\usepackage{amsmath}
\usepackage{graphicx}
\usepackage{caption}
\usepackage{setspace}
\usepackage{xcolor}
\usepackage{titlesec}
\usepackage{amssymb}
\usepackage{tcolorbox}
\usepackage{wrapfig}
\usepackage{amsthm} % For proofs
\usepackage{amsfonts} % For \mathbb{R}

\graphicspath{{graph/}}
\title{Set theory - Functions}
\date{}
\author{}
\setstretch{1.3}

% \subsection* format (no number)
\titleformat{name=\section, numberless}
  {\normalfont\large\bfseries\color{blue}}
  {}
  {0pt}
  {}
\geometry{a4paper, margin=1in}

% Proof environment style
\newtheoremstyle{mystyle}% name
  {}% Space above
  {}% Space below
  {\itshape}% Body font
  {}% Indent amount
  {\bfseries}% Theorem head font
  {.}% Punctuation after theorem head
  {.5em}% Space after theorem head
  {}% Theorem head spec (can be left empty, meaning `normal')
\theoremstyle{mystyle}
\newtheorem*{theorem}{Theorem} % Use unnumbered theorem environment

\begin{document}
\maketitle

\section*{Definitions of Taylor Series and Maclaurin Series}

\begin{figure}[htbp]
    \centering
    \includegraphics[width=0.5\textwidth]{graph 78.png}
\end{figure}


\begin{tcolorbox}[
    colback=white,
    colframe=orange!80!white,
    title=Theorem 5,
    boxrule=0.5mm,
    arc=3mm
    ]
    If \(f\) has a power series representation (expansion) at \(a\), that is, if
    \[ f(x) = \sum_{n=0}^{\infty} c_n (x-a)^n \quad |x-a| < R \]
    then its coefficients are given by the formula
    \[ c_n = \dfrac{f^{(n)}(a)}{n!} \]
\end{tcolorbox}

\end{document}