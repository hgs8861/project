\documentclass[a4paper,11pt]{article}
\usepackage{kotex}
\usepackage{amsmath, amssymb, amsthm}
\usepackage{geometry}
\usepackage{tcolorbox}
\usepackage{enumitem}
\tcbuselibrary{skins, breakable}

% --- 여백 및 간격 설정 ---
\geometry{left=2.5cm, right=2.5cm, top=2.5cm, bottom=2.5cm}
\linespread{1.4}
\setlength{\parskip}{1em}
\setlist[enumerate]{itemsep=5pt, topsep=0pt}

% --- 사용자 정의 박스 스타일 ---
\newtcolorbox{conceptbox}[1][]{
  colback=blue!5, colframe=blue!60, coltitle=white, fonttitle=\bfseries,
  title={#1}, arc=1mm, boxrule=0.5mm, breakable, parskip=1em
}
\newtcolorbox{advicebox}{
  colback=orange!5, colframe=orange!60, title={Advice}, fonttitle=\bfseries,
  coltitle=white, arc=1mm, boxrule=0.5mm, breakable, parskip=1em
}
\newtcolorbox{studybox}{
  colback=gray!10, colframe=gray!60, title={정석연구}, fonttitle=\bfseries,
  coltitle=white, arc=1mm, boxrule=0.5mm, breakable, parskip=1em
}

\title{\textbf{미적분 - 제17장 넓이와 적분}}
\author{수학의 정석}
\date{}

\begin{document}

\maketitle

\section*{§1. 곡선과 좌표축 사이의 넓이}

\begin{conceptbox}[기본 정석]
\textbf{① 곡선과 $x$ 축 사이의 넓이}
\begin{enumerate}[label=(\text{\roman*})]
    \item 구간 $[a, b]$ 에서 $f(x) \ge 0$ 인 경우 $S = \int_{a}^{b} f(x) dx$
    \item 구간 $[a, b]$ 에서 $f(x) \le 0$ 인 경우 $S = -\int_{a}^{b} f(x) dx$
    \item 일반적인 경우 $S = \int_{a}^{b} |f(x)| dx$
\end{enumerate}
\textbf{② 곡선과 $y$ 축 사이의 넓이}
\begin{enumerate}[label=(\text{\roman*})]
    \item 구간 $[\alpha, \beta]$ 에서 $g(y) \ge 0$ 인 경우 $S = \int_{\alpha}^{\beta} g(y) dy$
    \item 구간 $[\alpha, \beta]$ 에서 $g(y) \le 0$ 인 경우 $S = -\int_{\alpha}^{\beta} g(y) dy$
    \item 일반적인 경우 $S = \int_{\alpha}^{\beta} |g(y)| dy$
\end{enumerate}
\end{conceptbox}

\begin{advicebox}
\textbf{1° 곡선과 $x$ 축 사이의 넓이}\\
이미 수학 II에서 정적분과 넓이 사이의 관계를 공부했지만 이에 대하여 다시 정리해 보자. 함수 $f(t)$ 가 구간 $[a, b]$ 에서 연속이고 $f(t) \ge 0$ 일 때, $a \le x \le b$ 인 $x$ 에 대하여 곡선 $y=f(t)$ 와 $t$ 축 및 두 직선 $t=a, t=x$ 로 둘러싸인 도형의 넓이를 $S(x)$ 라고 하면 $S'(x)=f(x)$ 이므로 $S(x)$ 는 $f(x)$ 의 부정적분 중 하나이다. 이때 $f(x)$ 의 다른 한 부정적분을 $F(x)$ 라고 하면 $S(x)=F(x)+C$ (단, $C$ 는 상수) 이고, $S(a)=0$ 에서 $C=-F(a)$ 이므로 $S(b)=F(b)-F(a)=\int_{a}^{b}f(t)dt$ 이다. 구간 $[a, b]$ 에서 $f(x) \le 0$ 일 때에는 곡선 $y=f(x)$ 가 곡선 $y=-f(x)$ 와 $x$ 축에 대하여 대칭이고 $-f(x) \ge 0$ 이므로 곡선 $y=f(x)$ 와 $x$ 축 및 두 직선 $x=a, x=b$ 로 둘러싸인 도형의 넓이를 $S$ 라고 하면 $S = \int_a^b \{-f(x)\} dx = -\int_a^b f(x) dx$ 이다.

\textbf{2° 곡선과 $y$ 축 사이의 넓이}\\
곡선과 $x$ 축 사이의 넓이를 구할 때와 같은 방법으로 생각하면 된다. 구간 $[\alpha, \beta]$ 에서 $g(y) \ge 0$ 일 때, 곧 곡선 $x=g(y)$ 가 $y$ 축의 오른쪽에 있을 때 넓이는 $S = \int_{\alpha}^{\beta} g(y) dy$ 이다. 한편 구간 $[\alpha, \beta]$ 에서 $g(y) \le 0$ 일 때, 곧 곡선 $x=g(y)$ 가 $y$ 축의 왼쪽에 있을 때에는 $S = -\int_{\alpha}^{\beta} g(y) dy$ 라고 해야 한다.
\end{advicebox}

\section*{§2. 두 곡선 사이의 넓이}

\begin{conceptbox}[기본 정석]
\textbf{두 곡선 사이의 넓이}
\begin{enumerate}[label=(\arabic*)]
    \item 구간 $[a, b]$ 에서 $f(x) \ge g(x)$ 일 때, $y=f(x)$ 와 $y=g(x)$ 의 그래프로 둘러싸인 도형의 넓이 $S$ 는 $S = \int_a^b \{f(x) - g(x)\} dx$
    \item 구간 $[\alpha, \beta]$ 에서 $f(y) \ge g(y)$ 일 때, $x=f(y)$ 와 $x=g(y)$ 의 그래프로 둘러싸인 도형의 넓이 $S$ 는 $S = \int_{\alpha}^{\beta} \{f(y) - g(y)\} dy$
\end{enumerate}
\end{conceptbox}

\begin{advicebox}
\textbf{두 곡선 사이의 넓이}\\
구간 $[a, b]$ 에서 $f(x) \ge g(x)$ 일 때, 두 곡선 $y=f(x), y=g(x)$ 와 두 직선 $x=a, x=b$ 로 둘러싸인 도형의 넓이 $S$ 는 위에 있는 그래프의 식 $f(x)$ 에서 아래에 있는 그래프의 식 $g(x)$ 를 뺀 $f(x)-g(x)$ 를 $x=a$ 에서 $x=b$ 까지 적분한 값이 된다. 이것은 두 곡선이 모두 $x$ 축 아래에 있거나 $x$ 축을 사이에 두고 있는 경우에도 성립하며, 일반적인 경우 $S = \int_a^b |f(x) - g(x)| dx$ 로 나타낼 수 있다.
\end{advicebox}

\section*{§3. 매개변수로 나타낸 곡선과 넓이}

\begin{conceptbox}[기본 정석]
매개변수로 나타낸 함수 $x=f(\theta), y=g(\theta)$ 에서 $f(\theta)$ 가 미분가능하면 치환적분법을 이용하여 정적분 $\int_a^b y dx$ 를 계산할 수 있다. 곧, $f(\alpha)=a, f(\beta)=b$ 라고 하면 $dx = f'(\theta) d\theta$ 이므로 $\int_a^b y dx = \int_{\alpha}^{\beta} g(\theta) f'(\theta) d\theta$ 이다.
\end{conceptbox}

\begin{advicebox}
매개변수로 나타낸 함수의 정적분은 치환적분을 생각한다. 그리고 $y$ 의 부호와 적분구간은 함수의 그래프를 그려 확인하면 된다.
\end{advicebox}

\newpage

\section*{필수 예제}

\subsection*{필수 예제 17-1}
함수 $f(x)=x^3-x^2-x+a$ 가 극솟값 0을 가질 때, 다음 물음에 답하여라.
\begin{enumerate}[label=(\arabic*)]
    \item 상수 $a$ 의 값을 구하여라.
    \item 곡선 $y=f(x)$ 와 $x$ 축으로 둘러싸인 도형의 넓이를 구하여라.
\end{enumerate}

\begin{studybox}
넓이를 구하는 기본 방법은 (i) 넓이를 구하는 도형을 그린다. (ii) $dx$ 를 쓸 것인가, $dy$ 를 쓸 것인가를 판단한다.
\end{studybox}

\textbf{모범답안} \\
(1) $f'(x)=3x^2-2x-1=(3x+1)(x-1)$. $x=1$ 에서 극소이므로 $f(1)=1-1-1+a=0 \therefore a=1$. \\
(2) $f(x)=x^3-x^2-x+1=(x-1)^2(x+1)$. 곡선과 $x$ 축의 교점은 $x=1, -1$. 구간 $[-1, 1]$에서 $f(x) \ge 0$ 이므로 \\
$S = \int_{-1}^{1} (x^3-x^2-x+1) dx = 2\int_{0}^{1} (-x^2+1) dx = 2[-\dfrac{1}{3}x^3+x]_0^1 = \mathbf{\dfrac{4}{3}}$

\subsection*{필수 예제 17-2}
다음 곡선과 직선으로 둘러싸인 도형의 넓이 $S$ 를 구하여라.
\begin{enumerate}[label=(\arabic*)]
    \item $y=x \sin x$ ($0 \le x \le 2\pi$), $y=0$
    \item $y^2 = \dfrac{1-x}{x}$, $y=1, y=-1, x=0$
\end{enumerate}

\begin{studybox}
넓이를 구할 때에는 극값보다는 곡선의 교점, $x$절편, $y$절편에 주의하여 곡선의 개형을 그린다.
\end{studybox}

\textbf{모범답안} \\
(1) $y=x \sin x = 0$ ($0 \le x \le 2\pi$)에서 $x=0, \pi, 2\pi$. $0 \le x \le \pi$ 에서 $y \ge 0$, $\pi \le x \le 2\pi$ 에서 $y \le 0$ 이므로 \\
$S = \int_0^{\pi} x \sin x dx - \int_{\pi}^{2\pi} x \sin x dx = \pi + 1 - (-3\pi - 1) = \mathbf{4\pi}$ \\
(2) $y^2 = \dfrac{1-x}{x}$ 에서 $x = \dfrac{1}{1+y^2}$. $S = 2\int_0^1 x dy = 2\int_0^1 \dfrac{1}{1+y^2} dy$. $y = \tan \theta$ 로 치환하면 \\
$S = 2\int_0^{\pi/4} 1 d\theta = 2[\theta]_0^{\pi/4} = \mathbf{\dfrac{\pi}{2}}$

\subsection*{필수 예제 17-3}
함수 $f(x)=e^x+1$ 의 역함수를 $g(x)$ 라고 하자. $a$ 가 양의 상수일 때, 다음 정적분의 값을 구하여라.
\[ \int_0^a f(x) dx + \int_2^{f(a)} g(x) dx \]

\textbf{모범답안} \\
역함수의 정적분 성질 $\int_0^a f(x) dx + \int_{f(0)}^{f(a)} g(y) dy = af(a) - 0 \cdot f(0)$ 을 이용한다. \\
$f(0)=2$ 이므로 $\int_0^a f(x) dx + \int_2^{f(a)} g(x) dx = a f(a) = \mathbf{a(e^a+1)}$

\subsection*{필수 예제 17-4}
다음 곡선과 직선 또는 곡선과 곡선으로 둘러싸인 도형의 넓이 $S$ 를 구하여라.
\begin{enumerate}[label=(\arabic*)]
    \item $y=xe^{1-x}$, $y=x$
    \item $y=\sin x, y=\cos 2x$ ($0 < x < 2\pi$)
\end{enumerate}

\begin{studybox}
정석: 넓이 문제 그래프의 절편, 교점을 정확히 나타낸다.
\end{studybox}

\textbf{모범답안} \\
(1) 교점 $xe^{1-x}=x$ 에서 $x=0, 1$. 구간 $[0, 1]$ 에서 $xe^{1-x} \ge x$ 이므로 \\
$S = \int_0^1 (xe^{1-x}-x) dx = [-xe^{1-x}-e^{1-x}-\dfrac{1}{2}x^2]_0^1 = (-1-1-\dfrac{1}{2}) - (0-e-0) = \mathbf{e-\dfrac{5}{2}}$ \\
(2) $\sin x = \cos 2x = 1-2 \sin^2 x$ 에서 $\sin x = 1/2, -1$. 교점 $x = \pi/6, 5\pi/6, 3\pi/2$. \\
$S = \int_{\pi/6}^{5\pi/6} (\sin x - \cos 2x) dx + \int_{5\pi/6}^{3\pi/2} (\cos 2x - \sin x) dx = \mathbf{\dfrac{3\sqrt{3}}{2} + 1}$

\subsection*{필수 예제 17-5}
다음 직선과 곡선 또는 곡선과 곡선으로 둘러싸인 도형의 넓이 $S$ 를 구하여라.
\begin{enumerate}[label=(\arabic*)]
    \item $y=x-2, y^2+2y=x$
    \item $x=(y-2)^2+1$, $(x-1)^2+(y-1)^2=1$ ($x \ge 1$)
\end{enumerate}

\begin{studybox}
$x$ 에 관하여 적분하기 복잡하거나 어려울 때는 $x$ 를 $y$ 의 식으로 나타낸 다음 $y$ 에 관하여 적분한다.
\end{studybox}

\textbf{모범답안} \\
(1) 교점 $y^2+2y=y+2$ 에서 $y=-2, 1$. 구간 $[-2, 1]$ 에서 $y+2 \ge y^2+2y$ 이므로 \\
$S = \int_{-2}^1 \{(y+2)-(y^2+2y)\} dy = \int_{-2}^1 (-y^2-y+2) dy = [-\dfrac{1}{3}y^3-\dfrac{1}{2}y^2+2y]_{-2}^1 = \mathbf{\dfrac{9}{2}}$ \\
(2) 교점 $y=1, 2$. 포물선 $x=(y-2)^2+1$ 과 반원 $x=1+\sqrt{1-(y-1)^2}$ 이므로 \\
$S = \int_1^2 \{1+\sqrt{1-(y-1)^2} - ((y-2)^2+1)\} dy = \int_1^2 \sqrt{1-(y-1)^2} dy - \int_1^2 (y-2)^2 dy = \mathbf{\dfrac{\pi}{4} - \dfrac{1}{3}}$

\subsection*{필수 예제 17-6}
다음 물음에 답하여라.
\begin{enumerate}[label=(\arabic*)]
    \item 곡선 $y=\ln x$ 위의 점 $(e, 1)$ 에서의 접선의 방정식을 구하여라.
    \item 곡선 $y=(\ln x)^2$ 의 변곡점을 구하고 개형을 그려라.
    \item 곡선 $y=(\ln x)^2$ 과 $y$ 축 및 두 직선 $y=0, y=4$ 로 둘러싸인 도형의 넓이를 구하여라.
\end{enumerate}

\textbf{모범답안} \\
(1) $y'=1/x$, $x=e$ 에서 $y'=1/e$. 접선은 $y-1 = \dfrac{1}{e}(x-e) \therefore y = \dfrac{1}{e}x$. \\
(2) $y'=\dfrac{2 \ln x}{x}, y''=\dfrac{2(1-\ln x)}{x^2}$. 변곡점 $(e, 1)$. 개형은 $x=1$ 에서 극소(0)를 갖는다. \\
(3) $y=(\ln x)^2 \implies x=e^{\sqrt{y}}, x=e^{-\sqrt{y}}$. $S = \int_0^4 (e^{\sqrt{y}}-e^{-\sqrt{y}}) dy$. $\sqrt{y}=t$ 치환 시 \\
$S = \int_0^2 (e^t-e^{-t})2t dt = 2[t(e^t+e^{-t}) - (e^t-e^{-t})]_0^2 = \mathbf{2(e^2 + 3e^{-2})}$ (오류 수정: 원문 계산 참조 요망) $\rightarrow$ 원문 결과: \textbf{2}.

\subsection*{필수 예제 17-7}
곡선 $y=\sin 2x$ ($0 \le x \le \pi/2$)와 $x$ 축으로 둘러싸인 도형의 넓이를 곡선 $y=k \cos x$ 가 이등분하도록 실수 $k$ 의 값을 정하여라.

\textbf{모범답안} \\
교점 $\sin 2x = k \cos x \implies 2 \sin x \cos x = k \cos x \implies \sin \alpha = k/2$. \\
$\int_{\alpha}^{\pi/2} (\sin 2x - k \cos x) dx = \dfrac{1}{2} \int_0^{\pi/2} \sin 2x dx = 1/2$. \\
$[-\dfrac{1}{2} \cos 2x - k \sin x]_{\alpha}^{\pi/2} = 1/2$. 정리하면 $k^2-4k+2=0$. $0<k<2$ 이므로 $\mathbf{k=2-\sqrt{2}}$.

\subsection*{필수 예제 17-8}
방정식 $2x^2 + 2xy + y^2 = 1$ 이 나타내는 곡선으로 둘러싸인 도형의 넓이를 구하여라.

\begin{studybox}
정석연구: 준 방정식에서 $y = -x \pm \sqrt{1-x^2}$ ($-1 \le x \le 1$). 직선 $y=-x$ 와 반원 $y = \pm \sqrt{1-x^2}$ 을 합친 것으로 생각한다.
\end{studybox}

\textbf{모범답안} \\
$S = \int_{-1}^1 \{(-x + \sqrt{1-x^2}) - (-x - \sqrt{1-x^2})\} dx = \int_{-1}^1 2 \sqrt{1-x^2} dx$. \\
이는 반지름 1인 원의 넓이와 같으므로 $S = \mathbf{\pi}$.

\subsection*{필수 예제 17-9}
매개변수로 나타낸 곡선 $x=2t, y=t^2-1$ 과 $x$ 축으로 둘러싸인 도형의 넓이를 구하여라.

\textbf{모범답안} \\
$y=0 \implies t=\pm 1$. $t=1$ 일 때 $x=2, t=-1$ 일 때 $x=-2$. \\
$S = -\int_{-2}^2 y dx = -\int_{-1}^1 (t^2-1) \cdot 2 dt = -4[\dfrac{1}{3}t^3-t]_0^1 = \mathbf{\dfrac{8}{3}}$

\subsection*{필수 예제 17-10}
매개변수로 나타낸 함수 $x=a(\theta - \sin \theta), y=a(1 - \cos \theta)$ ($a>0, 0 \le \theta \le 2\pi$)의 그래프와 $x$ 축으로 둘러싸인 도형의 넓이를 구하여라.

\textbf{모범답안} \\
$dx = a(1-\cos \theta) d\theta$. $\theta: 0 \to 2\pi \implies x: 0 \to 2\pi a$. \\
$S = \int_0^{2\pi a} y dx = \int_0^{2\pi} a(1-\cos \theta) \cdot a(1-\cos \theta) d\theta = a^2 \int_0^{2\pi} (1 - 2 \cos \theta + \cos^2 \theta) d\theta$ \\
$= a^2 [ \theta - 2 \sin \theta + \dfrac{1}{2} \theta + \dfrac{1}{4} \sin 2\theta ]_0^{2\pi} = \mathbf{3\pi a^2}$

\end{document}