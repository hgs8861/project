\documentclass[a4paper,11pt]{article}
\usepackage{kotex}
\usepackage{amsmath, amssymb, amsthm}
\usepackage{geometry}
\usepackage{tcolorbox}
\usepackage{enumitem}
\tcbuselibrary{skins, breakable}

% --- 여백 및 간격 설정 ---
\geometry{left=2.5cm, right=2.5cm, top=2.5cm, bottom=2.5cm}
\linespread{1.4}
\setlength{\parskip}{1em}
\setlist[enumerate]{itemsep=5pt, topsep=0pt}

% --- 사용자 정의 박스 스타일 ---
\newtcolorbox{conceptbox}[1][]{
  colback=blue!5, colframe=blue!60, coltitle=white, fonttitle=\bfseries,
  title={#1}, arc=1mm, boxrule=0.5mm, breakable, parskip=1em
}
\newtcolorbox{advicebox}{
  colback=orange!5, colframe=orange!60, title={Advice}, fonttitle=\bfseries,
  coltitle=white, arc=1mm, boxrule=0.5mm, breakable, parskip=1em
}
\newtcolorbox{studybox}{
  colback=gray!10, colframe=gray!60, title={정석연구}, fonttitle=\bfseries,
  coltitle=white, arc=1mm, boxrule=0.5mm, breakable, parskip=1em
}

\title{\textbf{미적분 - 제15장 정적분의 계산}}
\author{수학의 정석}
\date{}

\begin{document}

\maketitle

\section*{§1. 정적분의 정의와 계산}

\begin{conceptbox}[기본 정석]
\begin{enumerate}
    \item \textbf{정적분의 정의}\\
    함수 $f(x)$가 닫힌구간 $[a, b]$에서 연속일 때, $f(x)$의 한 부정적분을 $F(x)$라고 하면
    \[ \int_{a}^{b} f(x) dx = \Big[ F(x) \Big]_{a}^{b} = F(b) - F(a) \]
    \item $\displaystyle\int_{a}^{a} f(x) dx = 0$
    \item $\displaystyle\int_{a}^{b} f(x) dx = -\displaystyle\int_{b}^{a} f(x) dx$
    \item \textbf{정적분의 기본 공식}\\
    (1) $\displaystyle\int_{a}^{b} k f(x) dx = k \int_{a}^{b} f(x) dx$ (단, $k$는 상수)\\
    (2) $\displaystyle\int_{a}^{b} \{f(x) \pm g(x)\} dx = \int_{a}^{b} f(x) dx \pm \int_{a}^{b} g(x) dx$\\
    (3) $\displaystyle\int_{a}^{b} f(x) dx = \int_{a}^{c} f(x) dx + \int_{c}^{b} f(x) dx$
\end{enumerate}
\end{conceptbox}



\begin{advicebox}
\textbf{정적분의 값과 변수}\\
부정적분 $\int f(x) dx$는 $x$의 함수이지만, 정적분 $\int_{a}^{b} f(x) dx$는 그 결과가 상수이다. 따라서 적분 변수를 다른 문자로 바꾸어도 그 값은 변하지 않는다.
\[ \int_{a}^{b} f(x) dx = \int_{a}^{b} f(t) dt = \int_{a}^{b} f(u) du \]
\end{advicebox}

\section*{§2. 치환적분법과 부분적분법}

\begin{conceptbox}[기본 정석]
\begin{enumerate}
    \item \textbf{정적분의 치환적분법}\\
    함수 $g(x)$의 도함수 $g'(x)$가 구간 $[a, b]$에서 연속이고, $f(x)$가 $g(x)$의 함숫값을 포함하는 구간에서 연속일 때, $g(a)=\alpha, g(b)=\beta$이면
    \[ \int_{a}^{b} f(g(x))g'(x) dx = \int_{\alpha}^{\beta} f(t) dt \quad (\text{단, } g(x)=t) \]
    \item \textbf{정적분의 부분적분법}\\
    두 함수 $f(x), g(x)$가 미분가능하고, $f'(x), g'(x)$가 연속일 때
    \[ \int_{a}^{b} f(x)g'(x) dx = \Big[ f(x)g(x) \Big]_{a}^{b} - \int_{a}^{b} f'(x)g(x) dx \]
\end{enumerate}
\end{conceptbox}

\begin{advicebox}
\textbf{삼각치환법}\\
정적분에서 다음과 같은 꼴이 포함된 경우 삼각함수를 이용하여 치환한다.\\
(1) $\sqrt{a^2-x^2}$ 꼴 : $x = a \sin \theta$ ($-\frac{\pi}{2} \le \theta \le \frac{\pi}{2}$)로 치환\\
(2) $a^2+x^2$ 꼴 : $x = a \tan \theta$ ($-\frac{\pi}{2} < \theta < \frac{\pi}{2}$)로 치환
\end{advicebox}



\newpage

\section*{필수 예제}

\subsection*{필수 예제 15-1}
다음 정적분의 값을 구하여라.
\begin{enumerate}[label=(\arabic*)]
    \item $\displaystyle\int_{1}^{4} \frac{(\sqrt{x}+1)^3}{\sqrt{x}} dx$
    \item $\displaystyle\int_{-1}^{2} \frac{x}{x^2-x-6} dx$
    \item $\displaystyle\int_{0}^{\pi} (e^{4x} - \sin^2 x) dx$
\end{enumerate}

\begin{studybox}
부정적분을 구한 다음 정적분의 정의를 이용한다. (2)는 분모를 인수분해하여 부분분수로 변형하고, (3)은 반각의 공식을 이용하여 차수를 낮춘다.
\end{studybox}

\textbf{모범답안}
\begin{enumerate}[label=(\arabic*)]
    \item $\sqrt{x}=t$로 놓으면 $\frac{1}{2\sqrt{x}}dx=dt$이다. $x=1 \to t=1, x=4 \to t=2$\\
    $\text{준식} = 2 \int_{1}^{2} (t+1)^3 dt = 2 \left[ \frac{1}{4}(t+1)^4 \right]_{1}^{2} = \frac{1}{2}(81-16) = \mathbf{\frac{65}{2}}$
    \item $\frac{x}{(x-3)(x+2)} = \frac{1}{5} \left( \frac{3}{x-3} + \frac{2}{x+2} \right)$ 이므로\\
    $\text{준식} = \frac{1}{5} [3\ln|x-3| + 2\ln|x+2|]_{-1}^{2} = \frac{1}{5} \{(3\ln 1 + 2\ln 4) - (3\ln 4 + 2\ln 1)\} = \mathbf{-\frac{2}{5}\ln 2}$
    \item $\int_{0}^{\pi} (e^{4x} - \frac{1-\cos 2x}{2}) dx = \left[ \frac{1}{4}e^{4x} - \frac{1}{2}x + \frac{1}{4}\sin 2x \right]_{0}^{\pi} = \mathbf{\frac{1}{4}(e^{4\pi}-1) - \frac{\pi}{2}}$
\end{enumerate}

\vspace{0.5cm}
\hrule
\vspace{0.5cm}

\subsection*{필수 예제 15-2}
다음 정적분의 값을 구하여라.
\begin{enumerate}[label=(\arabic*)]
    \item $\displaystyle\int_{0}^{1} \frac{x^3}{x+1} dx + \int_{0}^{1} \frac{1}{t+1} dt$
    \item $\displaystyle\int_{0}^{\ln 3} \frac{e^{3x}}{e^x+1} dx - \int_{\ln 3}^{0} \frac{1}{e^t+1} dt$
\end{enumerate}

\begin{studybox}
정적분의 성질 $\int_a^b f(x)dx + \int_a^b g(x)dx = \int_a^b \{f(x)+g(x)\}dx$와 적분 변수의 임의성을 이용한다.
\end{studybox}

\textbf{모범답안}
\begin{enumerate}[label=(\arabic*)]
    \item $\text{준식} = \int_{0}^{1} \frac{x^3+1}{x+1} dx = \int_{0}^{1} (x^2-x+1) dx = \left[ \frac{1}{3}x^3 - \frac{1}{2}x^2 + x \right]_{0}^{1} = \mathbf{\frac{5}{6}}$
    \item $\text{준식} = \int_{0}^{\ln 3} \frac{e^{3x}}{e^x+1} dx + \int_{0}^{\ln 3} \frac{1}{e^x+1} dx = \int_{0}^{\ln 3} (e^{2x}-e^x+1) dx = \left[ \frac{1}{2}e^{2x}-e^x+x \right]_{0}^{\ln 3} = \mathbf{2+\ln 3}$
\end{enumerate}

\vspace{0.5cm}
\hrule
\vspace{0.5cm}

\subsection*{필수 예제 15-3}
다음 정적분의 값을 구하여라.
\begin{enumerate}[label=(\arabic*)]
    \item $\displaystyle\int_{0}^{1} |e^x-2| dx$
    \item $\displaystyle\int_{0}^{\pi} (|\sin x| + |\cos 2x|) dx$
\end{enumerate}

\begin{studybox}
절댓값 기호 안의 식의 부호가 바뀌는 점을 경계로 적분 구간을 나누어 계산한다.
\end{studybox}

\textbf{모범답안}
\begin{enumerate}[label=(\arabic*)]
    \item $e^x-2=0 \to x=\ln 2$. 구간 $[0, \ln 2]$에서 $e^x-2 \le 0$, $[\ln 2, 1]$에서 $e^x-2 \ge 0$\\
    $\text{준식} = \int_{0}^{\ln 2} (2-e^x) dx + \int_{\ln 2}^{1} (e^x-2) dx = [2x-e^x]_0^{\ln 2} + [e^x-2x]_{\ln 2}^1 = \mathbf{4\ln 2 + e - 5}$
    \item $\int_{0}^{\pi} \sin x dx + \int_{0}^{\pi} |\cos 2x| dx = [-\cos x]_0^\pi + 4\int_{0}^{\pi/4} \cos 2x dx = 2 + 4[\frac{1}{2}\sin 2x]_0^{\pi/4} = \mathbf{4}$
\end{enumerate}

\vspace{0.5cm}
\hrule
\vspace{0.5cm}

\subsection*{필수 예제 15-4}
$a$가 실수일 때, 다음 물음에 답하여라.
\begin{enumerate}[label=(\arabic*)]
    \item 정적분 $I = \displaystyle\int_{0}^{1} |e^x-a| dx$ 의 값을 $a$로 나타내어라.
    \item $I$의 값이 최소가 되는 $a$의 값을 구하여라.
\end{enumerate}

\begin{studybox}
$a$의 값의 범위에 따라 구간을 나누어 정적분한다. $I$가 $a$에 대한 함수이므로 미분을 통해 최솟값을 찾는다.
\end{studybox}

\textbf{모범답안}
\begin{enumerate}[label=(\arabic*)]
    \item (i) $a \le 1$ 일 때 : $I = \int_{0}^{1} (e^x-a) dx = e-a-1$\\
    (ii) $1 < a < e$ 일 때 : $I = \int_{0}^{\ln a} (a-e^x) dx + \int_{\ln a}^{1} (e^x-a) dx = 2a\ln a - 3a + e + 1$\\
    (iii) $a \ge e$ 일 때 : $I = \int_{0}^{1} (a-e^x) dx = a-e+1$
    \item $f(a) = 2a\ln a - 3a + e + 1$이라 하면 $f'(a) = 2\ln a - 1 = 0 \to a = \sqrt{e}$.\\
    증감표를 그리면 $a=\mathbf{\sqrt{e}}$에서 최소이다.
\end{enumerate}

\vspace{0.5cm}
\hrule
\vspace{0.5cm}

\subsection*{필수 예제 15-5}
다음 정적분의 값을 구하여라.
\begin{enumerate}[label=(\arabic*)]
    \item $\displaystyle\int_{0}^{\pi} (1-\cos^3 x)\cos x \sin x dx$
    \item $\displaystyle\int_{\pi/6}^{\pi/2} \frac{\cos x}{\sin x + \sin^3 x} dx$
    \item $\displaystyle\int_{1}^{e} \ln x^{\frac{1}{x}} dx$
    \item $\displaystyle\int_{0}^{1} \frac{e^x-1}{e^{2x}+e^{-x}} dx$
\end{enumerate}

\begin{studybox}
치환적분법을 이용한다. (1) $\cos x = t$, (2) $\sin x = t$, (3) $\ln x = t$, (4) $e^x = t$로 치환한다.
\end{studybox}

\textbf{모범답안}
\begin{enumerate}[label=(\arabic*)]
    \item $\cos x = t \to -\sin x dx = dt$. $x: 0 \to \pi \Rightarrow t: 1 \to -1$\\
    $\int_{1}^{-1} (1-t^3)t (-dt) = \int_{-1}^{1} (t-t^4) dt = 2\int_{0}^{1} (-t^4) dt = \mathbf{-\frac{2}{5}}$
    \item $\sin x = t \to \cos x dx = dt$. $x: \pi/6 \to \pi/2 \Rightarrow t: 1/2 \to 1$\\
    $\int_{1/2}^{1} \frac{1}{t+t^3} dt = \int_{1/2}^{1} (\frac{1}{t} - \frac{t}{1+t^2}) dt = [\ln |t| - \frac{1}{2}\ln(1+t^2)]_{1/2}^1 = \mathbf{\frac{1}{2}\ln \frac{5}{2}}$
    \item $\ln x = t \to \frac{1}{x}dx = dt$. $x: 1 \to e \Rightarrow t: 0 \to 1$\\
    $\int_{0}^{1} t dt = [\frac{1}{2}t^2]_0^1 = \mathbf{\frac{1}{2}}$
    \item $e^x=t \to e^xdx=dt$. $\int_{1}^{e} \frac{t-1}{t^2+1/t} \cdot \frac{1}{t} dt = \int_{1}^{e} \frac{t-1}{t^3+1} dt = \int_{1}^{e} (\frac{-1}{t+1} + \frac{t}{t^2-t+1}) dt$\\
    계산하면 $\mathbf{\frac{1}{2}\ln \frac{e^2-e+1}{(e+1)^2} + \frac{\sqrt{3}\pi}{18}}$ (생략 가능)
\end{enumerate}

\vspace{0.5cm}
\hrule
\vspace{0.5cm}

\subsection*{필수 예제 15-6}
다음 정적분의 값을 구하여라. ($a>0$)
\begin{enumerate}[label=(\arabic*)]
    \item $\displaystyle\int_{0}^{a} \sqrt{a^2-x^2} dx$
    \item $\displaystyle\int_{0}^{a} \frac{1}{x^2+a^2} dx$
\end{enumerate}

\begin{studybox}
삼각치환법을 이용한다. (1) $x = a \sin \theta$, (2) $x = a \tan \theta$로 치환한다.
\end{studybox}

\textbf{모범답안}
\begin{enumerate}[label=(\arabic*)]
    \item $x=a\sin\theta, dx=a\cos\theta d\theta$. $x: 0 \to a \Rightarrow \theta: 0 \to \pi/2$\\
    $\int_{0}^{\pi/2} \sqrt{a^2-a^2\sin^2\theta} \cdot a\cos\theta d\theta = a^2 \int_{0}^{\pi/2} \cos^2\theta d\theta = a^2 \int_{0}^{\pi/2} \frac{1+\cos 2\theta}{2} d\theta = \mathbf{\frac{\pi a^2}{4}}$
    \item $x=a\tan\theta, dx=a\sec^2\theta d\theta$. $x: 0 \to a \Rightarrow \theta: 0 \to \pi/4$\\
    $\int_{0}^{\pi/4} \frac{1}{a^2\sec^2\theta} \cdot a\sec^2\theta d\theta = \frac{1}{a} \int_{0}^{\pi/4} 1 d\theta = \mathbf{\frac{\pi}{4a}}$
\end{enumerate}

\vspace{0.5cm}
\hrule
\vspace{0.5cm}

\subsection*{필수 예제 15-7}
다음을 증명하여라.
\begin{enumerate}[label=(\arabic*)]
    \item $f(x)$가 우함수($f(-x)=f(x)$)이면 $\displaystyle\int_{-a}^{a} f(x) dx = 2\int_{0}^{a} f(x) dx$
    \item $f(x)$가 기함수($f(-x)=-f(x)$)이면 $\displaystyle\int_{-a}^{a} f(x) dx = 0$
\end{enumerate}

\begin{studybox}
$\int_{-a}^{a} f(x) dx = \int_{-a}^{0} f(x) dx + \int_{0}^{a} f(x) dx$로 나누고, 앞의 적분에서 $x=-t$로 치환한다.
\end{studybox}

\textbf{모범답안}
\begin{enumerate}[label=(\arabic*)]
    \item $\int_{-a}^{0} f(x) dx$에서 $x=-t$로 치환하면 $dx=-dt$. $x:-a \to 0 \Rightarrow t: a \to 0$\\
    $\int_{a}^{0} f(-t)(-dt) = \int_{0}^{a} f(-t) dt = \int_{0}^{a} f(t) dt = \int_{0}^{a} f(x) dx$\\
    따라서 $\int_{-a}^{a} f(x) dx = \int_{0}^{a} f(x) dx + \int_{0}^{a} f(x) dx = 2\int_{0}^{a} f(x) dx$
    \item 우함수의 증명과 동일하게 $\int_{-a}^{0} f(x) dx = \int_{0}^{a} f(-x) dx$이다.\\
    기함수는 $f(-x)=-f(x)$이므로 $\int_{0}^{a} -f(x) dx + \int_{0}^{a} f(x) dx = \mathbf{0}$
\end{enumerate}

\vspace{0.5cm}
\hrule
\vspace{0.5cm}

\subsection*{필수 예제 15-8}
다음 정적분의 값을 구하여라.
\begin{enumerate}[label=(\arabic*)]
    \item $\displaystyle\int_{0}^{\pi} x|\cos x| dx$
    \item $\displaystyle\int_{0}^{1} \ln(\sqrt{x^2+1}-x) dx$
    \item $\displaystyle\int_{0}^{1} x^2 e^{2x} dx$
\end{enumerate}

\begin{studybox}
정적분의 부분적분법을 이용한다. (1)은 절댓값 때문에 구간을 나누어 부분적분을 적용한다.
\end{studybox}

\textbf{모범답안}
\begin{enumerate}[label=(\arabic*)]
    \item $\int_{0}^{\pi/2} x\cos x dx - \int_{\pi/2}^{\pi} x\cos x dx = [x\sin x + \cos x]_0^{\pi/2} - [x\sin x + \cos x]_{\pi/2}^\pi = (\frac{\pi}{2}-1) - (-1-\frac{\pi}{2}) = \mathbf{\pi}$
    \item $u = \ln(\sqrt{x^2+1}-x), v' = 1$이라 하면 $u' = \frac{-1}{\sqrt{x^2+1}}, v = x$\\
    $[x\ln(\sqrt{x^2+1}-x)]_0^1 + \int_0^1 \frac{x}{\sqrt{x^2+1}} dx = \ln(\sqrt{2}-1) + [\sqrt{x^2+1}]_0^1 = \mathbf{\ln(\sqrt{2}-1) + \sqrt{2}-1}$
    \item $[x^2 \cdot \frac{1}{2}e^{2x}]_0^1 - \int_0^1 x e^{2x} dx = \frac{e^2}{2} - ([x \cdot \frac{1}{2}e^{2x}]_0^1 - \int_0^1 \frac{1}{2}e^{2x} dx) = \frac{e^2}{2} - \frac{e^2}{2} + [\frac{1}{4}e^{2x}]_0^1 = \mathbf{\frac{e^2-1}{4}}$
\end{enumerate}

\vspace{0.5cm}
\hrule
\vspace{0.5cm}

\subsection*{필수 예제 15-9}
다음을 만족시키는 연속함수 $f(x), g(x)$를 구하여라.
\[ f(x) = x^2 + \int_{0}^{1} tg(t) dt, \quad g(x) = e^{-x} + x \int_{0}^{1} f(t) dt \]

\begin{studybox}
정적분의 결과는 상수임을 이용한다. $\int_{0}^{1} tg(t) dt = a, \int_{0}^{1} f(t) dt = b$로 놓고 연립방정식을 푼다.
\end{studybox}

\textbf{모범답안}\\
$f(x) = x^2+a, g(x) = e^{-x}+bx$를 적분식에 대입\\
$a = \int_0^1 t(e^{-t}+bt) dt = [-te^{-t}-e^{-t}+\frac{b}{3}t^3]_0^1 = 1-\frac{2}{e}+\frac{b}{3}$\\
$b = \int_0^1 (t^2+a) dt = [\frac{1}{3}t^3+at]_0^1 = \frac{1}{3}+a$\\
연립하면 $a = \frac{5}{3}-\frac{3}{e}, b = 2-\frac{3}{e}$\\
$\therefore \mathbf{f(x) = x^2 + \frac{5}{3}-\frac{3}{e}, \quad g(x) = e^{-x} + (2-\frac{3}{e})x}$

\vspace{0.5cm}
\hrule
\vspace{0.5cm}

\subsection*{필수 예제 15-10}
$f(y) = \displaystyle\int_{0}^{y} e^{ax}\cos x dx$ 일 때, 다음 물음에 답하여라.
\begin{enumerate}[label=(\arabic*)]
    \item $f(y)$를 구하여라.
    \item $\displaystyle\lim_{y \to \infty} f(y)$의 값이 존재하기 위한 $a$의 범위와 극한값을 구하여라.
\end{enumerate}

\begin{studybox}
부분적분을 두 번 사용하여 정적분을 구한다. (2)는 지수함수의 극한 성질을 이용한다.
\end{studybox}

\textbf{모범답안}
\begin{enumerate}[label=(\arabic*)]
    \item $\int e^{ax}\cos x dx = \frac{e^{ax}(a\cos x + \sin x)}{a^2+1} + C$ 이므로\\
    $f(y) = \mathbf{\frac{e^{ay}(a\cos y + \sin y) - a}{a^2+1}}$
    \item $y \to \infty$ 일 때 $e^{ay}$가 수렴해야 하므로 $\mathbf{a < 0}$이다. 이때 극한값은 $\mathbf{-\frac{a}{a^2+1}}$
\end{enumerate}

\vspace{0.5cm}
\hrule
\vspace{0.5cm}

\subsection*{필수 예제 15-11}
자연수 $n$에 대하여 $I_n = \displaystyle\int_{0}^{\pi/2} \sin^n x dx$ 라 할 때, 다음을 구하여라.
\begin{enumerate}[label=(\arabic*)]
    \item $I_n = \frac{n-1}{n} I_{n-2}$ ($n \ge 2$)가 성립함을 증명하여라.
    \item $I_5, I_{10}$의 값을 구하여라.
\end{enumerate}

\begin{studybox}
부분적분을 통해 점화식을 유도한다. $\sin^n x = \sin^{n-1} x \sin x$로 생각한다.
\end{studybox}

\textbf{모범답안}
\begin{enumerate}[label=(\arabic*)]
    \item $u = \sin^{n-1}x, v' = \sin x$로 부분적분하면\\
    $I_n = [-\sin^{n-1}x \cos x]_0^{\pi/2} + (n-1) \int_0^{\pi/2} \sin^{n-2}x \cos^2 x dx$\\
    $= (n-1) \int_0^{\pi/2} \sin^{n-2}x (1-\sin^2 x) dx = (n-1)I_{n-2} - (n-1)I_n$\\
    정리하면 $n I_n = (n-1)I_{n-2} \to \mathbf{I_n = \frac{n-1}{n} I_{n-2}}$
    \item $I_5 = \frac{4}{5} \cdot \frac{2}{3} \cdot I_1 = \frac{8}{15} \cdot [-\cos x]_0^{\pi/2} = \mathbf{\frac{8}{15}}$\\
    $I_{10} = \frac{9}{10} \cdot \frac{7}{8} \cdot \frac{5}{6} \cdot \frac{3}{4} \cdot \frac{1}{2} \cdot I_0 = \frac{63}{512} \cdot \frac{\pi}{2} = \mathbf{\frac{63\pi}{1024}}$
\end{enumerate}

\end{document}