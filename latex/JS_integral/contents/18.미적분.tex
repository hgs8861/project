\documentclass[a4paper,11pt]{article}
\usepackage{kotex}
\usepackage{amsmath, amssymb, amsthm}
\usepackage{geometry}
\usepackage{tcolorbox}
\usepackage{enumitem}
\tcbuselibrary{skins, breakable}

% --- 여백 및 간격 설정 ---
\geometry{left=2.5cm, right=2.5cm, top=2.5cm, bottom=2.5cm}
\linespread{1.4}
\setlength{\parskip}{1em}
\setlist[enumerate]{itemsep=5pt, topsep=0pt}

% --- 사용자 정의 박스 스타일 ---
\newtcolorbox{conceptbox}[1][]{
  colback=blue!5, colframe=blue!60, coltitle=white, fonttitle=\bfseries,
  title={#1}, arc=1mm, boxrule=0.5mm, breakable, parskip=1em
}
\newtcolorbox{advicebox}{
  colback=orange!5, colframe=orange!60, title={Advice}, fonttitle=\bfseries,
  coltitle=white, arc=1mm, boxrule=0.5mm, breakable, parskip=1em
}
\newtcolorbox{studybox}{
  colback=gray!10, colframe=gray!60, title={정석연구}, fonttitle=\bfseries,
  coltitle=white, arc=1mm, boxrule=0.5mm, breakable, parskip=1em
}

\title{\textbf{미적분 - 제18장 부피와 적분}}
\author{수학의 정석}
\date{}

\begin{document}

\maketitle

\section*{§1. 일반 입체의 부피}

\begin{conceptbox}[기본 정석]
\textbf{일반 입체의 부피} \\
$x$축에 수직인 평면으로 어떤 입체를 자를 때, 자른 단면의 넓이가 $S(x)$ 이면 이 입체의 $x=a$ 와 $x=b$(단, $a < b$) 사이에 있는 부분의 부피 $V$ 는 
\[ V = \int_{a}^{b} S(x) dx \]
\end{conceptbox}

\begin{advicebox}
\textbf{일반 입체의 부피} \\
주어진 입체에 대하여 한 직선을 $x$축으로 정하여 $x$좌표가 $x$(단, $a \le x \le b$)인 점을 지나고 $x$축에 수직인 평면으로 입체를 자른 단면의 넓이를 $S(x)$ 라고 할 때, 이 입체의 $x=a$ 와 $x=b$ 사이에 있는 부분의 부피 $V$ 를 구하는 방법을 알아보자. \\
닫힌구간 $[a, b]$ 를 $n$ 등분하여 양 끝 점과 각 분점의 $x$좌표를 왼쪽부터 $x_0(=a), x_1, x_2, \dots, x_{n-1}, x_n(=b)$ 이라 하고, 각 소구간의 길이를 $\Delta x$ 라고 하면 
\[ \Delta x = \dfrac{b-a}{n}, \quad x_k = a + k\Delta x \]
이때, $x$좌표가 $x_k (k=1, 2, \dots, n)$ 인 점을 지나고 $x$축에 수직인 평면으로 입체를 자른 단면을 밑면으로 하고 높이가 $\Delta x$ 인 기둥의 부피는 $S(x_k)\Delta x$ 이므로 이들 $n$ 개의 기둥의 부피의 합을 $V_n$ 이라고 하면 
\[ V_n = \sum_{k=1}^{n} S(x_k)\Delta x \]
이고, 여기에서 $n \to \infty$ 일 때 $V_n \to V$ 이다. 따라서 정적분과 급수의 관계에 의하여 구하는 부피 $V$ 는 
\[ V = \lim_{n \to \infty} V_n = \lim_{n \to \infty} \sum_{k=1}^{n} S(x_k)\Delta x = \int_{a}^{b} S(x) dx \]
곧, 단면의 넓이를 $x$의 함수 $S(x)$ 로 나타낼 수 있는 입체는 그 부피를 정적분으로 나타내어 구할 수 있다.
\end{advicebox}

\begin{advicebox}
이와 같은 입체의 부피와 정적분의 관계를 다음과 같이 생각할 수도 있다.
\begin{center}
부피 요소 $\xrightarrow{\text{합}}$ 부피 요소의 합 $\xrightarrow{\text{한없이 세분한 극한}}$ 부피 \\
$S(x_k)\Delta x \implies \sum_{k=1}^{n} S(x_k)\Delta x \implies \lim_{n \to \infty} \sum_{k=1}^{n} S(x_k)\Delta x = \int_{a}^{b} S(x) dx$
\end{center}
\end{advicebox}

\section*{§2. 회전체의 부피}

\begin{conceptbox}[기본 정석]
\textbf{① $x$축을 회전축으로 하는 회전체} \\
곡선 $y=f(x)$ (단, $a \le x \le b$)를 $x$축 둘레로 회전시킨 회전체의 부피를 $V$ 라고 하면 
\[ V = \int_{a}^{b} \pi y^2 dx = \pi \int_{a}^{b} \{f(x)\}^2 dx \]
\textbf{② $y$축을 회전축으로 하는 회전체} \\
곡선 $x=g(y)$ (단, $\alpha \le y \le \beta$)를 $y$축 둘레로 회전시킨 회전체의 부피를 $V$ 라고 하면 
\[ V = \int_{\alpha}^{\beta} \pi x^2 dy = \pi \int_{\alpha}^{\beta} \{g(y)\}^2 dy \]
\end{conceptbox}

\begin{advicebox}
\textbf{회전체의 부피} \\
회전체는 일반 입체의 특수한 경우로서 단면이 원이다. 따라서 단면인 원의 넓이를 $x$에 관한 식으로 나타낸 다음, 이를 일반 입체의 부피 공식 $\int_{a}^{b} S(x) dx$ 에 대입하면 된다. \\
곧, 회전체의 부피를 $V$ 라고 하면 \\
(i) 곡선 $y=f(x)$ ($a \le x \le b$)를 $x$축 둘레로 회전시킨 입체일 때에는 $S(x) = \pi y^2 = \pi \{f(x)\}^2$ 이므로 
\[ V = \int_{a}^{b} \pi y^2 dx = \pi \int_{a}^{b} y^2 dx = \pi \int_{a}^{b} \{f(x)\}^2 dx \]
(ii) 곡선 $x=g(y)$ ($\alpha \le y \le \beta$)를 $y$축 둘레로 회전시킨 입체일 때에는 $S(y) = \pi x^2 = \pi \{g(y)\}^2$ 이므로 
\[ V = \int_{\alpha}^{\beta} \pi x^2 dy = \pi \int_{\alpha}^{\beta} x^2 dy = \pi \int_{\alpha}^{\beta} \{g(y)\}^2 dy \]
\end{advicebox}

\begin{advicebox}
\textbf{회전축을 포함하는 경우} \\
회전하는 도형이 회전축을 포함하고 있으면 어떤 부분은 회전체의 모양에 영향을 주지 않는다. 이 경우 회전축의 한쪽으로 모아서 생각하면 된다. 
\end{advicebox}

\newpage

\section*{필수 예제}

\subsection*{필수 예제 18-1}
어떤 그릇에 수면의 높이가 $x$ cm가 되도록 물을 넣을 때, 물의 부피 $V$ cm$^3$ 는 다음과 같다고 한다. 
\[ V = x^3 - 3x^2 + 4x \]
(1) 수면의 높이가 5cm일 때, 수면의 넓이를 구하여라. \\
(2) 수면의 넓이가 13 cm$^2$ 일 때, 수면의 높이를 구하여라.

\begin{studybox}
수면의 높이가 $t$ 일 때 수면의 넓이를 $S(t)$ 라고 하면, 높이가 $x$ 일 때 물의 부피 $V$ 는 
\[ V = \int_{0}^{x} S(t) dt \]
이다. 그런데 문제의 조건에서 $V = x^3 - 3x^2 + 4x$ 이므로 
\[ \int_{0}^{x} S(t) dt = x^3 - 3x^2 + 4x \]
여기에서 $S(x)$ 를 구할 때에는 다음 정적분과 미분의 관계를 이용한다. 
\textbf{정석} $\dfrac{d}{dx} \int_{a}^{x} S(t) dt = S(x)$ ($a$는 상수)
\end{studybox}

\textbf{모범답안} \\
수면의 높이가 $t$ 일 때 수면의 넓이를 $S(t)$ 라고 하면, 높이가 $x$ 일 때 물의 부피는 $\int_{0}^{x} S(t) dt$ 이다. 따라서 문제의 조건으로부터 
\[ \int_{0}^{x} S(t) dt = x^3 - 3x^2 + 4x \]
양변을 $x$에 관하여 미분하면 $S(x) = 3x^2 - 6x + 4$ \\
(1) $x=5$ 일 때이므로 $S(5) = 3 \times 5^2 - 6 \times 5 + 4 = \mathbf{49(cm^2)}$ \\
(2) $S(x) = 13$ 일 때이므로 $3x^2 - 6x + 4 = 13 \implies 3(x-3)(x+1) = 0$ \\
$x > 0$ 이므로 $x = \mathbf{3(cm)}$

\subsection*{필수 예제 18-2}
$xy$평면에 곡선 $y = \dfrac{1}{x} \ln x$ 와 $x$축 및 직선 $x = e^2$ 으로 둘러싸인 도형이 있다. 이 도형을 밑면으로 하는 입체를 $x$축에 수직인 평면으로 자른 단면이 모두 반원일 때, 이 입체의 부피를 구하여라.

\begin{studybox}
단면의 넓이를 $S(x)$ 라고 할 때, 입체의 부피 $V$ 는 다음과 같이 정적분으로 나타내어 구할 수 있다. 
\[ V = \int_{a}^{b} S(x) dx \]
\end{studybox}

\textbf{모범답안} \\
$y = \dfrac{\ln x}{x}$ 에서 $y' = \dfrac{1-\ln x}{x^2}$ 이므로 $y'=0$ 에서 $x=e$. 
점 $(x, 0)$ 을 지나고 $x$축에 수직인 평면으로 자른 단면의 넓이를 $S(x)$ 라고 하면 
\[ S(x) = \dfrac{1}{2} \pi \left(\dfrac{y}{2}\right)^2 = \dfrac{\pi}{8} \times \dfrac{(\ln x)^2}{x^2} \]
따라서 구하는 부피를 $V$ 라고 하면 
\[ V = \int_{1}^{e^2} S(x) dx = \int_{1}^{e^2} \dfrac{\pi}{8} \dfrac{(\ln x)^2}{x^2} dx = \dfrac{\pi}{8} \left\{ \left[ -\dfrac{1}{x}(\ln x)^2 \right]_{1}^{e^2} + 2 \int_{1}^{e^2} \dfrac{\ln x}{x^2} dx \right\} \]
\[ = \dfrac{\pi}{8} \left( -\dfrac{4}{e^2} + 2 \left[ -\dfrac{1}{x} \ln x - \dfrac{1}{x} \right]_{1}^{e^2} \right) = \dfrac{\pi}{8} \left\{ -\dfrac{4}{e^2} + 2 \left( -\dfrac{2}{e^2} - \dfrac{1}{e^2} \right) + 2 \right\} = \mathbf{\dfrac{\pi}{4e^2}(e^2 - 5)} \]

\subsection*{필수 예제 18-3}
밑면의 반지름의 길이가 $a$ 이고 높이가 $2a$ 인 원기둥이 있다. 밑면의 한 지름을 포함하고 밑면과 이루는 각의 크기가 $60^\circ$ 인 평면으로 이 원기둥을 두 개의 부분으로 나눌 때, 작은 쪽의 부피 $V$ 를 구하여라.

\begin{studybox}
입체의 부피는 다음 순서로 구한다. \\
첫째-$x$축과 원점을 정한다. \\
둘째-$x$좌표가 $x$인 점을 지나고 $x$축에 수직인 평면으로 자른 입체의 단면의 넓이 $S(x)$ 를 구한다. \\
셋째-필요한 구간에서 단면의 넓이 $S(x)$ 를 적분한다. 
\end{studybox}

\textbf{모범답안} \\
밑면의 지름 $AB$ 를 $x$축, 밑면의 중심 $O$ 를 원점으로 하자. $x$축 위에 점 $P(x, 0) (-a \le x \le a)$ 을 잡고, 점 $P$ 를 지나고 $x$축에 수직인 평면으로 자른 입체의 단면의 넓이를 $S(x)$ 라고 하면 $\overline{PQ} = \sqrt{a^2-x^2}$ 이고 $\overline{QR} = \overline{PQ} \tan 60^\circ = \sqrt{3}\sqrt{a^2-x^2}$ 이다. 
\[ S(x) = \triangle PQR = \dfrac{1}{2} \times \overline{PQ} \times \overline{QR} = \dfrac{\sqrt{3}}{2}(a^2-x^2) \]
\[ V = \int_{-a}^{a} S(x) dx = 2 \int_{0}^{a} \dfrac{\sqrt{3}}{2}(a^2-x^2) dx = \sqrt{3} \left[ a^2x - \dfrac{1}{3}x^3 \right]_{0}^{a} = \mathbf{\dfrac{2\sqrt{3}}{3} a^3} \]

\subsection*{필수 예제 18-4}
(1) 반원 $x^2+y^2=4 (x \ge 0)$ 와 포물선 $y^2=3x$ 로 둘러싸인 도형을 $x$축 둘레로 회전시킨 입체의 부피 $V$ 를 구하여라. \\
(2) 곡선 $y = \sqrt{x}\sin x (0 \le x \le \pi)$ 와 $x$축으로 둘러싸인 도형을 $x$축 둘레로 회전시킨 입체의 부피 $V$ 를 구하여라.

\begin{studybox}
\textbf{정석} $x$축 둘레로 회전시킨 입체의 부피는 $\pi \int_{a}^{b} y^2 dx$
\end{studybox}

\textbf{모범답안} \\
(1) 교점의 $x$좌표는 $x^2+3x-4=0$ 에서 $x=1, -4$. $x \ge 0$ 이므로 $x=1$. 
\[ V = \pi \int_{0}^{1} 3x dx + \pi \int_{1}^{2} (4-x^2) dx = 3\pi \left[ \dfrac{1}{2}x^2 \right]_{0}^{1} + \pi \left[ 4x - \dfrac{1}{3}x^3 \right]_{1}^{2} = \mathbf{\dfrac{19}{6}\pi} \]
(2) $V = \pi \int_{0}^{\pi} y^2 dx = \pi \int_{0}^{\pi} x \sin^2 x dx = \dfrac{\pi}{2} \int_{0}^{\pi} x(1-\cos 2x) dx$ 
\[ = \dfrac{\pi}{2} \left( \left[ \dfrac{1}{2}x^2 \right]_{0}^{\pi} - \left[ \dfrac{x\sin 2x}{2} \right]_{0}^{\pi} + \int_{0}^{\pi} \dfrac{\sin 2x}{2} dx \right) = \dfrac{\pi^3}{4} + \dfrac{\pi}{2} \left[ -\dfrac{\cos 2x}{4} \right]_{0}^{\pi} = \mathbf{\dfrac{\pi^3}{4}} \]

\subsection*{필수 예제 18-5}
다음 곡선과 직선으로 둘러싸인 도형을 $y$축 둘레로 회전시킨 입체의 부피 $V$ 를 구하여라. \\
(1) $y = e^x, y=e, x=0$ \quad (2) $y = \sin x (0 \le x \le \pi/2), y=1, x=0$

\begin{studybox}
\textbf{정석} $y$축 둘레로 회전시킨 입체의 부피는 $\pi \int_{\alpha}^{\beta} x^2 dy$ \\
(2)에서와 같이 $x$를 $y$로 나타내기 힘든 경우 다음 정석을 이용한다. 
\[ \int_{\alpha}^{\beta} x^2 dy = \int_{a}^{b} x^2 g'(x) dx \quad (\text{단, } y=g(x), g(a)=\alpha, g(b)=\beta) \]
\end{studybox}

\textbf{모범답안} \\
(1) $y=e^x \implies x = \ln y$. 
\[ V = \pi \int_{1}^{e} (\ln y)^2 dy = \pi \left( [y(\ln y)^2]_1^e - 2 \int_{1}^{e} \ln y dy \right) = \pi(e - 2[y\ln y - y]_1^e) = \mathbf{\pi(e-2)} \]
(2) $V = \pi \int_{0}^{1} x^2 dy$. $y=\sin x$ 에서 $dy = \cos x dx$. 
\[ V = \pi \int_{0}^{\pi/2} x^2 \cos x dx = \pi ([x^2 \sin x]_0^{\pi/2} - \int_{0}^{\pi/2} 2x \sin x dx) = \dfrac{\pi^3}{4} - 2\pi[ \sin x ]_0^{\pi/2} = \mathbf{\dfrac{\pi^3}{4} - 2\pi} \]

\subsection*{필수 예제 18-6}
(1) 포물선 $y=4-x^2$ 과 직선 $y=x+2$ 로 둘러싸인 도형을 $x$축 둘레로 회전시킨 입체의 부피 $V$ 를 구하여라. \\
(2) 곡선 $y=x^2 (x \ge 0)$ 과 두 직선 $x=0, y=x+2$ 로 둘러싸인 도형을 $y$축 둘레로 회전시킨 입체의 부피 $V$ 를 구하여라.

\begin{studybox}
도형 $ABCD$ 를 회전시킨 입체의 부피는 바깥쪽 곡선을 회전시킨 부피에서 안쪽 곡선을 회전시킨 부피를 뺀 것과 같다. 
\end{studybox}

\textbf{모범답안} \\
(1) 교점은 $x=-2, 1$. $V = \pi \int_{-2}^{1} (4-x^2)^2 dx - \pi \int_{-2}^{1} (x+2)^2 dx = \pi \int_{-2}^{1} (x^4 - 9x^2 - 4x + 12) dx = \mathbf{\dfrac{108}{5}\pi}$ \\
(2) 교점은 $y=4$. $V = \pi \int_{0}^{4} y dy - \pi \int_{2}^{4} (y-2)^2 dy = \pi [\dfrac{1}{2}y^2]_0^4 - \pi [\dfrac{1}{3}(y-2)^3]_2^4 = \mathbf{\dfrac{16}{3}\pi}$

\subsection*{필수 예제 18-7}
곡선 $y=x^2$ 과 직선 $y=mx (m>0)$ 로 둘러싸인 도형을 $x$축 둘레로 회전시킨 입체의 부피를 $V_x$, $y$축 둘레로 회전시킨 입체의 부피를 $V_y$ 라고 하자. \\
(1) $V_x = V_y$ 일 때, 상수 $m$ 의 값을 구하여라. \\
(2) $V_y - V_x$ 의 최댓값과 이때 상수 $m$ 의 값을 구하여라.

\textbf{모범답안} \\
교점은 $(0, 0), (m, m^2)$ 이다. 
$V_x = \pi \int_{0}^{m} (mx)^2 dx - \pi \int_{0}^{m} (x^2)^2 dx = \dfrac{2}{15}\pi m^5$, 
$V_y = \pi \int_{0}^{m^2} y dy - \pi \int_{0}^{m^2} (\dfrac{1}{m}y)^2 dy = \dfrac{\pi}{6} m^4$ \\
(1) $V_x = V_y \implies 12m^5 = 15m^4 \implies \mathbf{m = 5/4}$ \\
(2) $f(m) = V_y - V_x = \dfrac{\pi}{6}m^4 - \dfrac{2}{15}\pi m^5 \implies f'(m) = -\dfrac{2}{3}\pi m^3(m-1)$. 
증감표에 의해 $m=1$ 일 때 \textbf{최댓값} $\mathbf{\pi/30}$ 이다.

\subsection*{필수 예제 18-8}
두 곡선 $y = \sin x, y = \cos x$ 로 둘러싸인 $\dfrac{\pi}{4} \le x \le \dfrac{5}{4}\pi$ 인 도형을 $x$축 둘레로 회전시킨 입체의 부피 $V$ 를 구하여라.

\begin{studybox}
\textbf{정석} 회전축을 포함하는 경우 회전축의 한쪽으로 모은다. 
\end{studybox}

\textbf{모범답안} \\
점 찍은 두 부분에 의하여 생기는 입체의 부피가 같으므로 
$V = 2\pi \left( \int_{\pi/4}^{3\pi/4} \sin^2 x dx - \int_{\pi/4}^{\pi/2} \cos^2 x dx \right) = 2\pi \left( \int_{\pi/4}^{3\pi/4} \dfrac{1-\cos 2x}{2} dx - \int_{\pi/4}^{\pi/2} \dfrac{1+\cos 2x}{2} dx \right)$ 
\[ = 2\pi \left( \left[ \dfrac{x}{2} - \dfrac{\sin 2x}{4} \right]_{\pi/4}^{3\pi/4} - \left[ \dfrac{x}{2} + \dfrac{\sin 2x}{4} \right]_{\pi/4}^{\pi/2} \right) = \mathbf{\dfrac{\pi}{4}(\pi + 6)} \]

\subsection*{필수 예제 18-9}
곡선 $y = \ln x$ 와 원점에서 이 곡선에 그은 접선 및 $x$축으로 둘러싸인 도형을 $F$ 라고 할 때, $x$축 및 $y$축 회전체의 부피 $V_x, V_y$ 를 구하여라.

\textbf{모범답안} \\
접선의 방정식은 $y = \dfrac{1}{e}x$ 이다. \\
(1) $V_x = \pi \int_{0}^{e} (\dfrac{1}{e}x)^2 dx - \pi \int_{1}^{e} (\ln x)^2 dx = \dfrac{1}{3}\pi e - \pi(e - 2) = \mathbf{\dfrac{2}{3}\pi(3-e)}$ \\
(2) $V_y = \pi \int_{0}^{1} (e^y)^2 dy - \pi \int_{0}^{1} (ey)^2 dy = \pi [\dfrac{1}{2}e^{2y}]_0^1 - \pi [e^2 \times \dfrac{1}{3}y^3]_0^1 = \mathbf{\dfrac{\pi}{6}(e^2-3)}$

\subsection*{필수 예제 18-10}
원 $x^2+y^2+4x-6y+12=0$ 을 \\
(1) $x$축 둘레로 회전시킨 부피를 구하여라. \\
(2) 직선 $3x-4y+8=0$ 둘레로 회전시킨 부피를 구하여라.

\textbf{모범답안} \\
원의 방정식은 $(x+2)^2 + (y-3)^2 = 1$ 이다. \\
(1) $V_1 = \pi \int_{-3}^{-1} [\{3+\sqrt{1-(x+2)^2}\}^2 - \{3-\sqrt{1-(x+2)^2}\}^2] dx = 12\pi \int_{-3}^{-1} \sqrt{1-(x+2)^2} dx$ \\
$x+2 = \sin \theta$ 로 치환하여 계산하면 $V_1 = \mathbf{6\pi^2}$ \\
(2) 원의 중심 $(-2, 3)$ 과 직선 사이의 거리는 2이다. 중심이 $(0, 2)$ 이고 반지름이 1인 원의 회전체 부피와 같으므로 $V_2 = \pi \int_{-1}^{1} \{(2+\sqrt{1-X^2})^2 - (2-\sqrt{1-X^2})^2\} dX = 16\pi \int_{0}^{1} \sqrt{1-X^2} dX = \mathbf{4\pi^2}$

\end{document}