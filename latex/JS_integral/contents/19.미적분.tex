\documentclass[a4paper,11pt]{article}
\usepackage{kotex}
\usepackage{amsmath, amssymb, amsthm}
\usepackage{geometry}
\usepackage{tcolorbox}
\usepackage{enumitem}
\tcbuselibrary{skins, breakable}

% --- 여백 및 간격 설정 ---
\geometry{left=2.5cm, right=2.5cm, top=2.5cm, bottom=2.5cm}
\linespread{1.4}
\setlength{\parskip}{1em}
\setlist[enumerate]{itemsep=5pt, topsep=0pt}

% --- 사용자 정의 박스 스타일 ---
\newtcolorbox{conceptbox}[1][]{
  colback=blue!5, colframe=blue!60, coltitle=white, fonttitle=\bfseries,
  title={#1}, arc=1mm, boxrule=0.5mm, breakable, parskip=1em
}
\newtcolorbox{advicebox}{
  colback=orange!5, colframe=orange!60, title={Advice}, fonttitle=\bfseries,
  coltitle=white, arc=1mm, boxrule=0.5mm, breakable, parskip=1em
}
\newtcolorbox{studybox}{
  colback=gray!10, colframe=gray!60, title={정석연구}, fonttitle=\bfseries,
  coltitle=white, arc=1mm, boxrule=0.5mm, breakable, parskip=1em
}

\title{\textbf{미적분 - 제19장 속도·거리와 적분}}
\author{수학의 정석}
\date{}

\begin{document}

\maketitle

\section*{§1. 속도와 거리}

\begin{conceptbox}[기본 정석 - 속도와 거리]
수직선 위를 움직이는 점 P의 시각 $t$ 에서의 속도가 $v(t)$ 일 때, 점 P가 $t=a$ 일 때부터 $t=b$ 일 때까지 움직이면
\begin{itemize}
    \item 점 P의 위치의 변화량 $\Rightarrow \int_{a}^{b} v(t) dt$
    \item 점 P가 움직인 거리 $\Rightarrow \int_{a}^{b} |v(t)| dt$ (점 찍은 부분의 넓이의 합)
\end{itemize}
\end{conceptbox}

\begin{advicebox}
\textbf{1. 속도와 거리} \\
수직선 위를 움직이는 점 P의 시각 $t$ 에서의 위치 $x$ 가 $x=f(t)$ 일 때, 속도 $v(t)$ 는
\[ v(t) = \dfrac{dx}{dt} = f^{\prime}(t) \]
이므로
\[ \int_{t_{0}}^{t} v(t) dt = f(t) - f(t_{0}) \]
이다. 이때, 시각 $t_{0}$ 에서의 점 P의 위치를 $x_{0}$ 이라고 하면 시각 $t$ 에서의 점 P의 위치 $f(t)$ 는
\[ f(t) = f(t_{0}) + \int_{t_{0}}^{t} v(t) dt = x_{0} + \int_{t_{0}}^{t} v(t) dt \]
이다. 따라서 $t=a$ 일 때부터 $t=b$ 일 때까지 점 P의 위치의 변화량은
\[ f(b) - f(a) = \{x_{0} + \int_{t_{0}}^{b} v(t) dt\} - \{x_{0} + \int_{t_{0}}^{a} v(t) dt\} = \int_{a}^{b} v(t) dt \]
이다.
\end{advicebox}

\begin{advicebox}
\textbf{2. 구분구적법을 이용한 속도와 거리의 관계} \\
속도와 거리의 관계를 앞에서 공부한 구분구적법을 이용하여 다음과 같이 생각할 수도 있다. 속도가 일정할 때
\[ (\text{속도}) \times (\text{시간}) = (\text{거리}) \]
이다. 따라서 수직선 위를 움직이는 점 P의 시각 $t$ 에서의 속도가 $v(t)$ 일 때, $v(t)$ 가 연속이고 $v(t) \ge 0$ 이면 점 P가 움직인 거리는 다음과 같이 나타낼 수 있다.
\[ v(t_k) \Delta t \Rightarrow \sum_{k=1}^{n} v(t_k) \Delta t \Rightarrow \lim_{n \to \infty} \sum_{k=1}^{n} v(t_k) \Delta t \Rightarrow \int_{a}^{b} v(t) dt \]
(거리 요소 $\Rightarrow$ 거리 요소의 합 $\Rightarrow$ 한없이 세분한 극한 = 거리)
\end{advicebox}

\section*{§2. 평면 위의 운동}

\begin{conceptbox}[기본 정석 - 곡선의 길이]
함수 $f(x)$ 가 구간 $[a, b]$ 에서 미분가능하고 도함수 $f^{\prime}(x)$ 가 연속이면 구간 $[a, b]$ 에서 곡선 $y=f(x)$ 의 길이 $l$ 은 다음과 같다.
\[ l = \int_{a}^{b} \sqrt{1 + \{f^{\prime}(x)\}^2} dx \]
*Note: $f(x)$ 가 구간 $[a, b]$ 를 포함하는 어떤 열린구간에서 미분가능하면 구간 $[a, b]$ 에서 미분가능하다고 한다.
\end{conceptbox}

\begin{conceptbox}[기본 정석 - 평면 위의 운동]
좌표평면 위를 움직이는 점 P의 시각 $t$ 에서의 위치 $(x, y)$ 가 $x=f(t)$, $y=g(t)$ 이고 구간 $[a, b]$ 에서 $f(t), g(t)$ 가 연속인 도함수를 가지면 이 구간에서 점 P가 움직인 거리 $l$ 은 다음과 같다.
\[ l = \int_{a}^{b} \sqrt{(\dfrac{dx}{dt})^2 + (\dfrac{dy}{dt})^2} dt = \int_{a}^{b} \sqrt{\{f^{\prime}(t)\}^2 + \{g^{\prime}(t)\}^2} dt \]
\end{conceptbox}

\begin{advicebox}
\textbf{1$^{\circ}$ 곡선의 길이} \\
구간 $[a, b]$ 를 $n$ 등분한 점과 양 끝 점을 $x_0(=a), x_1, x_2, \dots, x_n(=b)$ 이라 하고, 이에 대응하는 곡선 위의 점을 각각 $P_0, P_1, P_2, \dots, P_n$ 이라고 하자. 이때,
\[ \overline{P_0P_1} + \overline{P_1P_2} + \overline{P_2P_3} + \dots + \overline{P_{n-1}P_n} \]
은 $n \to \infty$ 일 때 곡선의 길이에 수렴한다. 평균값 정리에 의하여
\[ \dfrac{f(x_k) - f(x_{k-1})}{x_k - x_{k-1}} = f^{\prime}(t_k) \text{ 곧, } f(x_k) - f(x_{k-1}) = f^{\prime}(t_k)(x_k - x_{k-1}) \]
을 만족시키는 $t_k$ 가 구간 $(x_{k-1}, x_k)$ 에 존재한다.
\[ \overline{P_kP_{k-1}}^2 = (x_k - x_{k-1})^2 + \{f(x_k) - f(x_{k-1})\}^2 = [1 + \{f^{\prime}(t_k)\}^2](x_k - x_{k-1})^2 \]
\[ \therefore \overline{P_kP_{k-1}} = \sqrt{1 + \{f^{\prime}(t_k)\}^2}(x_k - x_{k-1}) \]
$n \to \infty$ 일 때 이 합의 극한값은 $\int_{a}^{b} \sqrt{1 + \{f^{\prime}(x)\}^2} dx$ 이고, 이 값이 곡선의 길이이다.
\end{advicebox}

\begin{advicebox}
\textbf{2$^{\circ}$ 평면 위의 운동} \\
$x=f(t), y=g(t)$ 와 같이 매개변수로 나타낸 곡선의 길이 $l$ 은
\[ l = \int_{f(a)}^{f(b)} \sqrt{1 + (\dfrac{dy}{dx})^2} dx = \int_{a}^{b} \sqrt{1 + (\dfrac{dy}{dt} \times \dfrac{dt}{dx})^2} \dfrac{dx}{dt} dt = \int_{a}^{b} \sqrt{(\dfrac{dx}{dt})^2 + (\dfrac{dy}{dt})^2} dt \]
이고, 이것은 좌표평면 위를 움직이는 점 P가 $t=a$ 일 때부터 $t=b$ 일 때까지 움직인 거리와 같다.
\end{advicebox}

\newpage

\section*{필수 예제}

\subsection*{필수 예제 19-1}
수직선 위를 움직이는 점 P의 시각 $t$ 에서의 속도 $v(t)$ 는 $v(t)=(t^2-1)e^{-t}$ 이고, $t=0$ 일 때 점 P는 원점에 있다.\\
(1) $v(t)$ 가 최소가 되는 시각 $t$ 를 구하여라.\\
(2) $t=1$ 일 때의 점 P의 위치를 구하여라.\\
(3) $t=-2$ 일 때부터 $t=2$ 일 때까지 점 P가 움직인 거리를 구하여라.

\begin{studybox}
점 P의 위치의 변화량과 점 P가 움직인 거리를 구분할 수 있어야 한다.
점 P의 위치의 변화량 $\Rightarrow \int_{a}^{b} v(t) dt$, 움직인 거리 $\Rightarrow \int_{a}^{b} |v(t)| dt$
\end{studybox}

\textbf{모범답안} \\
(1) $v^{\prime}(t)=2te^{-t}+(t^2-1)(-e^{-t})=-(t^2-2t-1)e^{-t}$ \\
$v^{\prime}(t)=0$ 에서 $t=1\pm\sqrt{2}$. 증감을 조사하면 $t=1-\sqrt{2}$ 일 때 최소이다. \textbf{답: $t=1-\sqrt{2}$} \\
(2) 시각 $t$ 에서의 점 P의 위치를 $x(t)$ 라고 하자. $t=0$ 일 때 원점에 있으므로
$x(-1) = 0 + \int_{0}^{-1} v(t) dt = \int_{0}^{-1} (t^2-1)e^{-t} dt = [-(t+1)^2 e^{-t}]_0^{-1} = \mathbf{1}$ \\
(3) $l = \int_{-2}^{2} |t^2-1|e^{-t} dt = \int_{-2}^{-1} (t^2-1)e^{-t} dt + \int_{-1}^{1} (1-t^2)e^{-t} dt + \int_{1}^{2} (t^2-1)e^{-t} dt$ \\
$= [-(t+1)^2 e^{-t}]_{-2}^{-1} + [(t+1)^2 e^{-t}]_{-1}^{1} + [-(t+1)^2 e^{-t}]_{1}^{2} = \mathbf{e^2 + 8e^{-1} - 9e^{-2}}$

\subsection*{필수 예제 19-2}
수직선 위를 움직이는 두 점 $P_1, P_2$ 의 시각 $t$ 에서의 속도 $v_1, v_2$ 는 $v_1 = \sin t, v_2 = 1 - \cos t$ 라고 한다. 또, $t=0$ 일 때 두 점은 모두 원점에 있다. 선분 $P_1P_2$ 의 중점을 Q라고 할 때, 다음 물음에 답하여라.\\
(1) 시각 $t$ 에서의 점 Q의 속도 $v(t)$ 를 $v_1, v_2$ 로 나타내어라.\\
(2) 시각 $t$ 에서의 점 Q의 위치를 $t$ 로 나타내어라.\\
(3) $t=0$ 일 때부터 $t=2\pi$ 일 때까지 점 Q가 움직인 거리 $l$ 을 구하여라.

\textbf{모범답안} \\
(1) 위치 $x = \dfrac{1}{2}(x_1 + x_2)$ 이므로 $v(t) = \dfrac{dx}{dt} = \dfrac{1}{2}(\dfrac{dx_1}{dt} + \dfrac{dx_2}{dt}) = \mathbf{\dfrac{1}{2}(v_1 + v_2)}$ \\
(2) $v(t) = \dfrac{1}{2}(\sin t + 1 - \cos t)$ 이므로 $x = \int_{0}^{t} \dfrac{1}{2}(\sin t - \cos t + 1) dt = \mathbf{\dfrac{1}{2}(-\cos t - \sin t + t + 1)}$ \\
(3) $l = \dfrac{1}{2} \int_{0}^{2\pi} |\sqrt{2}\sin(t - \dfrac{\pi}{4}) + 1| dt = \dots = \mathbf{\dfrac{\pi+4}{2}}$

\subsection*{필수 예제 19-3}
수직선 위를 움직이는 두 점 A, B의 시각 $t$ 에서의 속도가 각각 $\sin t, \cos 2t$ 이고, $t=0$ 일 때 점 A는 원점에서, 점 B는 좌표가 1인 점에서 동시에 출발한다.\\
(1) $0 < t \le 2\pi$ 에서 점 A와 B가 만나는 횟수를 구하여라.\\
(2) $0 < t \le 2\pi$ 에서 점 A와 B가 가장 멀어질 때의 시각 $t$ 를 구하여라. 또, 이때 두 점 사이의 거리를 구하여라.

\begin{studybox}
두 점의 위치 $x_A, x_B$ 는 각각 $x_A = x_1 + \int_{0}^{t} v_A dt$, $x_B = x_2 + \int_{0}^{t} v_B dt$ 이다.
\end{studybox}

\textbf{모범답안} \\
(1) $x_A = 1 - \cos t, x_B = 1 + \dfrac{1}{2}\sin 2t$. $x_A = x_B \Rightarrow \cos t(\sin t + 1) = 0$. $t = \dfrac{\pi}{2}, \dfrac{3}{2}\pi$ 이므로 \textbf{2회} \\
(2) $f(t) = x_B - x_A = \dfrac{1}{2}\sin 2t + \cos t$. $f^{\prime}(t) = -(2\sin t - 1)(\sin t + 1) = 0$ 에서 $t = \dfrac{\pi}{6}, \dfrac{5}{6}\pi, \dfrac{3}{2}\pi$. $|f(t)|$ 의 최대는 $t = \dfrac{\pi}{6}, \dfrac{5}{6}\pi$ 일 때 \textbf{거리 $\dfrac{3\sqrt{3}}{4}$}

\subsection*{필수 예제 19-4}
다음 주어진 구간에서 곡선의 길이 $l$ 을 구하여라.\\
(1) $y = \dfrac{1}{2}(e^x + e^{-x}) (-1 \le x \le 1)$ \quad\\
(2) $y = \ln x (1 \le x \le 2)$

\textbf{모범답안} \\
(1) $l = 2 \int_{0}^{1} \sqrt{1 + \dfrac{1}{4}(e^x - e^{-x})^2} dx = \int_{0}^{1} (e^x + e^{-x}) dx = \mathbf{e - \dfrac{1}{e}}$ \\
(2) $l = \int_{1}^{2} \dfrac{\sqrt{x^2+1}}{x} dx = [t + \dfrac{1}{2}\ln \dfrac{t-1}{t+1}]_{\sqrt{2}}^{\sqrt{5}} = \mathbf{\sqrt{5} - \sqrt{2} + \ln \dfrac{(\sqrt{5}-1)(\sqrt{2}+1)}{2}}$

\subsection*{필수 예제 19-5}
$a > 0, 0 \le t \le 2\pi$ 일 때, 매개변수로 나타낸 곡선 $x = a(t - \sin t), y = a(1 - \cos t)$ 의 길이를 구하여라.

\textbf{모범답안} \\
$l = \int_{0}^{2\pi} \sqrt{a^2(1 - \cos t)^2 + a^2 \sin^2 t} dt = a \int_{0}^{2\pi} \sqrt{2(1 - \cos t)} dt = 2a \int_{0}^{2\pi} \sin \dfrac{t}{2} dt = \mathbf{8a}$

\subsection*{필수 예제 19-6}
좌표평면 위를 움직이는 점 P의 시각 $t$ 에서의 위치가 $(e^{-t} \cos t, e^{-t} \sin t)$ 일 때, 다음 물음에 답하여라.\\
(1) 시각 $t$ 에서의 점 P의 속력을 구하여라.\\
(2) $t=0$ 일 때부터 $t=a(a > 0)$ 일 때까지 점 P가 움직인 거리 $l$ 을 구하여라.\\
(3) (2)에서 $a \to \infty$ 일 때, $l$ 의 극한값을 구하여라.

\textbf{모범답안} \\
(1) $\dfrac{dx}{dt} = -e^{-t}(\sin t + \cos t), \dfrac{dy}{dt} = -e^{-t}(\sin t - \cos t)$. 속력은 \textbf{$\sqrt{2}e^{-t}$} \\
(2) $l = \int_{0}^{a} \sqrt{2}e^{-t} dt = \mathbf{\sqrt{2}(1 - e^{-a})}$ \\
(3) $\lim_{a \to \infty} l = \mathbf{\sqrt{2}}$

\end{document}