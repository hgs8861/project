\documentclass[10pt, a4paper]{article} % 글자 크기를 10pt로 축소

% --- 1. 필수 패키지 설정 ---
\usepackage[a4paper, top=18mm, bottom=15mm, left=12mm, right=12mm, headheight=25pt, headsep=8mm]{geometry}
\usepackage{amsmath, amssymb, amsthm}
\usepackage{fancyhdr}
\usepackage{enumitem}
\usepackage{array}
\usepackage{longtable}

% --- 2. 한글 설정 (가장 표준적인 kotex 사용) ---
% 별도의 폰트 파일 없이 시스템 기본 한글 폰트를 사용합니다.
\usepackage{kotex}

% --- 3. 헤더 설정 ---
\pagestyle{fancy}
\fancyhf{}
\lhead{\large \textbf{미적분학 1 : 연습문제}}
\rhead{\small 학번: \underline{\hspace{3.5cm}} 이름: \underline{\hspace{3cm}}}
\cfoot{\small \thepage}
\renewcommand{\headrulewidth}{0.6pt} % 선 두께 0.6pt 유지

% --- 4. 문제 박스 명령어 정의 ---
% 글자 크기를 \small로 지정하여 더 작게 출력
\newcounter{probcount}
\newcommand{\exercise}[1]{%
    \stepcounter{probcount}%
    \begin{minipage}[t][110mm][t]{0.44\textwidth}
        \small % 문제 내부 글자 크기 축소
        \vspace{1mm}
        \noindent \textbf{문제 \arabic{probcount}.} #1
    \end{minipage}%
}

\begin{document}

% --- 5. 문제 입력 섹션 ---
% {p|p} 구조로 중앙 세로줄만 생성
\noindent
\begin{longtable}{p{0.47\textwidth} | p{0.47\textwidth}}
    
    % [1페이지 상단] 문제 1 & 2
    \exercise{다음 부정적분을 구하여라.
    \begin{enumerate}[label=(\arabic*), leftmargin=8mm, nosep]
        \item $\displaystyle \int\dfrac{x}{(2x+1)^{3}}dx$
        \item $\displaystyle \int\dfrac{x}{\sqrt{1-x^{2}}}dx$
        \item $\displaystyle \int x\sin x^{2}dx$
        \item $\displaystyle \int\dfrac{e^{x}-1}{e^{x}+1}dx$
        \item $\displaystyle \int\dfrac{\ln x}{x}dx$
    \end{enumerate}} 
    & 
    \exercise{함수 $f(x), g(x)$ 가 다음 세 조건을 만족시킬 때, $f(x)+g(x)$ 를 구하여라.
    \begin{enumerate}[label=(\text{\roman*}), leftmargin=10mm, nosep]
        \item $f^{\prime}(x)=2g(x), \ g^{\prime}(x)=2f(x)$
        \item $f(0)=1, \ g(0)=e-1$
        \item $f(x)>0, \ g(x)>0$
    \end{enumerate}} \\
    \hline % 페이지 중앙 가로선 (십자 모양 완성)
    
    % [1페이지 하단] 문제 3 & 4
    \exercise{다음 부정적분을 구하여라.
    \begin{enumerate}[label=(\arabic*), leftmargin=8mm, nosep]
        \item $\displaystyle \int\dfrac{x^{4}}{(x-1)^{3}}dx$
        \item $\displaystyle \int\sqrt{1+\sqrt{x}}dx$
    \end{enumerate}} 
    & 
    \exercise{다음 부정적분을 구하여라.
    \begin{enumerate}[label=(\arabic*), leftmargin=8mm, nosep]
        \item $\displaystyle \int\dfrac{1}{\sin x}dx$
        \item $\displaystyle \int\dfrac{\sin x}{1+\sin x}dx$
    \end{enumerate}} \\
    % 페이지 하단 테두리 방지를 위해 \hline 생략

    % --- 2페이지로 자동 전환 ---
    
    \exercise{부정적분 $\displaystyle \int x\cot^{2}x~dx$ 를 구하여라.} 
    & 
    \exercise{다음 등식이 성립함을 증명하여라. (단, $n$ 은 2 이상의 자연수)
    \begin{enumerate}[label=(\arabic*), leftmargin=8mm, nosep]
        \item $\displaystyle \int \tan^{n}x~dx=\dfrac{\tan^{n-1}x}{n-1}-\int \tan^{n-2}x~dx$
        \item $\displaystyle \int \sin^{n}x~dx=-\dfrac{\sin^{n-1}x\cos x}{n}+\dfrac{n-1}{n}\int \sin^{n-2}x~dx$
    \end{enumerate}} \\
    \hline 
    
    \exercise{미분방정식 $y' + y = e^x$의 일반해를 구하시오.} 
    & 
    \exercise{함수 $f(x,y) = x^2 + y^2$의 점 $(1,2)$에서의 기울기 벡터를 구하시오.} \\

\end{longtable}

\end{document}