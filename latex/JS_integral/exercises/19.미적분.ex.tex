\documentclass[10pt, a4paper]{article}

% --- 1. 필수 패키지 설정 ---
\usepackage[a4paper, top=18mm, bottom=15mm, left=12mm, right=12mm, headheight=25pt, headsep=8mm]{geometry}
\usepackage{amsmath, amssymb, amsthm}
\usepackage{fancyhdr}
\usepackage{enumitem}
\usepackage{array}
\usepackage{longtable}
\usepackage{kotex}

% --- 2. 헤더 설정 ---
\pagestyle{fancy}
\fancyhf{}
\lhead{\large \textbf{미적분 : 제19장 속도·거리와 적분 연습문제}}
\rhead{\small 학번: \underline{\hspace{3.5cm}} 이름: \underline{\hspace{3cm}}}
\cfoot{\small \thepage}
\renewcommand{\headrulewidth}{0.6pt}

% --- 3. 문제 박스 명령어 정의 ---
\newcounter{probcount}
\newcommand{\exercise}[1]{%
    \stepcounter{probcount}%
    \begin{minipage}[t][120mm][t]{0.44\textwidth}
        \small 
        \vspace{1mm}
        \noindent \textbf{문제 \arabic{probcount}.} #1
    \end{minipage}%
}

\begin{document}

\noindent
\begin{longtable}{p{0.47\textwidth} | p{0.47\textwidth}}
    
    % [기본 문제 섹션 - 4문항 선정 (전체 8문항의 절반)]
    \exercise{수직선 위를 움직이는 점 P의 시각 $t$ 에서의 속도 $v(t)$ 는 $v(t) = 30 - 3t - \sqrt{t}$ 라고 한다. $t=0$ 일 때부터 점 P의 속도가 0이 될 때까지 점 P가 움직인 거리를 구하여라.} 
    & 
    \exercise{수직선 위를 움직이는 점 P의 시각 $t$ 에서의 속도 $v(t)$ 는 $v(t) = \cos t + \cos 2t$ 이고, $t=0$ 일 때 점 P는 원점에 있다. 
    \begin{enumerate}[label=(\arabic*), leftmargin=8mm, nosep]
        \item 시각 $t$ 에서의 점 P의 위치를 $t$ 로 나타내어라.
        \item $t=0$ 일 때부터 $t=\pi$ 일 때까지 점 P가 움직인 거리를 구하여라.
    \end{enumerate}} \\
    \hline 
    
    \exercise{다음 주어진 구간에서 곡선의 길이를 구하여라.
    \begin{enumerate}[label=(\arabic*), leftmargin=8mm, nosep]
        \item $y = x\sqrt{x} (0 \le x \le 2)$
        \item $y = \ln(\sec x) (0 \le x \le \dfrac{\pi}{4})$
    \end{enumerate}} 
    & 
    \exercise{곡선 $y=f(x)$ 위의 점 $(0, 1)$ 부터 곡선 위의 임의의 점 $(x, y)$ 까지 곡선의 길이가 $e^{2x} + y - 2$ 일 때, $f(x)$ 를 구하여라. (단, $f(x)$ 는 $x \ge 0$ 에서 정의되고, $x > 0$ 에서 미분가능한 함수이다.)} \\
    \hline

    % [실력 문제 섹션 - 3문항 선정 (전체 6문항의 절반)]
    \exercise{$t=0$ 일 때 동시에 원점을 출발하여 수직선 위를 움직이는 점 P, Q의 시각 $t$ 에서의 속도가 각각 $\sin \pi t, 2\sin 2\pi t$ 라고 한다. 원점을 출발한 후 처음으로 두 점이 만날 때까지 점 P, Q가 움직인 거리를 각각 구하여라.} 
    & 
    \exercise{좌표평면 위를 움직이는 점 P의 시각 $t(t \ge 0)$ 에서의 위치 $(x, y)$ 가 $x = \int_{0}^{t} \theta \cos \theta d\theta, y = \int_{0}^{t} \theta \sin \theta d\theta$ 일 때, 다음 물음에 답하여라.
    \begin{enumerate}[label=(\arabic*), leftmargin=8mm, nosep]
        \item $t=0$ 일 때부터 $t=2\pi$ 일 때까지 점 P가 움직인 거리를 구하여라.
        \item $0 \le t \le 2\pi$ 일 때, 선분 OP의 길이의 최댓값을 구하여라. (단, O는 원점이다.)
    \end{enumerate}} \\
    \hline

    \exercise{실수 전체의 집합에서 이계도함수를 가지는 함수 $f(t)$ 에 대하여 좌표평면 위를 움직이는 점 P의 시각 $t$ 에서의 위치 $(x, y)$ 가 $x = 4\cos t, y = f(t)$ 이다. 점 P가 $t=0$ 일 때부터 $t=s(s > 0)$ 일 때까지 움직인 거리는 $\dfrac{9}{2}s - \dfrac{\sin 2s}{4}$ 이고, $t=\pi$ 일 때 점 P의 속도는 $(0, 4)$ 이다. $t=\dfrac{\pi}{4}$ 일 때, 점 P의 가속도의 크기를 구하여라.} 
    & 
    \\ 

\end{longtable}

\end{document}