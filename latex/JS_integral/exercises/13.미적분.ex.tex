\documentclass[a4paper,11pt]{article}
\usepackage{kotex}
\usepackage{amsmath, amssymb}
\usepackage{geometry}
\usepackage{enumitem}

% --- 여백 및 간격 설정 ---
\geometry{left=2.5cm, right=2.5cm, top=2.5cm, bottom=2.5cm}
\linespread{1.3}
\setlength{\parskip}{1em}

\title{\textbf{미적분 - Chapter 13 연습문제}}
\date{}

\begin{document}
\maketitle

\section*{기본 문제}

\begin{enumerate}
    \item \textbf{Exercise 13-1}\\
    $f(x)=\int(x \ln x+e^{x}+x+1)dx$ 일 때, 다음 극한값을 구하여라.
    \begin{enumerate}[label=(\arabic*)]
        \item $\displaystyle \lim_{h\rightarrow0}\frac{f(1+h)-f(1)}{h}$
        \item $\displaystyle \lim_{h\rightarrow0}\frac{f(1+h+h^{2})-f(1-h)}{h}$
    \end{enumerate}

    \item \textbf{Exercise 13-3}\\
    다음 부정적분을 구하여라.
    \begin{enumerate}[label=(\arabic*)]
        \item $\displaystyle \int\frac{3 \sin^{3}x-3 \sin x+\cos^{3}x-2}{\cos^{2}x}dx$
        \item $\displaystyle \int\frac{1}{\tan(x/2)+\cot(x/2)}dx$
    \end{enumerate}
\end{enumerate}

\section*{실력 문제}

\begin{enumerate}
    \item \textbf{Exercise 13-5}\\
    $x>0$에서 정의된 연속함수 $f(x)$ 에 대하여 함수 $(x-1)f(x)$ 의 도함수는 $\frac{1}{x}$ 이다. 이때, $f(x)$ 를 구하여라.

    \item \textbf{Exercise 13-8}\\
    미분가능한 함수 $f(x)$ 가 모든 실수 $x$에 대하여 $f'(x)=-f(x)+e^{-x}\cos x$ 를 만족시킨다. $g(x)=e^{x}f(x)$ 이고 $f(0)=1$ 일 때, 다음 물음에 답하여라.
    \begin{enumerate}[label=(\arabic*)]
        \item $g''(x)$ 를 구하여라.
        \item $f(x)$ 를 구하여라.
    \end{enumerate}

    \item \textbf{Exercise 13-9}\\
    미분가능한 함수 $f(x)$ 가 모든 실수 $x$에 대하여 
    \[ f(x)=f(-x)+2x, \quad f(x)f'(x)+f(-x)f'(-x)=6 \cos x+2 \]
    를 만족시킬 때, $f(x)$ 를 구하여라.
\end{enumerate}

\end{document}