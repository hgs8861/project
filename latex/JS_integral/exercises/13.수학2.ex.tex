\documentclass[10pt, a4paper]{article}

% --- 1. 필수 패키지 설정 ---
\usepackage[a4paper, top=18mm, bottom=15mm, left=12mm, right=12mm, headheight=25pt, headsep=8mm]{geometry}
\usepackage{amsmath, amssymb, amsthm}
\usepackage{fancyhdr}
\usepackage{enumitem}
\usepackage{array}
\usepackage{longtable}
\usepackage{kotex}

% --- 2. 헤더 설정 ---
\pagestyle{fancy}
\fancyhf{}
\lhead{\large \textbf{수학2: 제13장 정적분의 계산 연습문제}}
\rhead{\small 학번: \underline{\hspace{3.5cm}} 이름: \underline{\hspace{3cm}}}
\cfoot{\small \thepage}
\renewcommand{\headrulewidth}{0.6pt}

% --- 3. 문제 박스 명령어 정의 ---
\newcounter{probcount}
\newcommand{\exercise}[1]{%
    \stepcounter{probcount}%
    \begin{minipage}[t][110mm][t]{0.44\textwidth}
        \small 
        \vspace{1mm}
        \noindent \textbf{문제 \arabic{probcount}.} #1
    \end{minipage}% 
    }

\begin{document}

\noindent
\begin{longtable}{p{0.47\textwidth} | p{0.47\textwidth}}
    
    % [기본 문제 섹션 - 6문항 선별]
    \exercise{원점을 출발하여 수직선 위를 움직이는 점 P의 시각 $t$에서의 속도 $v(t)$가 $v(t)=4-2t$일 때, $t=0$에서 $t=4$까지 점 P가 움직인 거리를 구하여라.} 
    & 
    \exercise{    지면에서 $30\text{m/s}$의 속도로 똑바로 위로 던진 물체의 $t$초 후의 속도가 $v(t)=30-10t$이다. 이 물체가 지면에 떨어질 때까지 움직인 총 거리를 구하여라.} \\
    \hline

    \exercise{수직선 위를 움직이는 두 점 P, Q의 시각 $t$에서의 속도가 각각 $v_P=4t^3-12t, v_Q=2t$이다. $t=0$일 때 두 점 P, Q의 위치가 각각 $10, 2$라면, 두 점이 만나는 시각 $t$를 구하여라.} 
    & 
    \exercise{시각 $t=0$일 때 동시에 원점을 출발하여 수직선 위를 움직이는 두 점 P, Q의 시각 $t$에서의 속도가 각각 $v_P = 1-2t, v_Q = 3t^2-1$이다. 출발 후 두 점 P, Q가 다시 만날 때까지 두 점 사이의 거리의 최댓값을 구하여라.} \\
    
    \exercise{원점을 출발하여 수직선 위를 움직이는 점 P의 시각 $t$에서의 속도 $v(t)$의 그래프가 아래와 같을 때, $t=0$에서 $t=6$까지 점 P가 실제로 움직인 거리를 구하여라. (그래프는 $t=2, 4$에서 $t$축과 만나는 직선들로 구성됨)} 
    & 
    \exercise{    가속도가 $a(t)=6t-12$인 물체가 $t=0$일 때 속도 $v_0=9$로 원점을 출발하였다. 이 물체가 다시 원점을 통과하는 시각 $t$를 구하여라.} \\
    \hline

    %
\end{longtable}

\end{document}