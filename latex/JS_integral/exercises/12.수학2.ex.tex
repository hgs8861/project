\documentclass[10pt, a4paper]{article}

% --- 1. 필수 패키지 설정 ---
\usepackage[a4paper, top=18mm, bottom=15mm, left=12mm, right=12mm, headheight=25pt, headsep=8mm]{geometry}
\usepackage{amsmath, amssymb, amsthm}
\usepackage{fancyhdr}
\usepackage{enumitem}
\usepackage{array}
\usepackage{longtable}
\usepackage{kotex}

% --- 2. 헤더 설정 ---
\pagestyle{fancy}
\fancyhf{}
\lhead{\large \textbf{수학2: 제12장 정적분의 계산 연습문제}}
\rhead{\small 학번: \underline{\hspace{3.5cm}} 이름: \underline{\hspace{3cm}}}
\cfoot{\small \thepage}
\renewcommand{\headrulewidth}{0.6pt}

% --- 3. 문제 박스 명령어 정의 ---
\newcounter{probcount}
\newcommand{\exercise}[1]{%
    \stepcounter{probcount}%
    \begin{minipage}[t][110mm][t]{0.44\textwidth}
        \small 
        \vspace{1mm}
        \noindent \textbf{문제 \arabic{probcount}.} #1
    \end{minipage}% 
    }

\begin{document}

\noindent
\begin{longtable}{p{0.47\textwidth} | p{0.47\textwidth}}
    
    % [기본 문제 섹션 - 6문항 선별]
    \exercise{다음 그래프로 둘러싸인 도형의 넓이를 구하여라.
    \begin{enumerate}
        \item $y=|x^3+3x^2-x-3|$, $y=0$
        \item $y=|x^2-1|$, $y=1$
    \end{enumerate}
} 
    & 
    \exercise{$f(x)=x^3-3x+\int_0^2 f(t) dt$를 만족시키는 다항함수 $f(x)$에 대하여 $y=f(x)$의 그래프와 $x$축으로 둘러싸인 도형의 넓이를 구하여라.} \\
    \hline

    \exercise{이차함수 $f(x)=-\dfrac{1}{4}(x+1)^2+4$의 그래프와 $y$축 및 직선 $y=k$ ($0 < k < 15/4$)로 둘러싸인 제1사분면의 넓이를 $S_1$, 곡선 $y=f(x)$와 $x$축과의 교점 $A$를 지나고 $x$축에 수직인 직선 $l$ 및 직선 $y=k$로 둘러싸인 넓이를 $S_2$라 할 때, $S_1=S_2$가 되도록 하는 $k$의 값을 구하여라.} 
    & 
    \exercise{양수 $r$에 대하여 두 곡선 $y=x^2-2, y=-x^2+\dfrac{2}{r^2}$로 둘러싸인 넓이를 $S_r$이라 할 때, $\lim_{r \to \infty} S_r$의 값을 구하여라.} \\
    
    \exercise{함수 $f(x)=x^3-6x^2$의 그래프가 $x$축과 만나는 점 중 원점이 아닌 점을 $A$라 하자. 점 $A$를 지나고 $x$축에 수직인 직선 위의 점 $B$와 $y$축 위의 점 $C$에 대하여 직사각형 $OABC$의 넓이가 곡선과 $x$축 사이의 넓이와 같을 때, 직선 $BC$가 항상 지나는 점의 좌표를 구하여라. (단, $B, C$의 $y$좌표는 음수)} 
    & 
    \exercise{곡선 $y=-3x(x+1)$의 $x \le 0$인 부분과 곡선 $x=-3y(y+1)$의 $y \le 0$인 부분으로 둘러싸인 도형의 넓이를 구하여라.} \\
    \hline

    
    \exercise{    곡선 $y=|x|x-2|-3|$과 직선 $y=x+1$로 둘러싸인 도형의 넓이를 구하여라.} 
    & 
    \exercise{    세 곡선 $y^2-x+1=0, y^2-4x+16=0, y^2-9x+81=0$의 제1사분면에 있는 부분으로 둘러싸인 도형의 넓이를 구하여라.} \\
    
    \exercise{    곡선 $y=x^2$ 위의 점 $P(a, a^2)$에서의 접선과 곡선 $y=-x^2$ 위의 점 $Q(a, -a^2)$에서의 접선이 점 $R$에서 수직으로 만날 때, 두 곡선과 선분 $PR, QR$로 둘러싸인 도형의 넓이를 구하여라. (단, $a>0$)} 
    & 
    \exercise{    두 점 $A(3, 0), B(0, 2)$에 대하여 삼각형 $OAB$의 내부를 곡선 $y=\sqrt{ax}$ ($a>0$)가 나누는 두 부분의 넓이를 $S_1, S_2$라 하자. $S_1:S_2 = 5:7$일 때, 상수 $a$의 값을 구하여라.} \\
    \hline

      
    \exercise{포물선 $y=ax^2$ ($a>0$) 위의 점 $P(1, a)$에서의 법선이 $y$축과 만나는 점을 $Q$라 하자. 선분 $PQ$와 $y$축 및 포물선으로 둘러싸인 넓이를 $S(a)$라 할 때, $S(a)$의 최솟값을 구하여라.} 
    & 
    \exercise{    점 $(1, 2)$를 지나고 기울기가 $k$인 직선이 포물선 $y=x^2$과 만나는 두 점 $P, Q$에서의 접선과 포물선으로 둘러싸인 넓이를 $S(k)$라 하자. $-3 \le k \le 3$일 때, $S(k)$의 최댓값과 최솟값을 구하여라.} \\
    
    \exercise{    곡선 $y=x^2$ 위의 점 $P(a, a^2)$에서의 접선과 곡선 $y=-x^2$ 위의 점 $Q(a, -a^2)$에서의 접선이 점 $R$에서 수직으로 만날 때, 두 곡선과 선분 $PR, QR$로 둘러싸인 도형의 넓이를 구하여라. (단, $a>0$)} 
    & 
    \exercise{    두 점 $A(3, 0), B(0, 2)$에 대하여 삼각형 $OAB$의 내부를 곡선 $y=\sqrt{ax}$ ($a>0$)가 나누는 두 부분의 넓이를 $S_1, S_2$라 하자. $S_1:S_2 = 5:7$일 때, 상수 $a$의 값을 구하여라.} \\
    \hline

    %
\end{longtable}

\end{document}