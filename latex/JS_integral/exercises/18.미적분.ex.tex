\documentclass[10pt, a4paper]{article}

% --- 1. 필수 패키지 설정 ---
\usepackage[a4paper, top=18mm, bottom=15mm, left=12mm, right=12mm, headheight=25pt, headsep=8mm]{geometry}
\usepackage{amsmath, amssymb, amsthm}
\usepackage{fancyhdr}
\usepackage{enumitem}
\usepackage{array}
\usepackage{longtable}
\usepackage{kotex}

% --- 2. 헤더 설정 ---
\pagestyle{fancy}
\fancyhf{}
\lhead{\large \textbf{미적분 : 제18장 부피와 적분 연습문제}}
\rhead{\small 학번: \underline{\hspace{3.5cm}} 이름: \underline{\hspace{3cm}}}
\cfoot{\small \thepage}
\renewcommand{\headrulewidth}{0.6pt}

% --- 3. 문제 박스 명령어 정의 ---
\newcounter{probcount}
\newcommand{\exercise}[1]{%
    \stepcounter{probcount}%
    \begin{minipage}[t][120mm][t]{0.44\textwidth}
        \small 
        \vspace{1mm}
        \noindent \textbf{문제 \arabic{probcount}.} #1
    \end{minipage}%
}

\begin{document}

\noindent
\begin{longtable}{p{0.47\textwidth} | p{0.47\textwidth}}
    
    % [기본 문제 섹션 - 5문항 선정]
    \exercise{반지름의 길이가 3인 반원 모양의 판자 두 개가 지름 AB를 공유하고, 두 판자가 이루는 각의 크기는 $60^\circ$ 이다. 지름 AB 위의 점 P를 지나고 직선 AB에 수직인 평면이 두 반원의 호와 만나는 점을 각각 Q, R라고 하자. 점 P가 점 A부터 점 B까지 움직일 때, 삼각형 PQR에 의하여 생기는 입체의 부피를 구하여라.} 
    & 
    \exercise{$0 < k < 1$ 일 때, 곡선 $y = \sin x - k$ 와 세 직선 $x = 0, x = \pi, y = 0$ 으로 둘러싸인 도형을 밑면으로 하는 입체를 $x$축에 수직인 평면으로 자른 단면은 모두 정사각형이다. 이 입체의 부피가 최소가 되는 상수 $k$ 의 값을 구하여라.} \\
    \hline 
    
    \exercise{다음 곡선과 $x$축으로 둘러싸인 도형을 $x$축 둘레로 회전시킨 입체의 부피를 구하여라.
    \begin{enumerate}[label=(\arabic*), leftmargin=8mm, nosep]
        \item $y = x^4 - 2x^2 + 1$
        \item $y = e^x \sqrt{1-x^2}$
    \end{enumerate}} 
    & 
    \exercise{구간 $[0, 2\pi]$ 에서 곡선 $y = x + 2\sin x$ 와 직선 $y = x$ 로 둘러싸인 도형을 $x$축 둘레로 회전시킨 입체의 부피를 구하여라.} \\
   

    \exercise{오른쪽 그림은 반지름의 길이가 1인 반원이고, 현 AP, AQ와 지름 AB가 이루는 각의 크기는 각각 $\pi/6, \pi/3$ 이다. 이때, 현 AP, AQ와 호 QP로 둘러싸인 도형을 지름 AB 둘레로 회전시킨 입체의 부피를 구하여라.} 
    & 
    % [실력 문제 섹션 - 4문항 선정]
    \exercise{두 곡선 $y=x^2, y=-x^2+2x$ 로 둘러싸인 도형을 밑면으로 하는 입체를 $x$축에 수직인 평면으로 자른 단면이 모두 반원일 때, 이 입체의 부피를 구하여라.} \\
    \hline

    \exercise{밑면의 반지름의 길이가 $a$ 인 두 원기둥이 서로 수직으로 만날 때, 겹쳐진 부분의 부피를 구하여라.} 
    & 
    \exercise{한 모서리의 길이가 2인 정육면체 ABCD-EFGH를 밑면의 대각선 AC 둘레로 회전시킬 때 생기는 입체의 부피를 구하여라.} \\
    

    \exercise{포물선 $y = x^2 - 1$ 과 $x$축으로 둘러싸인 도형을 직선 $y = 3$ 둘레로 회전시킨 입체의 부피를 구하여라.} 
    & 
    \\ 

\end{longtable}

\end{document}