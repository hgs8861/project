\documentclass[10pt, a4paper]{article}

% --- 1. 필수 패키지 설정 ---
\usepackage[a4paper, top=18mm, bottom=15mm, left=12mm, right=12mm, headheight=25pt, headsep=8mm]{geometry}
\usepackage{amsmath, amssymb, amsthm}
\usepackage{fancyhdr}
\usepackage{enumitem}
\usepackage{array}
\usepackage{longtable}
\usepackage{kotex}

% --- 2. 헤더 설정 ---
\pagestyle{fancy}
\fancyhf{}
\lhead{\large \textbf{수학2: 제12장 정적분의 계산 연습문제}}
\rhead{\small 학번: \underline{\hspace{3.5cm}} 이름: \underline{\hspace{3cm}}}
\cfoot{\small \thepage}
\renewcommand{\headrulewidth}{0.6pt}

% --- 3. 문제 박스 명령어 정의 ---
\newcounter{probcount}
\newcommand{\exercise}[1]{%
    \stepcounter{probcount}%
    \begin{minipage}[t][110mm][t]{0.44\textwidth}
        \small 
        \vspace{1mm}
        \noindent \textbf{문제 \arabic{probcount}.} #1
    \end{minipage}% 
    }

\begin{document}

\noindent
\begin{longtable}{p{0.47\textwidth} | p{0.47\textwidth}}
    
    % [기본 문제 섹션 - 6문항 선별]
    \exercise{$f(x) = \int_{-1}^{x} |(t+2)(t-1)| dt$ 일 때, 다음 극한값을 구하여라.
    \begin{enumerate}[label=(\arabic*)]
        \item $\displaystyle \lim_{h \to 0} \frac{f(3+2h) - f(3)}{h}$
        \item $\displaystyle \lim_{h \to 0} \frac{f(x+h) - f(x-h)}{h}$
    \end{enumerate}
} 
    & 
    \exercise{함수 $f(x) = \int_{0}^{x} (t-1)(t-2) dt$ 가 증가하는 $x$의 범위를 구하여라.} \\
    \hline

    \exercise{$\displaystyle \int_{1}^{x} f(t) dt = x^3 + ax^2 - 2$ 를 만족시키는 다항함수 $f(x)$와 상수 $a$를 구하여라.} 
    & 
    \exercise{함수 $f(x) = x^2 + ax + b$ 에 대하여 $\displaystyle \frac{d}{dx} \int_{0}^{x} f(t) dt = \int_{1}^{x} f'(t) dt$ 가 성립할 때, $a$의 값을 구하여라.} \\
    
    \exercise{$x \ge -1$ 일 때, 함수 $f(x) = \int_{-1}^{x} |t|(1-t) dt$ 의 최댓값을 구하여라.} 
    & 
    \exercise{모든 실수 $x$에 대하여 $f(x) = x^2 + \int_{0}^{x} (t-x)g(t) dt$ 를 만족시키는 다항함수 $f(x), g(x)$가 있다. $f(x)$가 $(x-2)^2$으로 나누어떨어질 때, $g(2)$의 값을 구하여라.} \\
    \hline

    
    \exercise{연속함수 $f(x)$가 모든 실수 $x$에 대하여 $f(x) = x^2 - 2x + \int_{0}^{2} |x-t|f(t) dt$ 를 만족시킬 때, $f(1)$의 값을 구하여라.} 
    & 
    \exercise{삼차함수 $f(x)$에 대하여 $g(x) = \int_{0}^{x} f(t) dt$ 라 하자. $y=f(x)$의 그래프가 $x$축과 서로 다른 세 점 $(0,0), (\alpha, 0), (\beta, 0)$에서 만날 때, $g(x)$가 극값을 갖지 않기 위한 조건을 구하여라.} \\
    %
\end{longtable}

\end{document}