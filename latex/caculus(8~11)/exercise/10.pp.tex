\documentclass[12pt, a4paper]{article}

% Required Packages
\usepackage{amsmath, graphicx}
\usepackage{amssymb}
\usepackage{kotex} % For Korean text
\usepackage{geometry}
\geometry{a4paper, margin=1in}
\graphicspath{{graph/}}
% Title
\title{Calculus - Chapter 10 Review Selected Exercises}
\author{}
\date{}

\begin{document}
\maketitle
\hrulefill
\vspace{1em}

\subsection*{Problems Plus}
\begin{enumerate}
    \item \textbf{(1)} The outer circle in the figure has radius 1 and the centers of the interior circular arcs lie on the outer circle. Find the area of the shaded region.
    
    \item \textbf{(2)} (a) Find the highest and lowest points on the curve $x^4 + y^4 = x^2 + y^2$.
    (b) Sketch the curve. (Notice that it is symmetric with respect to both axes and both of the lines $y = \pm x$, so it suffices to consider $y \ge x \ge 0$ initially.)
    (c) Use polar coordinates and a computer algebra system to find the area enclosed by the curve.

    \item \textbf{(3)} What is the smallest viewing rectangle that contains every member of the family of polar curves $r = 1 + c \sin \theta$, where $0 < c \le 1$? Illustrate your answer by graphing several members of the family in this viewing rectangle.

    \item \textbf{(4)} Four bugs are placed at the four corners of a square with side length a. The bugs crawl counterclockwise at the same speed and each bug crawls directly toward the next bug at all times. They approach the center of the square along spiral paths.
    (a) Find a polar equation of a bug’s path assuming the pole is at the center of the square. (Use the fact that the line joining one bug to the next is tangent to the bug’s path.)
    (b) Find the distance traveled by a bug by the time it meets the other bugs at the center.

    \item \textbf{(5)} Show that any tangent line to a hyperbola touches the hyperbola halfway between the points of intersection of the tangent and the asymptotes.
    
    \item \textbf{(6)} A circle C of radius 2r has its center at the origin. A circle of radius r rolls without slipping in the counterclockwise direction around C. A point P is located on a fixed radius of the rolling circle at a distance b from its center, $0 < b \le r$.
    (a) Using $\theta$ as a parameter, show that parametric equations of the path traced out by P are
    \[ x = b \cos 3\theta + 3r \cos \theta, \quad y = b \sin 3\theta + 3r \sin \theta \]
    Note: If b = 0, the path is a circle of radius 3r; if b = r, the path is an epicycloid. The path traced out by P for $0 < b < r$ is called an epitrochoid.
    (b) Graph the curve for various values of b between 0 and r.
\end{enumerate}

\end{document}

