\documentclass[12pt]{article}

% Required Packages
\usepackage{amsmath}
\usepackage{amssymb}
\usepackage{kotex} % 한국어 사용을 위한 패키지
\usepackage{geometry}
\geometry{a4paper, margin=1in} % 페이지 여백 설정

% Title
\title{Chapter 11.9 Exercises: Representations of Functions as Power Series}
\author{James Stewart, Calculus, Metric Edition}
\date{}

\begin{document}
\maketitle
\hrulefill % 제목 아래 수평선
\vspace{1em} % 제목과 내용 사이 간격

% --- 난이도 하 (Easy) ---
\subsection*{Difficulty: Easy (6 Problems)}
\begin{enumerate}
    \item \textbf{Exercise 3:} Find a power series representation for the function and determine the interval of convergence.
    \[ f(x) = \frac{1}{1+x} \]
    
    \item \textbf{Exercise 4:} Find a power series representation for the function and determine the interval of convergence.
    \[ f(x) = \frac{x}{1+x} \]
    
    \item \textbf{Exercise 7:} Find a power series representation for the function and determine the interval of convergence.
    \[ f(x) = \frac{2}{3-x} \]
    
    \item \textbf{Exercise 16(a):} Use Equation 1 to find a power series representation for $f(x) = \ln(1-x)$. What is the radius of convergence?
    
    \item \textbf{Exercise 27:} Evaluate the indefinite integral as a power series. What is the radius of convergence?
    \[ \int \frac{t}{1-t^8} dt \]
    
    \item \textbf{Exercise 28:} Evaluate the indefinite integral as a power series. What is the radius of convergence?
    \[ \int \frac{t}{1+t^3} dt \]
\end{enumerate}

\hrulefill
\vspace{1em}

% --- 난이도 중 (Medium) ---
\subsection*{Difficulty: Medium (12 Problems)}
\begin{enumerate}
    \setcounter{enumi}{6} % 이전 목록 번호 이어서 시작
    \item \textbf{Exercise 9:} Find a power series representation for the function and determine the interval of convergence.
    \[ f(x) = \frac{x^2}{x^4+16} \]
    
    \item \textbf{Exercise 13:} Express the function as the sum of a power series by first using partial fractions. Find the interval of convergence.
    \[ f(x) = \frac{2x-4}{x^2-4x+3} \]
    
    \item \textbf{Exercise 15(a):} Use differentiation to find a power series representation for
    \[ f(x) = \frac{1}{(1+x)^2} \]
    What is the radius of convergence?
    
    \item \textbf{Exercise 17:} Find a power series representation for the function and determine the radius of convergence.
    \[ f(x) = \frac{x}{(1+4x)^2} \]
    
    \item \textbf{Exercise 21:} Find a power series representation for the function and determine the radius of convergence.
    \[ f(x) = \ln(5-x) \]
    
    \item \textbf{Exercise 22:} Find a power series representation for the function and determine the radius of convergence.
    \[ f(x) = x^2 \tan^{-1}(x^3) \]
    
    \item \textbf{Exercise 31:} Use a power series to approximate the definite integral to six decimal places.
    \[ \int_{0}^{0.3} \frac{x}{1+x^3} dx \]
    
    \item \textbf{Exercise 36:} Use the result of Example 5 to compute $\ln 1.1$ correct to four decimal places.
    
    \item \textbf{Exercise 40:} The Bessel function of order 1 is defined by
    \[ J_1(x) = \sum_{n=0}^{\infty} \frac{(-1)^n x^{2n+1}}{n!(n+1)! 2^{2n+1}} \]
    \begin{enumerate}
        \item[(a)] Find the domain of $J_1$.
        \item[(b)] Show that $J_1$ satisfies the differential equation
        \[ x^2 J_1''(x) + x J_1'(x) + (x^2 - 1)J_1(x) = 0 \]
        \item[(c)] Show that $J_0'(x) = -J_1(x)$.
    \end{enumerate}
    
    \item \textbf{Exercise 49:} Use the power series for $\tan^{-1} x$ to prove the following expression for $\pi$ as the sum of an infinite series:
    \[ \pi = 2\sqrt{3} \sum_{n=0}^{\infty} \frac{(-1)^n}{(2n+1)3^n} \]
        
    \item \textbf{Exercise 50(a):} By completing the square, show that
    \[ \int_0^{1/2} \frac{dx}{x^2 - x + 1} = \frac{\pi}{3\sqrt{3}} \]
\end{enumerate}

\hrulefill
\vspace{1em}

% --- 난이도 상 (Hard) ---
\subsection*{Difficulty: Hard (6 Problems)}
\begin{enumerate}
    \setcounter{enumi}{18} % 이전 목록 번호 이어서 시작
    \item \textbf{Exercise 39(a):} Show that $J_0$ (the Bessel function of order 0 given in Example 8) satisfies the differential equation
    \[ x^2 J_0''(x) + x J_0'(x) + x^2 J_0(x) = 0 \]
    
    \item \textbf{Exercise 42:} If $f(x) = \sum_{n=0}^{\infty} c_n x^n$, where $c_{n+4} = c_n$ for all $n \ge 0$, find the interval of convergence of the series and a formula for $f(x)$.
    
    \item \textbf{Exercise 43:} A function $f$ is defined by
    \[ f(x) = 1 + 2x + x^2 + 2x^3 + x^4 + \dots \]
    that is, its coefficients are $c_{2n} = 1$ and $c_{2n+1} = 2$ for all $n \ge 0$. Find the interval of convergence of the series and find an explicit formula for $f(x)$.
    
    \item \textbf{Exercise 45:} Let
    \[ f(x) = \sum_{n=1}^{\infty} \frac{x^n}{n^2} \]
    Find the intervals of convergence for $f$, $f'$, and $f''$.
    
    \item \textbf{Exercise 46(c):} Find the sum of each of the following series.
    \[ (i) \sum_{n=2}^{\infty} n(n-1)x^n, \quad |x|<1 \qquad (ii) \sum_{n=2}^{\infty} \frac{n^2-n}{2^n} \qquad (iii) \sum_{n=1}^{\infty} \frac{n^2}{2^n} \]
    
    \item \textbf{Exercise 50(b):} By factoring $x^3+1$ as a sum of cubes, rewrite the integral in part (a). Then express $1/(x^3+1)$ as the sum of a power series and use it to prove the following formula for $\pi$:
    \[ \pi = \frac{3\sqrt{3}}{4} \sum_{n=0}^{\infty} \frac{(-1)^n}{8^n} \left( \frac{2}{3n+1} + \frac{1}{3n+2} \right) \]
\end{enumerate}

\end{document}